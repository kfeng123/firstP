\documentclass[review]{elsarticle}
 
\usepackage{lineno,hyperref}
\modulolinenumbers[5]

\journal{Journal of \LaTeX\ Templates}

%%%%%%%%%%%%%%%%%%%%%%%
%% Elsevier bibliography styles
%%%%%%%%%%%%%%%%%%%%%%%
%% To change the style, put a % in front of the second line of the current style and
%% remove the % from the second line of the style you would like to use.
%%%%%%%%%%%%%%%%%%%%%%%

%% Numbered
%\bibliographystyle{model1-num-names}

%% Numbered without titles
%\bibliographystyle{model1a-num-names}

%% Harvard
%\bibliographystyle{model2-names.bst}\biboptions{authoryear}

%% Vancouver numbered
%\usepackage{numcompress}\bibliographystyle{model3-num-names}

%% Vancouver name/year
\usepackage{numcompress}\bibliographystyle{model4-names}\biboptions{authoryear}

%% APA style
%\bibliographystyle{model5-names}\biboptions{authoryear}

%% AMA style
%\usepackage{numcompress}\bibliographystyle{model6-num-names}

%% `Elsevier LaTeX' style
%\bibliographystyle{elsarticle-num}
%%%%%%%%%%%%%%%%%%%%%%%
\usepackage{xeCJK}
\usepackage{bm}
\usepackage{amsmath}
\usepackage{amssymb}
\usepackage{amsthm}
\usepackage{graphicx}
\usepackage{color}
\usepackage{booktabs}

\theoremstyle{plain}
\newtheorem{theorem}{\quad\quad Theorem}
\newtheorem{proposition}{\quad\quad Proposition}
\newtheorem{corollary}[theorem]{Corollary}
\newtheorem{lemma}{Lemma}
\newtheorem{example}{Example}
\newtheorem{assumption}{\quad\quad Assumption}
\newtheorem{condition}{Condition}

\theoremstyle{definition}
\newtheorem{remark}{\quad\quad Remark}
\theoremstyle{remark}
\begin{document}

\begin{frontmatter}

\title{High-dimensional two-sample test under spiked covariance}

%% Group authors per affiliation:
    \author[mymainaddress]{Rui Wang}
    \author[mymainaddress,mysecondaryaddress]{Xingzhong Xu\corref{mycorrespondingauthor}}
\cortext[mycorrespondingauthor]{Corresponding author}
\ead{xuxz@bit.edu.cn}
    \address[mymainaddress]{School of Mathematics and Statistics, Beijing Institute of Technology, Beijing 
    100081,China}
    \address[mysecondaryaddress]{Beijing Key Laboratory on MCAACI, Beijing Institute of Technology, Beijing 100081,China}
%\fntext[myfootnote]{Since 1880.}

%% or include affiliations in footnotes:
%\author[mymainaddress,mysecondaryaddress]{Elsevier Inc}
%\ead[url]{www.elsevier.com}



\begin{abstract}
    This paper considers testing the means of two $p$-variate normal samples in high dimensional setting.  The covariance matrices are assumed to be spiked, which often arises in practice. 
    We propose a new test procedure through projection on the orthogonal complement of principal space.
    The asymptotic normality of the new test statistic is proved and the power function of the test is given.
    Theoretical and simulation results show that the new test outperforms existing methods substantially when the covariance matrices are spiked. Even when the covariance matrices are not spiked, the new test is acceptable.
\end{abstract}

\begin{keyword}
    high dimension, mean test, orthogonal complement of principal space, spiked covariance
\end{keyword}

\end{frontmatter}

%\linenumbers



\section{Introduction}

Suppose that $X_{k1},\ldots,X_{kn_k}$  are independent identically distributed (i.i.d.) as $N_p(\mu_k,\Sigma_k)$, where $\mu_k$ and $\Sigma_k$ are unknown, $k=1,2$. We consider the hypothesis testing problem:

\begin{equation}\label{problem}
    H_0:\mu_1=\mu_2\quad \textrm{vs.}\quad H_1:\mu_1\neq \mu_2.
\end{equation}
In this paper, the dimension $p$ varies as $n_1$ and $n_2$ increase, i.e., high dimensional setting is adopted.
Testing hypotheses~\eqref{problem} is important in many applications, including biology, finance and economics. Quite often,  these data have strong correlations between variables. When strong correlations exist, covariance matrices are often spiked in the sense that a few eigenvalues are distinctively larger than the others. The paper is devoted to
testing hypotheses~\eqref{problem} in high dimensional setting with spiked covariance.


If $\Sigma_1=\Sigma_2=\Sigma$ is unknown, a classical test for hypotheses~\eqref{problem} is Hotelling's $T^2$ test.  Hotelling's test statistic is ${(\bar{X}_1-\bar{X}_2)}^T S^{-1}(\bar{X}_1-\bar{X}_2)$, where $S$ is the pooled sample covariance matrix. However, Hotelling's test is not defined when $p\geq n_1+n_2-1$.
Moreover,~\cite{Bai1996Efiect} showed that even if $p<n_1+n_2-1$, Hotelling's test suffers from low power when $p$ is comparable to $n$.
Perhaps, the main reason for low power of Hotelling's test is due to that $S$ is a poor estimator of $\Sigma$ when $p$ is large compared with $n_1+n_2$. See~\cite{Chen2010A} and the references therein.
In high dimensional setting,  
many test statistics in the literatures are based on an estimator of ${(\mu_1-\mu_2)}^T A(\mu_1-\mu_2)$ for a given positive definite matrix $A$. 
For example,~\cite{Bai1996Efiect} proposed a test based on
\begin{equation*}
    T_{BS}=\|\bar{X}_1-\bar{X}_2\|^2-(\frac{1}{n_1}+\frac{1}{n_2})\mathrm{tr}S,
\end{equation*}
which is an unbiased estimator of $\|\mu_1-\mu_2\|^2$.~\cite{Chen2010A} modified $T_{BS}$ by removing terms $\sum_{i=1}^{n_k}X_{ki}^T X_{ki}$, $k=1,2$ and proposed a test based on
\begin{equation*}
    \begin{aligned}
        T_{CQ}&=\frac{\sum_{i\neq j}^{n_1}X_{1i}^T X_{1j}}{n_1(n_1-1)}+\frac{\sum_{i\neq j}^{n_2}X_{2i}^T X_{2j}}{n_2(n_2-1)}-2\frac{\sum_{i=1}^{n_1}\sum_{j=1}^{n_2}X_{1i}^T X_{2j}}{n_1n_2}
        \\
            &=\|\bar{X}_1-\bar{X}_2\|^2-\frac{1}{n_1}\mathrm{tr}S_1-\frac{1}{n_2}\mathrm{tr}S_2,
    \end{aligned}
\end{equation*}
where $S_1$ and $S_2$ are sample covariance matrices. Statistic $T_{CQ}$ 
is also an unbiased estimator of $\|\mu_1-\mu_2\|^2$. Choosing $A={[\mathrm{diag}(\Sigma)]}^{-1}$,~\cite{Srivastava2008A} proposed a test based on
\begin{equation*}
    T_{S}={(\bar{X}_1-\bar{X}_2)}^T {[\mathrm{diag}(S)]}^{-1}(\bar{X}_1-\bar{X}_2),
\end{equation*}
where $\textrm{diag} (A)$ is a diagonal matrix with the same diagonal elements as $A$'s.
%To characterize strong correlation between variables,~\cite{Ma2015A} adopted a factor model proposed a test based on
%\begin{equation}\label{compete2}
 %    T_{FAST}=\frac{n_1 n_2}{n_1+n_2}\|\bar{X}_1-\bar{X}_2\|^2-(\mathrm{tr} S- \sum_{i=1}^{\hat{r}} \lambda_l(S))
%\end{equation}

As~\cite{Ma2015A} pointed out, however, these test procedures may not be valid if strong correlations exist, i.e., $\Sigma$ is far away from diagonal matrix. For example, the assumption 
%$$
%\mathrm{tr}(\Sigma_i \Sigma_j \Sigma_l \Sigma_h)=o[\mathrm{tr}^2\{{(\Sigma_1+\Sigma_2)}^2\}]\quad\quad  \textrm{for}\, i,j,l,h=1\,\textrm{or}\,2
%$$ 
\begin{equation}\label{chenscondition}
\mathrm{tr}(\Sigma^4)=o[\mathrm{tr}^2\{{(\Sigma)}^2\}]
\end{equation}
adopted by~\cite{Chen2010A} can be violated when $\Sigma=(1-c)I_p+c\bm{1}_p \bm{1}_p^T$ where $-{1}/{(p-1)}<c<1$, $I_p$ is the $p$ dimensional identity matrix and $\bm{1}_p$ is the $p$ dimensional vector  with
elements $1$. 

Strong correlations between variables do exist in practice. In gene expression analysis, genes are correlated due to genetic regulatory networks (see~\cite{Thulin2014A}).~\cite{Chen2011A} pointed out that in terms of pathway analysis in proteomic studies,  test level can not be guaranteed if correlations are incorrectly assumed to be absent.
 As~\cite{Ma2015A} argued, there're strong correlations between different stock returns since they are all affected by the market index.

Incorrectly assuming the absence of correlation between variables will result in level inflation and low power for a test procedure. A class of test procedures is proposed through random projection (see~\cite{Lopes2015A},~\cite{Thulin2014A} and~\cite{Srivastava2014RAPTT}). The idea is to project data on some random lower-dimensional subspaces. It has been shown that these
procedures perform well under strong correlations. 

In many situations, the correlations are determined by a small number of factors.
Then $\Sigma$ is spiked (see~\cite{Cai2012Sparse}).
The random projection methods imply that test procedures are improved when data are projected on certain subspaces.
We will see that the ideal subspace is the orthogonal complement of the principal space.
Fortunately, the principal space can be estimated consistently even in high dimensional setting by the theory of principal component analysis (PCA).
%We find the ideal subspace is the orthogonal complement of the principal space.
%In this case, we know from the theory of principal component analysis (PCA) that the principal space can be estimated consistently even in high dimensional setting.
With the assumption of spiked covariance model, we propose a new test procedure through projection on the (estimated) ideal subspace.  
The asymptotic distribution of the test statistic is derived and hence asymptotic power is given.
%We will see that the asymptotic power function increases fast. In fact, the increasing rate is of a higher order than that of $T_{CQ}$.
We will see that the test is more powerful than $T_{CQ}$.
%Simulation study justifies the well-performance of the new test. Our theoretical results need the assumption $\sqrt{p}/(n_1+n_2)\to 0$. Simulation study shows that if it doesn't converge to $0$, the theorem may not be valid.
Moreover, even there's no strong correlation showing up, we prove that the new test performs equally well as $T_{CQ}$ does. The idea is also generalized to the unequal variance setting and similar results still hold.

%{\color{red}{To the best of our knowledge,~\cite{Ma2015A} and~\cite{2016arXiv160202491A} are the only work concerned on problem (~\eqref{problem}) when strong correlation exists.
%\cite{Ma2015A} adopted a factor model and modified the test statistic of~\cite{Chen2010A} to guarantee the test level. But we will see that the test still suffers from low power. In an independent working paper,~\cite{2016arXiv160202491A} adopted a spiked covariance structure, and their statistic is similar to ours. The main advantage of our work is that our theorems don't need strict relationship between $p$ and $n$. And our statistic is invariant under shift.
%}}


%{\color{red}{A fairly recent work~\cite{2016arXiv160202491A} proposed a new test for strongly spiked eigenvalue model. The proposed a test based on an estimation of
%\begin{equation}
%    \begin{aligned}
%        T_{AY}=&\frac{\sum_{i\neq j}^{n_1}X_{1i}^T\tilde{V}_1\tilde{V}_1^T X_{1j}}{n_1(n_1-1)}+\frac{\sum_{i\neq j}^{n_2}X_{2i}^T\tilde{V}_1\tilde{V}_1^T X_{2j}}{n_2(n_2-1)}
%        \\&-2\frac{\sum_{i=1}^{n_1}\sum_{j=1}^{n_2}X_{1i}^T\tilde{V}_1\tilde{V}_1^T\tilde{V}_2\tilde{V}_2^T X_{2j}}{n_1n_2}
%    \end{aligned}
%\end{equation}
%which is similar to our statistic in form. However, the theory framework is different. And we will see our statistic is different from theirs in some key properties.
%}}


The rest of the paper is organized as follows. In Section 2,  the model and some assumptions are given.  In Section 3, we propose a test procedure under $\Sigma_1=\Sigma_2$. Section 4 exploits properties of the test. In Section 5, we generalize our test procedure to the situation of $\Sigma_1\neq \Sigma_2$. In Section 6, simulations are carried out and  a real data example is given. Section 7 contains some discussion. All the technical details are in appendix.

\section{Model and assumptions}


Let $\{X_{k1},\ldots,X_{kn_k}\}$, $k=1, 2$ be two independent  random samples from $p$ dimensional normal distribution with means $\mu_1$ and $\mu_2$ respectively.

\begin{assumption}\label{balance}
Assume $p\to \infty$ as $n_1, n_2\to \infty$. Furthermore, assume two samples are balanced, that is,
\begin{equation*}
    \frac{n_1}{n_2}\to \xi \in (0,+\infty).
\end{equation*}
\end{assumption}

To characterize correlations between $p$ variables, we consider spiked covariance structure which is adopted by PCA study. See~\cite{Cai2012Sparse} and the references given there.
\begin{assumption}\label{theModel}
Suppose $X_{ki}$, $i=1,2,\ldots,n_k$ and $k=1,2$ are generated by  following model
\begin{equation*}
X_{ki}=\mu_k+V_k D_k U_{ki}+Z_{ki},
\end{equation*}
where
$U_{ki}$'s are i.i.d.\  random vectors distributed as $r_k$ dimensional standard normal distribution with $r_k$ fixed, 
$D_k=diag(\lambda_{k1}^{\frac{1}{2}},\ldots,\lambda_{k{r_k}}^{\frac{1}{2}})$ with $\lambda_{k1}\geq \cdots \geq \lambda_{k{r_k}}>0$,
$V_k$ is  a $p\times r_k$ orthonormal matrix, $Z_{ki}$'s are i.i.d.\ random vectors distributed as  $N_p(0,\sigma^2_k I_p)$ independent of $U_{ki}$'s and $\sigma^2_k>0$, $k=1,2$.
\end{assumption}
Then $X_{ki}\sim N(\mu_k,\Sigma_k)$, where $ 
\Sigma_k=\textrm{Var}(X_{ki})=V_k\Lambda_k V_k^T+\sigma^2_k I_p
$
, $\Lambda_k=D_k^2=diag(\lambda_{k1},\ldots,\lambda_{k{r_k}})$.
From Assumption~\ref{theModel}, $V_k V_k^T$ is the orthogonal projection matrix on the column space of $V_k$. Let $\tilde{V}_k$ be a $p\times (p-r_k)$ full column rank orthonormal matrix orthogonal to columns of  $V_k$.
%, that is $\tilde{V}_k^T V_k=O_{r_k\times(p-r_k )}$
 Note that $\tilde{V}_k$ may not be unique. But the projection matrix $\tilde{V}_k\tilde{V}_k^T$ is unique because $\tilde{V}_k\tilde{V}_k^T=I-V_k V_k^T$.


\begin{assumption}\label{orderOfBeta}
    Assume that there is some constant $\kappa$ such that
    \begin{equation*}
    \kappa \lambda\geq \lambda_{k1}\geq \cdots \geq\lambda_{kr_k}\geq \lambda>0,
\end{equation*}
    where $\lambda=cp^\beta$ for some $c>0$ and $\beta\geq \frac{1}{2}$.
\end{assumption}


The restriction $\beta\geq 1/2$ is assumed in Assumption~\ref{orderOfBeta}. If $\beta< 1/2$, condition~\eqref{chenscondition} is meet and~\cite{Chen2010A}'s  method is valid. 
 %By Theorem~\ref{testPowerh}, we will see that if $\beta< 1/2$, the power of our test and $T_{CQ}$'s test power are asymptotic equivalent.
 Hence $\beta=1/2$ is the boundary of the scope between $T_{CQ}$ and our new test.


Finally, let $\tau={(n_1+n_2)}/{(n_1n_2)}$, $S$ be the pooled sample covariance.
\begin{equation}
S=\frac{1}{n_1+n_2-2}\sum_{k=1}^2\sum_{i=1}^{n_k} (X_{ki}-\bar{X}_k) {(X_{ki}-\bar{X}_k)}^T
    =\frac{(n_1-1)S_1+(n_2-1)S_2}{n_1+n_2-2},
\end{equation}
where
\begin{equation}
S_k=\frac{1}{n_k -1}\sum_{i=1}^{n_k} (X_{ki}-\bar{X}_k) {(X_{ki}-\bar{X}_k)}^T,
\end{equation}
is the sample covariance  of the sample $k$, $k=1,2$.


\section{Methodology}

In this section, we consider testing the hypotheses~\eqref{problem} with equal covariance matrices.
\begin{assumption}\label{theModel2}
Assume $V_1=V_2$, $D_1=D_2$, $\Lambda_1=\Lambda_2$, $\sigma_1=\sigma_2$ and $r_1=r_2$.
\end{assumption}

To simplify notations, the subscript $k$ of $\Sigma_k$, $V_k$, $D_k$, $\Lambda_k$, $\sigma_k$ and $r_k$ are dropped.
%\begin{equation}
%X_{ki}=\mu_k+V D U_{ki}+Z_{ki}.
%\end{equation}

\subsection{Motivation}
 %Facing a testing problem, a general pattern to derive a new test can be summarized as $3$ steps.
 %The first step is to propose a new statistic $T(X)$. 
 %Usually,  $T(X)$ is chosen to be an estimator of certain `distance' between null and alternative. $\mathrm{E}T=0$ under null and $\mathrm{E}T> 0$  under alternative.
 In high dimensional setting, many test procedures for hypotheses~\eqref{problem} is based on a statistic $T(X)$ which estimates ${(\mu_1-\mu_2)}^T A(\mu_1-\mu_2)$.
 Usually, $T(X)$ satisfies $\mathrm{E}T=0$ under null hypothesis and $\mathrm{E}T> 0$  under alternative.
 To determine the critical value, the asymptotic distribution of $T$ need to be derived, say 
 $$\frac{T-\textrm{E}T}{\sqrt{\textrm{Var}(T)}}\xrightarrow{\mathcal{L}} N(0,1).$$
 Since $\textrm{Var}(T)$ may depend on parameters, a ratio consistent estimator $\widehat{\textrm{Var}(T)}$ of $\textrm{Var}(T)$ is necessary. Then
 the rejection region of a level $\alpha$ test can be defined as $T(X)\geq \widehat{\textrm{Var}(T)}^{\frac{1}{2}}z_{1-\alpha}$ where $z_{1-\alpha}$ is the $1-\alpha$ quantile of $N(0,1)$. 
% Tests derived by the above pattern have an advantage in that it's clear what kind of alternatives the test favors. Many test procedures have been proposed for different $A$. In essence, test procedures for different $A$ are incomparable since they test different alternatives. For example, $T_{CQ}$ outperforms $T_S$ when $\Sigma$ is nearly an identity matrix. However, $T_S$ performs better when different variables are in different scales. 
%In general, it remains an important question that how to boost test power for a given `distance'.
The asymptotic power of the test is 
$$\Phi(\frac{\mathrm{E}T}{\sqrt{\mathrm{Var}(T)}}-z_{1-\alpha}).$$
Thus, a general idea to boost the power of test is to reduce the variance $\mathrm{Var}(T)$ while the mean $\mathrm{E}(T)$ varies relatively little.

Now we revisit $T_{BS}$ and $T_{CQ}$ which are both based on the estimation of $\|\mu_1-\mu_2\|^2$. Denote the spectral decomposition of $\Sigma$ by $\Sigma =\sum_{i=1}^p \lambda_i p_i p_i^T$  with $\lambda_1\geq \cdots \geq \lambda_p$, where $p_i$, $i=1,\ldots,p$, are orthonormal $p$ dimensional vectors. The main body of both $T_{BS}$ and $T_{CQ}$ is 
\begin{equation}\label{qifa}
    \frac{n_1 n_2}{n_1+n_2}\sum_{i=1}^p {(\bar{X}_1-\bar{X}_2)}^T  p_i p_i^T (\bar{X}_1-\bar{X}_2),
\end{equation}
which is a sum of $p$ independent terms. Since $\sqrt{{n_1 n_2}/{(n_1+n_2)}}(\bar{X}_1-\bar{X}_2)$ is distributed as $N(0,\Sigma)$, the variance of ${n_1 n_2}/{(n_1+n_2)} {(\bar{X}_1-\bar{X}_2)}^T  p_i p_i^T (\bar{X}_1-\bar{X}_2)$ is $2\lambda_i^2$ which decreases in $i$. By our previous argument, if a few leading terms with significantly large variance are removed, the modified test will be more powerful.


The argument is also supported by the likelihood ratio test. If $\Sigma$ is known, the LRT is based on 
\begin{equation}\label{qifafa}
    {(\bar{X}_1-\bar{X}_2)}^T\Sigma^{-1}(\bar{X}_1-\bar{X}_2)=\frac{n_1 n_2}{n_1+n_2}\sum_{i=1}^p \lambda_i^{-1}{(\bar{X}_1-\bar{X}_2)}^T  p_i p_i^T (\bar{X}_1-\bar{X}_2).
\end{equation}
The difference between~\eqref{qifa} and~\eqref{qifafa} is the weights $\lambda_i^{-1}$.
% For LRT, large $\lambda_i$'s corresponds to small weights in the sum.
%If $\lambda_i$ is large, then the corresponding term has a small weight $\lambda_i^{-1}$ in the sum. 
Unfortunately, $\lambda_i$'s are hard to precisely estimate in high dimensional setting. See~\cite{bai2010spectral} for detail. Nevertheless, it's possible to identify which $\lambda_i$'s are large. LRT implies the corresponding terms should have small weights, which coincides with our previous idea.
%If we assume there are correlations between $p$ variables, e.g. $\Sigma=(1-c)I+c\bm{1}_p \bm{1}_p^T$ where c is a constant fulfill $-\frac{1}{p-1}<c<1$, then $\frac{n_1 n_2}{n_1+n_2} {(\bar{X}_1-\bar{X}_2)}^T  p_1 p_1^T (\bar{X}_1-\bar{X}_2)$ distributed as $(cp+1-c)\chi^2_1$ whose variance is of order $p^2$ while $\frac{n_1 n_2}{n_1+n_2}\sum_{i=2}^p {(\bar{X}_1-\bar{X}_2)}^T  p_i p_i^T (\bar{X}_1-\bar{X}_2)$ is distributed as $(1-c)\chi^2_{p-1}$ whose variance is of order $p$. 
%The large variance is totally caused by term $p\chi^2_1$. 
%If we remove $\frac{n_1 n_2}{n_1+n_2} {(\bar{X}_1-\bar{X}_2)}^T  p_1 p_1^T (\bar{X}_1-\bar{X}_2)$ from $T_{BS}$, the variance of $T_{BS}$ can be significantly reduced to order $p$ from order $p^2$.

%Note that $\Sigma=(1-c)I+c\bm{1}_p+\bm{1}_p^T$ is just a special case of spiked covariance.

Under Assumption~\ref{theModel}, $\Sigma=V\Lambda V^T +\sigma^2 I_p$. The eigenvalues of $\Sigma$ are $\lambda_1+\sigma^2,\ldots,\lambda_r+\sigma^2,\sigma^2,\ldots,\sigma^2$. The eigenvectors corresponding to the first $r$ eigenvalues are columns of $V$. Follow our previous argument, if the principal subspace $VV^T$ is known, we project $X_{ki}$ on the orthogonal complement space $\tilde{V}\tilde{V}^T$ and invoke the statistic of~\cite{Chen2010A}. We define the following statistic

\begin{equation*}
\begin{aligned}
    T_{1}&=\|\tilde{V}^T(\bar{X}_1-\bar{X}_2)\|^2-\frac{1}{n_1}\mathrm{tr}(\tilde{V}^T S_1\tilde{V})-\frac{1}{n_2}\mathrm{tr}(\tilde{V}^T S_2\tilde{V}).
    \\
    %&=\frac{\sum_{i\neq j}^{n_1}X_{1i}^T\tilde{V}\tilde{V}^T X_{1j}}{n_1(n_1-1)}+\frac{\sum_{i\neq j}^{n_2}X_{2i}^T\tilde{V}\tilde{V}^T X_{2j}}{n_2(n_2-1)}-2\frac{\sum_{i=1}^{n_1}\sum_{j=1}^{n_2}X_{1i}^T\tilde{V}\tilde{V}^T X_{2j}}{n_1n_2}
\end{aligned}
\end{equation*}
the asymptotic normality of $T_1$ can be obtained by~\cite{Chen2010A}'s Theorem 1.

\begin{proposition}\label{oracleTheorem}
    Under Assumptions~\ref{balance}-\ref{theModel2}, if local alternative holds, that is, $\frac{n_1+n_2}{p}\|\mu_1-\mu_2\|^2\to 0$, then we have 
    \begin{equation*}
        \frac{T_1-\|\tilde{V}(\mu_1-\mu_2)\|^2}
        {\sigma^2\sqrt{2\tau^2 p}}\xrightarrow{\mathcal{L}}N(0,1).
    \end{equation*}
\end{proposition}

\begin{remark}
    The asymptotic variance of $T_1$ is of order $\tau^2 p$ while the asymptotic variance of $T_{CQ}$ is of order $\tau^2 p^{2\beta}$ by~\cite{Chen2010A}'s Theorem 1. The asymptotic variance is reduced significantly if $\beta>1/2$ and $p$ is sufficiently large.
\end{remark}

%\begin{remark}If $V$ is known, the model in Assumption~\ref{theModel} is very similar to random effects model. And our idea is just like REstricted Maximum Likelihood (REML).
%\end{remark}

%\begin{remark}
%    Suppose $\mu_1=\mu_2$. When $\beta>1/2$, the order of $T_1$'s variance is smaller than the order of $T_{CQ}$'s variance, which implies $T_1/(\sigma^2\sqrt{2\tau^2 p})$ is asymptotically independent of $T_{CQ}/\sqrt{2\tau^2 \mathrm{tr}\Sigma^2}$. Hence $T_1$ does provide additional information, although $T_1$ is inherited from $T_{CQ}$.  
%\end{remark} 


\subsection{New Test}

Actually, $T_1$ is not a statistic since $VV^T$ and $\tilde{V}\tilde{V}^T$ are unknown. Fortunately, $V$ and $\tilde{V}$ can be estimated by $\hat{V}$ and $\hat{\tilde{V}}$, where $\hat{V}$ and $\hat{\tilde{V}}$ are the first $r$ and last $p-r$ eigenvectors of $S$, respectively. These simple estimators actually reach the optimal convergence rate. See~\cite{Cai2012Sparse}.

%We are now in a position to propose a new statistic for testing the hypotheses~\eqref{problem}.
Define

\begin{equation*}
\begin{aligned}
    T_2&=\|\hat{\tilde{V}}^T(\bar{X}_1-\bar{X}_2)\|^2-\frac{1}{n_1}tr(\hat{\tilde{V}}^T S_1\hat{\tilde{V}})-\frac{1}{n_2}tr(\hat{\tilde{V}}^T S_2\hat{\tilde{V}}).
    %T_2=\frac{\sum_{i\neq j}^{n_1}X_{1i}^T\hat{\tilde{V}}\hat{\tilde{V}}^T X_{1j}}{n_1(n_1-1)}+\frac{\sum_{i\neq j}^{n_2}X_{2i}^T\hat{\tilde{V}}\hat{\tilde{V}}^T X_{2j}}{n_2(n_2-1)}
%-2\frac{\sum_{i=1}^{n_1}\sum_{j=1}^{n_2}X_{1i}^T\hat{\tilde{V}}\hat{\tilde{V}}^T X_{2j}}{n_1n_2}
\end{aligned}
\end{equation*}
We propose our new test statistic as
\begin{equation}\label{myTest}
    Q=\frac{T_2}{\hat{\sigma}^2\sqrt{2p\tau^2}},
\end{equation}
where $\hat{\sigma}^2$ is a ratio consistent estimator of $\sigma^2$. In next section, it will be proved that the asymptotic distribution of $Q$ is $N(0,1)$. We reject the null hypothesis if $Q$ is larger than the upper $\alpha$ quantile of $N(0,1)$.

\begin{remark}
    Compared with random projection method, our projection is determined by the structure of $S$. Hence we don't  project multiple times as random projection method did, which leads to reproducibility.
\end{remark}


\begin{remark} The statistic $T_2$ is invariant under shift transformation, that is, $T_2$ is invariant when adding a vector to $X_{1i}$ and $X_{2j}$ simultaneously: $X_{1i}\mapsto X_{1i}+\mu$ and $X_{2j}\mapsto X_{2j}+\mu$, $i=1,\ldots,n_1$, $j=1,\ldots,n_2$.
\end{remark}


\begin{remark}
If $r$ is an unknown positive number, a consistent estimator of $r$ is
\begin{equation}\label{estimateR}
    \hat{r}=\textrm{argmax}_{l\leq R}\frac{\lambda_l(S)}{\lambda_{l+1}(S)},
\end{equation}
where $R$ is the maximum value of $r$ to be tested. See~\cite{Ahn2009Eigenvalue} for detail. Therefore, without loss of generality, we will assume that $r$ is known.
%and deal with it saperately if $r=0$.
\end{remark}

    Theoretical results will show that the asymptotic variance of $T_2$ is significantly smaller than $T_{CQ}$. 
    On the other hand, the new test statistic estimates $\|\tilde{V}(\mu_1-\mu_2)\|^2$. Hence the superiority of the new test will be established if 
    
\begin{equation}\label{yuedengyu}
    \frac{\|\tilde{V}(\mu_1-\mu_2)\|}{\|\mu_1-\mu_2\|}\approx 1.
\end{equation}
Unfortunately,~\eqref{yuedengyu}
is not always the case since there always exists some
$\tilde{V}$ and $\mu_1-\mu_2$ such that $\|\tilde{V}(\mu_1-\mu_2)\|=0$.
However,~\eqref{yuedengyu} is reasonable since $\tilde{V}\tilde{V}^T$ is nearly an identity matrix in the sense that
    ${\|I_p-\tilde{V}\tilde{V}^T\|_F^2}/{\|I_p\|_F^2}=r/p\to 0$. 
In bayes framework, if we assume that the elements of $\mu_k$ are independently generated from certain probability distribution, it can be established that 
\begin{equation*}
    \frac{\|\tilde{V}(\mu_1-\mu_2)\|}{\|\mu_1-\mu_2\|}\xrightarrow{P}1.
\end{equation*}
Such assumption of $\mu_k$ will be used in Theorem~\ref{sameTheorem}.


Next we concern the problem of estimating $\sigma^2$. We give two different estimators and later we will prove their consistency using different techniques. Note that $\sigma^2$ can be written as
\begin{equation}\label{jjjVariance}
    \sigma^2=\sum_{i=r+1}^{p}\lambda_i(\Sigma)=\frac{1}{p-r}\mathrm{tr}\tilde{V}^T\Sigma\tilde{V}.
\end{equation}
The first estimator we propose is to directly estimate~\eqref{jjjVariance} by
\begin{equation*}
    \hat{\sigma}^2_{(1)}=\frac{1}{p-r}\mathrm{tr}\hat{\tilde{V}}^T S\hat{\tilde{V}}.
\end{equation*}
The estimator $\hat{\sigma}^2_{(1)}$ uses the smallest $p-r$ eigenvalues of $S$. However some of the smallest $p-r$ eigenvalues may be also contaminated by $\Lambda$. Hence besides $\hat{\sigma}^2_{(1)}$, we give another estimator of $\sigma^2$:
\begin{equation*}
    \hat{\sigma}^2_{(2)}=\frac{1}{p-4r}\sum_{i=2r+1}^{p-2r} \lambda_i(S).
\end{equation*}
If estimator $\hat{\sigma}^2$ of $\sigma^2$ is ratio consistent, the asymptotic distribution of~\eqref{myTest} will not change if we replace $\sigma^2$ by $\hat{\sigma}^2$  due to Slutsky's theorem.

%When $\frac{\sqrt{p}}{n_1+n_2}\to 0$, the critical value of our test can be approximated by it's asymptotic distribution which we will encounter later.
%However, it is a more practical issue to deal with the case when $n$ is small or the case when $p$ is much larger than $n$. In these cases, the null distribution is complicated and asymptotic distribution is a poor approximation of true distribution. Fortunately, permutation method can be used with the price of heavier computational burden. See~\cite{Lehmann}.



\section{Theoretical results}

In this section, we derive some properties of the new test statistic. Our main results require the following relationship of $n_1,n_2$ and $p$.
\begin{assumption}\label{pAndN}
    We assume
    $$
    \frac{\sqrt{p}}{n_1+n_2}\to 0.
    $$
\end{assumption}


First we establish $\hat{\sigma}^2_{(1)}$ and $\hat{\sigma}^2_{(2)}$ are consistent estimators of $\sigma^2$. 
\begin{proposition}\label{varianceEstimation}
    Under Assumptions~\ref{balance}-\ref{theModel2}, %,~\ref{theModel},~\ref{orderOfBeta}
     $\hat{\sigma}_{(2)}^2$ is consistent.
%\begin{equation}
    %\hat{\sigma}_{(2)}^2\xrightarrow{P}\sigma^2
%\end{equation}
    If we further assume~\ref{pAndN}, then $\hat{\sigma}_{(1)}^2$ is consistent.
    %\begin{equation}
        %\hat{\sigma}_{(1)}^2\xrightarrow{P}\sigma^2
    %\end{equation}
\end{proposition}

\begin{remark}
    Compared with $\hat{\sigma}^2_{(1)}$, $\hat{\sigma}^2_{(2)}$ uses less information. Proposition~\ref{varianceEstimation}, however, shows that $\hat{\sigma}^2_{(2)}$ is consistent without requirement of the relationship between $n$ and $p$, while $\hat{\sigma}^2_{(1)}$ requires Assumption~\ref{pAndN}.
\end{remark}

Next we derive the asymptotic normality of the new test statistic. Consider the case  when the eigenvalues of $\Sigma$ is bounded, i.e., there is no clear correlation between variables.
In many practical problems, the alternative is `dense', i.e., under $H_1$ the signals in $\mu_1-\mu_2$ spread out over a large number of co-ordinates. See~\cite{Tony2013}.
We characterize `dense' alternative from bayes framework and assume that elements of $\mu_k$  are independently generated from normal distribution.
Next theorem shows that  the power of our new test is asymptotically the same as~\cite{Chen2010A}'s test in this case.


\begin{theorem}\label{sameTheorem}
   Assume $X_{ki}\sim N(\mu_k,\Sigma)$,  $i=1,\ldots,n_k$, $k=1,2$. Suppose that assumption~\ref{balance} holds, $0<c\leq\lambda_p(\Sigma)\leq\lambda_1(\Sigma)\leq C<\infty$ where $c$ and $C$ are constant, each element of $\mu_k$ is independently generated by $N(0,{(n_k\sqrt{p})}^{-1}\psi)$ for $k=1,2$, $\psi$ is a constant and  $\hat{r}$ is estimated by~\eqref{estimateR}. If $p\to\infty$ as $n\to\infty$ and $p=o({(n_1+n_2)}^2)$, then we have
    
\begin{equation*}
    \frac{T_2-\|\mu_1-\mu_2\|^2}{\sqrt{2\tau^2 \mathrm{tr}\Sigma^2}} \xrightarrow{\mathcal{L}} N(0,1).
\end{equation*}
\end{theorem}


Next we establish the asymptotic normality of the new test statistic under spiked covariance model.
Our first step is to prove the asymptotic normality under null hypothesis.

\begin{theorem}\label{myPanpan}
    Under Assumptions~\ref{balance}-\ref{pAndN} and ${\sqrt{p}}/{(n_1+n_2)}\to 0$,
if the null hypothesis holds, then 
    \begin{equation*}
        \frac{T_2}{\sigma^2\sqrt{2p \tau^2}}\xrightarrow{\mathcal{L}}N(0,1).
    \end{equation*}
\end{theorem} 
\begin{remark}
    The critical value of the new test can be determined by Theorem~\ref{myPanpan}, that is, reject when $T_2/(\sigma^2\sqrt{2p\tau^2})>z_{1-\alpha}$.
\end{remark}
    Then we generalize the Theorem~\ref{myPanpan} to the case of local alternative.

\begin{theorem}\label{spaceEstimation}
    Under Assumptions~\ref{balance}-\ref{pAndN},
if the local alternative holds, that is,
    $$\frac{(n_1+n_2)}{\sqrt{p}}\|\mu_1-\mu_2\|^2=O(1),$$
then 
\begin{equation*}
        \frac{T_2-\|\tilde{V}^T(\mu_1-\mu_2)\|^2}{\sigma^2\sqrt{2p\tau^2}}\xrightarrow{\mathcal{L}}N(0,1).
\end{equation*}
\end{theorem} 

%\begin{remark}  Compared with~\cite{2016arXiv160202491A}'s assumption (ix) which is equivalent to assuming $\frac{p^{2\beta-1}}{n_1+n_2}\to 0$ in model~\eqref{theModel}, our assumption $\frac{\sqrt{p}}{n_1+n_2}\to 0$ doesn't involved $\beta$.
%And when $\beta\geq \frac{3}{4}$, our assumption is weaker than~\cite{2016arXiv160202491A}'s. Note that when $\beta=\frac{1}{2}$, $\frac{\sqrt{p}}{n_1+n_2}$ is a necessary condition to make $\hat{V}\hat{V}^T$ a consistent
%estimator of $VV^T$ (see lemma 2 in appendix). So condition $\frac{\sqrt{p}}{n_1+n_2}$ is roughly the best we can do if the relationship between $p$ and $n$ doesn't involve $\beta$.
%\end{remark}

By Proposition~\ref{varianceEstimation}  and Theorem~\ref{spaceEstimation}, the power of our new test can be obtained immediately.


\begin{theorem}\label{testPowerh}
    Under Assumptions~\ref{balance}-\ref{pAndN},
    if we reject the null hypothesis when $Q$ is larger than $1-\alpha$ quantile of $N(0,1)$, then the power of our test is
    \begin{equation*}
        \Phi\Big(-\Phi^{-1}(1-\alpha)+\frac{\|\tilde{V}(\mu_1-\mu_2)\|^2}{\sigma^2\sqrt{2\tau^2p}}\Big).
    \end{equation*}
\end{theorem}


\begin{remark} The power of $T_{CQ}$ is of the form
\begin{equation*}
    \Phi\Big(-\Phi^{-1}(1-\alpha)+\frac{\|\mu_1-\mu_2\|^2}{\sqrt{2\tau^2\mathrm{tr}\Sigma^2}}\Big).
\end{equation*}
If $\|\tilde{V}\mu\|/\|\mu\|\to 1$, the relative efficiency of our test with respect to Chen's test is
\begin{equation*}
    \sqrt{\frac{\mathrm{tr}\Sigma^2}{(p-r)\sigma^4}}\sim p^{\beta-1/2}.
\end{equation*}
\end{remark}

\section{Unequal Variance}

In this section, we concern the situation with unequal covariance matrices.
%With the theoretic work we have done, it's not hard to deal with general case, that is, $\Sigma_1$ and $\Sigma_2$ are both spiked but don't need to be equal.
Assume $\{X_{11},\ldots, X_{1n_1}\}$ and $\{X_{21},\ldots, X_{2n_2}\}$ are both generated from the model in Assumption~\ref{theModel}.
Denote by $\hat{V}_k$ the first $r_k$ eigenvectors of $S_k$ for $k=1,2$.
With a little abuse of notation, let $VV^T$ be the projection on the sum of column spaces of $V_1$ and $V_2$, that is,
\begin{equation*}
    VV^T =(V_1,V_2){\big({(V_1,V_2)}^T (V_1,V_2)\big)}^{+}{(V_1,V_2)}^T.
\end{equation*}
where $A^{+}$ is the Moore-Penrose inverse of a matrix A. Similarly, let $\hat{V}\hat{V}^T$ be the projection matrix on the sum of column spaces of $\hat{V}_1$ and $\hat{V}_2$.
 We define $\tilde{V}\tilde{V}^T=I_{p}-VV^T$ and $\hat{\tilde{V}}\hat{\tilde{V}}^T=I_{p}-\hat{V}\hat{V}^T$. 

The previous statistic can not be directly used
since the principal subspace is different for $X_{1i}$ and $X_{2j}$. The idea here is to remove all large variance terms from $T_{CQ}$ by projecting data on the space $\tilde{V}\tilde{V}^T$. Thus, we propose a new test statistic as
\begin{equation*}
\begin{aligned}
    T_3&=\|\hat{\tilde{V}}^T(\bar{X}_1-\bar{X}_2)\|^2-\frac{1}{n_1}\mathrm{tr}(\hat{\tilde{V}}^T S_1\hat{\tilde{V}})-\frac{1}{n_2}\mathrm{tr}(\hat{\tilde{V}}^T S_2\hat{\tilde{V}}).
%    T_3=\frac{\sum_{i\neq j}^{n_1}X_{1i}^T\hat{\tilde{V}}\hat{\tilde{V}}^T X_{1j}}{n_1(n_1-1)}+\frac{\sum_{i\neq j}^{n_2}X_{2i}^T\hat{\tilde{V}}\hat{\tilde{V}}^T X_{2j}}{n_2(n_2-1)}
%    -2\frac{\sum_{i=1}^{n_1}\sum_{j=1}^{n_2}X_{1i}^T\hat{\tilde{V}}\hat{\tilde{V}}^T X_{2j}}{n_1n_2}
\end{aligned}
\end{equation*}


The theoretical results are parallel to those in equal variance setting.

%Compared with~\cite{2016arXiv160202491A}, our statistic have several advantages.
%First, our new statistic is invariance under transformation $X_{1i}\mapsto X_{1i}+\mu$ and $X_{2j}\mapsto X_{2j}+\mu$. So the null distribution of our test doesn't effected by $\mu$ and the test level can be guarenteed. 
%Second, our statistic doesn't rely on any single eigenvector of $\hat{V}$ but on the whole principal space $\hat{V}\hat{V}^T$. As a result, our statistic is uniquely defined. 
%Third, our statistic enjoys higher computation efficiency than~\cite{2016arXiv160202491A}'s method.

\begin{theorem}\label{myXiaopanpan}
    Under Assumptions~\ref{balance}-\ref{orderOfBeta} and~\ref{pAndN},
     if the null hypothesis holds, then 
\begin{equation*}
    \frac{T_3}{\sqrt{\sigma_n^2}}\xrightarrow{\mathcal{L}}N(0,1),
\end{equation*}
where
$\sigma_n^2=\frac{2(p-r_1-r_2)}{n_1(n_1-1)}\sigma_1^4+\frac{2(p-r_1-r_2)}{n_2(n_2-1)}\sigma_2^4+\frac{4(p-r_1-r_2)}{n_1n_2}\sigma_1^2\sigma_2^2$.
\end{theorem}
\begin{remark}
    Even if $\hat{\tilde{V}}_k\hat{\tilde{V}}_k^T$ is an consistent estimator of $\tilde{V}_k\tilde{V}_k^T$ for $k=1,2$, $\hat{\tilde{V}}\hat{\tilde{V}}^T$ may not be an consistent estimator of $\tilde{V}\tilde{V}^T$.
    Nevertheless, the asymptotic normality still holds.
\end{remark}
\begin{theorem}\label{myXiaopanpan2}
    Under Assumptions~\ref{balance}-\ref{orderOfBeta} and~\ref{pAndN},
 if the local alternative holds, that is,
    $$\frac{n_1+n_2}{\sqrt{p}}\|\mu_1-\mu_2\|^2=O(1),$$
then 
\begin{equation*}
        \frac{T_3-\|\tilde{V}^T(\mu_1-\mu_2)\|^2}{\sqrt{\sigma_n^2}}\xrightarrow{\mathcal{L}} N(0,1).
\end{equation*}
\end{theorem} 

 $\sigma_n^2$ can be estimated by ratio consistent estimator of $\sigma^2_k$ for $k=1,2$. Thus, if $n_k$ and $p$ are large and ${\sqrt{p}}/{(n_1+n_2)}$ is small, we reject when $T_3/\sqrt{\hat{\sigma}_n^2}>z_{1-\alpha}$. 
 %If $n$ is small or $p$ is large compared with n, we use permutation method to determine critical value.




\section{Numerical studies}
\subsection{Simulation results}

Our simulation study focus on equal variance case. 
We generate $X_{ki}$ by the model in Assumption~\ref{theModel}, where each element of $U_{ki}$ and $Z_{ki}$ are generated from $N(0,1)$.
$V$ is a random orthonormal matrix. 
We generate $\lambda_i$ as $p^{\beta}$ plus a random error from $U(0,1)$.

%The key to the validation of Theorem~\ref{myPanpan} is  that $T_{\textrm{dif}}=\frac{n_1n_2|T_1-T_2|}{\sqrt{2p}(n_1+n_2)\sigma^2}$ converges to $0$.
%Here we verify it by simulation.
%We set $n_1=n_2=n$, $p=n^i$ for $i=1,2$ and plot $T_{\textrm{dif}}$ versus $p$.
%The results are illustrated in figure~\ref{fig:fig1}.
%From the results we can find that $T_{\textrm{dif}}$ clearly converges to $0$ when $p=n$.
%In the case of $p=n^2$ which is exactly beyond the assumption of Theorem~\ref{myPanpan},
%$T_{\textrm{dif}}$ is large and it's not clear whether $T_{\textrm{dif}}$  converges to $0$.
%\begin{figure}
%    \centering 
%    \includegraphics[height=6cm]{code/difference1.jpeg}
%    \includegraphics[height=6cm]{code/difference2.jpeg}\\
%    \includegraphics[height=6cm]{code/difference3.jpeg}
%    \includegraphics[height=6cm]{code/difference4.jpeg}\\
%    \includegraphics[height=6cm]{code/difference5.jpeg}
%    \includegraphics[height=6cm]{code/difference6.jpeg}\\
%    \caption{These are plots of $T_{\textrm{dif}}$ versus $p$. The first column and the second column are the case of $p=n$ and $p=n^2$, separately. The cases of $\beta=1,2,3$ are in the row $1,2,3$ separately. $r$ is set to be $3$ in all cases. }\label{fig:fig1}
%\end{figure}

First we simulate the level of the new test. The nominal level $\alpha=0.05$ and we set $r=2$. Samples are repeatedly generated $1000$ times to calculate empirical level.  For comparison, we also give corresponding `oracle' level which is calculated by `statistic' ${T_1}/(\sigma^2\sqrt{2p\tau^2})$ whose asymptotic normality can be guaranteed by Theorem 1 in~\cite{Chen2010A}. The results are listed in
Table~\ref{biaoge1}. From the results, we can find that for small $n$ and $p$, even oracle level is not satisfied. Level of the new test is  a little inflated compared with oracle level and it performs better when $n$ is larger.

\input{code/level.tex}



Then we simulate the empirical power of our test and~\cite{Chen2010A}'s test. The simulation results of~\cite{Ma2015A} have showed that the level of the~\cite{Chen2010A}'s test can't be guaranteed when covariance is spiked. To be fair, we use permutation method to compute critical value. The validity of permutation method can be found in~\cite{Lehmann}'s Example 15.2.2. We plot the empirical power versus $\|\mu_1-\mu_2\|$ when other parameters hold constant. The results are illustrated in figure~\ref{fig:fig2}.
From the results, we can find that when $\Sigma$ is spiked, the new test outperforms $T_{CQ}$ substantially; when $\Sigma$ is not spiked, the new test and $T_{CQ}$ are comparable.
\begin{figure}
    \centering 
    \includegraphics[height=6cm]{code/fig1.jpeg}
    \includegraphics[height=6cm]{code/fig2.jpeg}
    \\
    \includegraphics[height=6cm]{code/fig3.jpeg}
    \includegraphics[height=6cm]{code/fig4.jpeg}
    \\
    \includegraphics[height=6cm]{code/fig5.jpeg}
    \includegraphics[height=6cm]{code/fig6.jpeg}
    \caption{Empirical power simulation. $\alpha$ is set to be $0.05$. $d$ is proportional to $\|\mu_1-\mu_2\|^2$. For each simulation, we do 50 permutations to determine critical value. We generate $100$ independent samples to compute empirical power. }\label{fig:fig2}
\end{figure}

%Permutation method is computation expensive. So when $p$ and $n$ are large, we simulate empirical power by asymptotic distribution. The results are illustrated in figure~\eqref{fig:fig3}.

%\begin{figure}\label{fig:fig3}
    %\centering 
    %\includegraphics[height=6cm]{code/newfig1.jpeg}
    %\includegraphics[height=6cm]{code/newfig2.jpeg}
    %\\
    %\includegraphics[height=6cm]{code/newfig3.jpeg}
    %\includegraphics[height=6cm]{code/newfig4.jpeg}
    %\\
    %\includegraphics[height=6cm]{code/newfig5.jpeg}
    %\includegraphics[height=6cm]{code/newfig6.jpeg}
    %\caption{Empirical Power (critical values are computed by asymptotic distribution)}\label{fig:fig3}
%\end{figure}

\subsection{Real data analysis}
In this section, we study the same practical problem as~\cite{Ma2015A} did. That is testing whether Monday stock returns are equal to those of other trading days on average. Define an observation be the log return of stocks in a day. Hence $p$ is the total number of stocks. Let sample $1$ and sample $2$ be the observations on Monday and the other trading days, respectively.  Then we would like to test $H_0\, :\mu_1=\mu_2$ v.s. $H_1\,:\mu_1\neq \mu_2$. We collected the data of $p=710$
 stocks of China
from 01/04/2013 to 12/31/2014. There are total $n_1=95$ Monday and $n_2=388$ other trading days. 

We assume $\Sigma_1=\Sigma_2$. The first eigenvaule of $S$ is $0.14$, which is significantly larger than the others.
In fact, the second eigenvalue is $0.02$.
Hence there's clearly a spiked eigenvalue. We set $r=1$ and perform our new test. The $p$ value is $0.149$, which is obtained by $1000$ permutations. Hence, the null hypothesis can not be rejected for $\alpha=0.05$. We draw the same conclusion as~\cite{Ma2015A}.

\section{Conclusion remark}

This paper is concerned with the problem of testing the equality of means in the setting of high dimension and spiked covariance. We removes big variance terms from $T_{CQ}$ and it's power is boosted substantially. The asymptotic normality of the new statistic is proved and the asymptotic power is given. %The new test outperforms $T_{CQ}$ substantially if the variance is spiked.
%We also generalize the test to unequal variance case.

In another paper,~\cite{Zhao2016A} proved their test statistic can be written in the form of projection. Their simulation results showed that their test performs well under strong correlations.
Our work partially explains why their test performs well although the projections are slightly different. 

 Spiked covariance is an important correlation pattern and has been widely studied in terms of PCA\@. In PCA, authors focus on the principal subspace. However, our work shows that in some circumstance, the complement of principal subspace is more useful. 

%Although we proved our test is still satisfied when $r=0$, the estimator $\hat{r}$ is not consistent. Therefore, it remains a question whether there's a better estimator of $r$, which is beyond the scope of this paper.

Our theoretical results rely on the assumption $\sqrt{p}/(n_1+n_2)\to 0$. In the situation of small sample or very large $p$, the critical value of the new test can be determined by permutation method. Our simulation shows that the new test still performs well. It remains a theoretical interest to derive the power function in these situations.



\section*{Appendix}
We denote by $\|\cdot \|$ and $\|\cdot\|_F$ the operator and Frobenius  norm of matrix, separately.

%\begin{lemma}\label{lemma1}
%    let $X$ be a $p$-dimensional random vector with distribution $N(0,\Sigma)$. Denote the spectral decomposition of $\Sigma$ by $\Sigma =\sum_{i=1}^p \lambda_i p_i p_i^T$ with $\lambda_1\geq \cdots \geq \lambda_p$. Then $X^T p_i p_i^T X$ is stochastically larger than $X^T p_j p_{j}^T X$ for $i<j$.
%\end{lemma}
%\begin{proof}[\textbf{Proof}]
%    The lemma is established immediately once we note that $X^T p_i p_i^T X/\sqrt{\lambda_i}$ is distributed as $\chi^2$ distribution with freedom $1$.
%\end{proof}


\begin{lemma}[Weyl's inequality]
Let $H$ and $P$ be two symmetric matrices and $M=H+P$. If $j+k-n\geq i\geq r+s-1$, we have
\begin{equation*}
\lambda_j(H)+\lambda_k(P)\leq \lambda_i(M) \leq \lambda_r(H)+\lambda_s(P).
\end{equation*}
\end{lemma}
\begin{corollary}\label{WeylCor}
    Let $H$ and $P$ be two symmetric matrices and $M=H+P$. If $\mathrm{rank}(P)< k$, then
    \begin{equation*}
        \lambda_k(M)\leq \lambda_1(H).
    \end{equation*}
\end{corollary}

\begin{lemma}\label{lemmaRankLim}
    Let $H$ and $P$ be two $p\times p$ positive semi-definite matrices and $M=H+P$.
    Suppose $\mathrm{Rank}(H)=d_1$, $\mathrm{Rank}(P)=d_2$, where $d_1$ and $d_2$ may depend on $n$. Assume $\mathrm{Rank}(M)=d_1+d_2$, $\lambda_1(M)\to 1$ and $\lambda_{d_1+d_2}(M)\to 1$, then we have
    \begin{equation}\label{lemma2r1}
        \lambda_1 (H)\to 1, \quad \lambda_{d_1} (H)\to 1,
    \end{equation}
    and
    \begin{equation}\label{lemma2r2}
        \lambda_1 (P)\to 1, \quad \lambda_{d_2} (P)\to 1,
    \end{equation}
\end{lemma}
\begin{remark}
    This generalizes the Cochran's theorem (See Anderson).
\end{remark}
\begin{proof}
    By Weyl's inequality, $\lambda_1 (H)\leq \lambda_1 (M)\to 1$ and
    $\lambda_{d_1} (H)\geq \lambda_{d_1+d_2}(M)\to 1$. Hence~\eqref{lemma2r1} holds. And~\eqref{lemma2r2} follows in the same way.
\end{proof}

\begin{lemma}[Convergence rate of principal space estimation]
    Under the model in Assumption~\ref{theModel2} with condition~\eqref{balance}, assume $\Sigma_1=\Sigma_2$. Suppose $p\to \infty$ as $n\to \infty$ and $\frac{\sqrt{p}}{n_1+n_2}\to 0$. We have
\begin{equation*}
E\|\hat{V}\hat{V}^T-VV^T\|^2_F =O(\frac{p}{p^{\beta}(n_1+n_2)}).
\end{equation*}
\end{lemma}


\begin{proof}[\textbf{Proof}]
    By theorem 5 of~\cite{Cai2012Sparse}, sample principal subspace $\hat{V}\hat{V}^T$ is a minimax rate estimator of $VV^T$, namely, it reaches the minimax convergence rate
    \begin{equation}\label{xiaopianpian}
        \inf_{\hat{V}}\sup_{\Sigma} E\|\hat{V}\hat{V}^T-VV^T\|^2_F\asymp r\wedge (p-r)\wedge \frac{r(p-r)}{(n_1+n_2-2)h(\lambda)}
    \end{equation}
    as long as the right hand side tends to $0$. Here $h(\lambda)=\frac{\lambda^2}{\lambda+1}$, $a_n\asymp b_n$ represents $a_n\geq cb_n$ and $a_n\leq Cb_n$ for some positive $c,C$ for every $n$. In model of Assumption~\ref{theModel},  $r$ is fixed, $\lambda=cp^\beta$. It's obvious that the right hand side of~\eqref{xiaopianpian} is of order $\frac{p^{1-\beta}}{n_1+n_2}$. We note that it is assumed $\beta\geq \frac{1}{2}$ in model of
    Assumption~\ref{theModel}, together with $\frac{\sqrt{p}}{n_1+n_2}\to 0$ we have
    $\frac{p^{1-\beta}}{n_1+n_2}\to 0$. Hence
    $\hat{V}\hat{V}^T$ reaches the optimal rate.

\end{proof}

\begin{lemma}\label{maxEigen}
    Suppose that $W_n$ is a $p \times p$ matrix distributed as $\textrm{Wishart}_p(n,I_{p})$, $p/n\to c\in [0,+\infty]$. Then as $n\to \infty$,
    $$
        \lambda_1(W_n)=O_P(\max(n,p)).
    $$
\end{lemma}
\begin{proof}[\textbf{Proof}]
    If $p$ is bounded as $n\to \infty$, then $p$ is fixed when $n$ is sufficiently large and the lemma can be established by weak law of large numbers. Otherwise, if $p\to \infty$ as $n\to \infty$, the lemma is a direct corollary of Theorem 1.1 of~\cite{Johnstone2001On} and Theorem 2 of~\cite{Karoui2003On}.
\end{proof}



\begin{lemma}\label{quadraticFormCLT}
    Suppose $X_{n}$ is a $k_n$ dimensional standard normal random vector and $A_n$ is a $k_n\times k_n$ symmetric matrix. Then a necessary and sufficient condition for
    \begin{equation}\label{quadratic}
        \frac{X_n^T A_n X_n-\mathrm{E}X_n^T A_n X_n}{{[\mathrm{Var}(X_n^T A_n X_n)]}^{1/2}}\xrightarrow{\mathcal{L}}N(0,1)
    \end{equation}
    is that
    \begin{equation}\label{quadraticEigen}
        \frac{\lambda_{\max}(A_n^2)}{\mathrm{tr}(A_n^2)}\to 0.
    \end{equation}
\end{lemma}
\begin{remark}
This lemma is the Example 5.1 of Jiming Jiang (1996). Here we give a proof by ch.f.
\end{remark}
\begin{proof}
    Let $\lambda_1(A_n)\geq\cdots\geq \lambda_{k_n}(A_n)$ be the eigenvalues of $A_n$, then 
    \begin{equation}
        \frac{X_n^T A_n X_n-\mathrm{E}X_n^T A_n X_n}{{[\mathrm{Var}(X_n^T A_n X_n)]}^{1/2}}=\sum_{i=1}^{k_n}\frac{\lambda_i(A_n)}{{\big[2\mathrm{tr}(A_n^2)\big]}^{1/2}}(Z_{ni}^2-1),
    \end{equation}
    where $Z_{ni}$'s ($i=1,\ldots,k_n$) are independent standard normal random variables.

    If~\ref{quadraticEigen} holds, then
    \begin{equation*}
        \begin{aligned}
            &\sum_{i=1}^{k_n}\mathrm{E}\Big[\frac{\lambda_i^2(A_n)}{2\mathrm{tr}(A_n^2)}{(Z_{ni}^2-1)}^2\Big\{\frac{\lambda_i^2(A_n)}{2\mathrm{tr}(A_n^2)}{(Z_{ni}^2-1)}^2\geq \epsilon\Big\}\Big]\\
            \leq&\sum_{i=1}^{k_n}
            \frac{\lambda_i^2(A_n)}{2\mathrm{tr}(A_n^2)}
            \mathrm{E}\Big[{(Z_{n1}^2-1)}^2\Big\{\frac{\lambda_{\max}(A_n^2)}{2\mathrm{tr}(A_n^2)}{(Z_{n1}^2-1)}^2\geq \epsilon\Big\}\Big]\\
            =&
            \frac{1}{2}\mathrm{E}\Big[{(Z_{n1}^2-1)}^2\Big\{\frac{\lambda_{\max}(A_n^2)}{2\mathrm{tr}(A_n^2)}{(Z_{n1}^2-1)}^2\geq \epsilon\Big\}\Big]\to 0.
        \end{aligned}
    \end{equation*}
    Hence~\ref{quadratic} follows by Lindeberg theorem.

    Conversely, if~\ref{quadratic} holds, we will prove that there is a subsequence of $\{n\}$ along which~\ref{quadraticEigen} holds. Then~\ref{quadraticEigen} will hold by a standard contradiction argument. 

    Denote $c_{ni}=\lambda_i(A_n)/{\big[2\mathrm{tr}(A_n^2)\big]}^{1/2}$ ($i=1,\ldots,k_n$), we have $c_{ni}\in[-\sqrt{2}/2,\sqrt{2}/2]$.
    Since~\ref{quadratic} holds, the characteristic function of
        $
        \sum_{i=1}^{k_n}c_{ni}(Z_{ni}^2-1)
    $
    converges to $\exp(-t^2/2)$ for every $t$. For $t\in (-1,1)$, we have
    \begin{equation*}
        \begin{aligned}
            &\log \mathrm{E}\exp{\big(it \sum_{j=1}^{k_n}c_{nj}(Z_{nj}^2-1)\big)}\\
            =&
            -i(\sum_{j=1}^{k_n}c_{nj})t-
            \frac{1}{2}\sum_{j=1}^{k_n}\log(1-i2c_{nj}t)\\
            =&
            -i(\sum_{j=1}^{k_n}c_{nj})t+
            \frac{1}{2}\sum_{j=1}^{k_n}\sum_{l=1}^{+\infty}\frac{1}{l}{(i2c_{nj}t)}^l\\
            =&
            -i(\sum_{j=1}^{k_n}c_{nj})t+
            \frac{1}{2}\sum_{l=1}^{+\infty}\Big[\sum_{j=1}^{k_n}{(c_{nj})}^l\Big]\frac{1}{l}{(i2t)}^l\\
            =&-\frac{1}{2}t^2+
            \frac{1}{2}\sum_{l=3}^{+\infty}\Big[\sum_{j=1}^{k_n}{(c_{nj})}^l\Big]\frac{1}{l}{(i2t)}^l.
        \end{aligned}
    \end{equation*}
    Denote $b_{nl}=\sum_{j=1}^{k_n}{(c_{nj})}^l$, $n=1,2,\cdots$ and $l=3,4,\cdots$. For $l\geq 3$, $\big|\sum_{j=1}^{k_n}{(c_{nj})}^l\big|\leq \big|\sum_{j=1}^{k_n}{(c_{nj})}^2\big|=1/2$.
    By Helly's selection theorem, there's a subsequence of $\{n\}$ along which $\lim_{n\to \infty}b_{nl}=b_l$ exists for every $l$.
    Apply dominated convergence theorem to this subsequence we have
            $\log \mathrm{E}\exp{\big(it \sum_{j=1}^{k_n}c_{nj}(Z_{nj}^2-1)\big)}\to
            -\frac{1}{2}t^2+
            \frac{1}{2}\sum_{l=3}^{+\infty}b_l\frac{1}{l}{(i2t)}^l$ for $t\in(-1/2,1/2)$.
            By the property of power series, we have $b_l=0$ for $l\geq 3$. Then~\ref{quadraticEigen} follows by noting that $b_{n4}\geq \max_j{(c_{nj})}^4$.
\end{proof}



\begin{theorem}
    Suppose $X_{ki}$'s are distributed as $N_p(0,\Sigma)$ ($k=1,2$, $i=1,2,\ldots,n_k$), $n_1/n_2\to\psi\in (0,+\infty)$. Then a necessary and sufficient condition for 
    \begin{equation}
        \frac{T_{CQ}-\mathrm{E}T_{CQ}}{{\big[\mathrm{Var}(T_{CQ})\big]}^{1/2}}\xrightarrow{L}N(0,1)
    \end{equation}
    is that
    \begin{equation}\label{ChenMaxEigen}
        \frac{\lambda_{\max}(\Sigma)}{{\big[\mathrm{tr}\Sigma^2\big]}^{1/2}}\to 0.
    \end{equation}

\begin{remark}
    The condition~\ref{ChenMaxEigen} is equivalent to condition~\ref{chenscondition}.
\end{remark}


    \begin{proof}
        Let $X_{ki}=\Sigma^{1/2}Z_{ki}$ with $Z_{ki}$ distributed as $N_p(0,I_{p})$. Denote $Z={[Z_{11},\ldots,Z_{1n_1},Z_{21},\ldots,Z_{2n_2}]}^T$. Then $T_{CQ}$ is a quadratic form of $Z$:
        \begin{equation}
            T_{CQ}=Z^T \big( B_n\otimes \Sigma \big) Z,
        \end{equation}
        where 
        \begin{equation}
            B_n=\begin{pmatrix}
                \frac{1}{n_1(n_1-1)}(n_1\alpha\alpha^T-I_{n_1})&
                -\frac{1}{\sqrt{n_1 n_2}}\alpha \beta^T\\
                -\frac{1}{\sqrt{n_1 n_2}}\beta \alpha^T&
                \frac{1}{n_2(n_2-1)}(n_2\beta\beta^T-I_{n_2})\\
            \end{pmatrix},
        \end{equation}
        $\alpha$ is an $n_1$ dimensional vector with all elements equal to $1/\sqrt{n_1}$ and $\beta$ is an $n_2$ dimensional vector with all elements equal to $1/\sqrt{n_2}$.

        By direct calculation, it can be seen that $B_n$'s eigenvalues are $-1/n_1(n_1-1)$, $-1/n_2(n_2-1)$, $(n_1+n_2)/n_1 n_2$ and $0$ with multiplicities $n_1-1$, $n_2-1$, $1$ and $1$ respectively.
        The eigenspace corresponding to $-1/n_1(n_1-1)$ is
        \begin{equation*}
            \{{(\eta^T,\underbrace{0,\ldots,0}_{n_2})}^T| \eta\,\, \textrm{is of}\,\, n_1\,\, \textrm{dimension and}\,\, \eta^T \alpha=0\}.
        \end{equation*}
        The eigenspace corresponding to $-1/n_2(n_2-1)$ is
        \begin{equation*}
            \{{(\underbrace{0,\ldots,0}_{n_1},\eta^T)}^T| \eta\,\, \textrm{is of}\,\, n_2\,\, \textrm{dimension and}\,\, \eta^T \beta=0\}.
        \end{equation*}
        The eigenvector corresponding to $(n_1+n_2)/n_1n_2$ is
        \begin{equation*}
            {\Big(-\sqrt{\frac{n_2}{n_1+n_2}}\alpha^T, \sqrt{\frac{n_1}{n_1+n_2}}\beta^T\Big)}^T.
        \end{equation*}
        The eigenvector corresponding to $0$ is
        \begin{equation*}
            {\Big(\sqrt{\frac{n_1}{n_1+n_2}}\alpha^T, \sqrt{\frac{n_2}{n_1+n_2}}\beta^T\Big)}^T.
        \end{equation*}
It follows that 
        \begin{equation*}
            \mathrm{tr}{(B_n\otimes \Sigma)}^2=\mathrm{tr}(B_n^2)\mathrm{tr}\Sigma^2=(\frac{1}{n_1(n_1-1)}+\frac{1}{n_1(n_1-1)}+\frac{2}{n_1 n_2})\mathrm{tr}\Sigma^2.
        \end{equation*}
And
        \begin{equation*}
            \lambda_{\max}\Big({(B_n\otimes \Sigma)}^2\Big)=\lambda_{\max}(B_n^2)\lambda_{\max}(\Sigma^2)={\Big(\frac{1}{n_1}+\frac{1}{n_2}\Big)}^2\lambda_{\max}(\Sigma^2).
        \end{equation*}
        The theorem follows by previous lemma.
    \end{proof}
\end{theorem}


\section{PCA Theory}

We give some PCA theory here. Compared with existing results, we impose less assumptions since our main task is to obtain the properties of principal space.
    \begin{assumption}\label{PCAassump}
    Suppose $X_i=\mu +U\Lambda^{1/2}Z_i$, $i=1,\ldots,n$, where $Z_{i}$ is a random vector with $p$ indepent entries with $\mathrm{E}Z_{ij}=0$ and $\mathrm{Var}Z_{ij}=1$, $i=1,\ldots,n$, $j=1,\ldots,p$, $\Lambda=\mathrm{diag}(\lambda_1,\ldots,\lambda_p)$ with $\lambda_1\geq\cdots\geq \lambda_p$ and $U$ is a $p$ dimensional orthogonal matrix.
    Suppose  $\lambda_{r+1}\asymp \lambda_p\asymp 1$. 
    \end{assumption}


    Denote by $S=\frac{1}{n}\sum_{i=1}^n (X_i-\bar{X}){(X_i-\bar{X})}^T$ the sample covariance matrix, $S=\hat{U}\hat{\Lambda}\hat{U}^T$ the spectral decomposition of $S$ where $\hat{\Lambda}=\mathrm{diag}(\hat{\lambda}_1,\ldots,\hat{\lambda}_p)$ and $\hat{U}$ is a orthogonal matrix.
    
    Let $u_i$ be the $i$th column of $U$, $1\leq i\leq p$. And denote $U=(V,\tilde{V})$, where $V$ is the first $r$ columns of $U$ and $\tilde{V}$ is the larst $p-r$ columns of $U$. Similarly define $\hat{u}_i$, $\hat{V}$ and $\hat{\tilde{V}}$.

The proof is similar to Dan Shen's paper.

    Let $Z\overset{def}{=}(Z_1,\ldots,Z_n)\overset{def}{=}{(\tilde{Z}_1^T,\ldots,\tilde{Z}_p^T)}^T$ and $\tilde{Z}_{(1)}={(\tilde{Z}_1^T,\ldots,\tilde{Z}_r^T)}^T$, $\tilde{Z}_{(2)}={(\tilde{Z}_{r+1}^T,\ldots,\tilde{Z}_p^T)}^T$.
    Let $\Lambda_{(1)}=\mathrm{diag}(\lambda_1,\ldots,\lambda_r)$ and $\Lambda_{(2)}=\mathrm{diag}(\lambda_{r+1},\ldots,\lambda_p)$. Define $\hat{\Lambda}_{(1)}$ and $\hat{\Lambda}_{(2)}$ in a similar way.
    Deote by $J$ the $p\times p$ matrix with all elements equal to $1$. Then
    \begin{equation*}
        S=\frac{1}{n}U\Lambda^{1/2}Z(I-\frac{1}{n}J)Z^T \Lambda^{1/2} U^T.
    \end{equation*}
    The positive eigenvalues of $S$ are the same as those of  $S^*=\frac{1}{n}(I-\frac{1}{n}J)Z^T \Lambda Z (I-\frac{1}{n}J)$, the dual matrix of $S$.


\begin{theorem}
    Suppose Assumption~\ref{PCAassump} holds and $p/n\to \infty$.
    For $1\leq i\leq r$, we have:

    If  $\frac{p}{n\lambda_i}\to 0$, then
    $\frac{\hat{\lambda}_i}{\lambda_i}\xrightarrow{a.s.} 1$.
    If  $\frac{p}{n\lambda_i}\to \infty$, then
    $\hat{\lambda}_i\asymp\frac{p}{n}$ almost surely.
\end{theorem}
\begin{proof}
    Let
    \begin{equation*}
        S^*=\frac{1}{n}(I-\frac{1}{n}J)\tilde{Z}_{(1)}^T \Lambda_{(1)} \tilde{Z}_{(1)} (I-\frac{1}{n}J)+\frac{1}{n}(I-\frac{1}{n}J)\tilde{Z}_{(2)}^T \Lambda_{(2)} \tilde{Z}_{(2)} (I-\frac{1}{n}J)\overset{def}{=}A+B.
    \end{equation*}

    For $1\leq i\leq r$, we prove the $i$th sample eigenvalue is consistent. By Corollary~\ref{WeylCor}, we have
    \begin{equation*}
        \begin{aligned}
            \frac{\lambda_i(A)}{\lambda_i}&\leq \frac{1}{\lambda_i}\lambda_{\max}\Big(\frac{1}{n}(I-\frac{1}{n}J)\tilde{Z}_{(1)}^T \mathrm{diag}(\underbrace{0,\ldots,0}_{i-1},\underbrace{\lambda_i,\ldots,\lambda_i}_{r-i+1}) \tilde{Z}_{(1)} (I-\frac{1}{n}J)\Big)\\
            &\leq \lambda_{\max}\Big(\frac{1}{n} \tilde{Z}_{(1)} (I-\frac{1}{n}J)\tilde{Z}_{(1)}^T \Big)\\
        \end{aligned}
    \end{equation*}
    But $\frac{1}{n}\tilde{Z}_{(1)} (I-\frac{1}{n}J)\tilde{Z}_{(1)}^T\xrightarrow{a.s.}I_r$ by law of large number. It follows that the right hand side tends to $1$ almost surely. On the other hand, by Weyl's inequility, we have

\begin{equation*}
    \begin{aligned}
        \frac{\lambda_i(A)}{\lambda_i}&\geq
        \frac{1}{\lambda_i}\lambda_{i}\Big(\frac{1}{n}(I-\frac{1}{n}J)\tilde{Z}_{(1)}^T \mathrm{diag}(\underbrace{\lambda_i,\ldots,\lambda_i}_{i},\underbrace{0,\ldots,0}_{r-i}) \tilde{Z}_{(1)} (I-\frac{1}{n}J)\Big).\\
    \end{aligned}
\end{equation*}
    The right hand side tends to $1$ almost surely. As a result, $\lambda_i(A)/\lambda_i$ tends to $1$ almost surely.


    By Weyl's inequality, we have that
    \begin{equation}\label{PCA19}
        \frac{\max(\lambda_i(A),\lambda_i(B))}{\lambda_i}\leq\frac{\lambda_i(S)}{\lambda_i}\leq \frac{\lambda_i(A)}{\lambda_i} +\frac{\lambda_{\max}(B)}{\lambda_i}.
    \end{equation}
    By Bai Yin's law, $\frac{n}{p}\lambda_{i}(B)\asymp 1$ almost surely for $1\leq i\leq n-1$. It follows that 
    \begin{equation}\label{PCA20}
        \frac{\lambda_{i}(B)}{\lambda_i}\asymp\frac{p}{n\lambda_i}
        \quad \textrm{and}\quad
        \frac{\lambda_{\max}(B)}{\lambda_i}\asymp\frac{p}{n\lambda_i},
    \end{equation}
    where $1\leq i\leq n-1$. The theorem follows from~\eqref{PCA19} and~\eqref{PCA20}.
\end{proof}



\begin{theorem}
    Suppose Assumption~\ref{PCAassump} holds and $p/n\to \infty$.
    If  $\frac{p}{n\lambda_r}\to 0$, then almost surely we have

    %\begin{equation}\label{PCAINeedThis}
    %    \mathrm{tr}\tilde{V}^T \hat{V}\Lambda_{(1)}\hat{V}^T \tilde{V}\asymp \frac{p}{n}
    %\end{equation}
%and
    \begin{equation}\label{PCAtheorem101}
        \|\hat{V}\hat{V}^T-VV^T\|^2_F\asymp\frac{p}{n\lambda_r }.
    \end{equation}
    If  $\frac{p}{n\lambda_r}\to \infty$, then
    \begin{equation}\label{PCAtheorem102}
        r-\frac{1}{2}\|\hat{V}\hat{V}^T-VV^T\|^2_F=O_{a.s.}(\frac{n\lambda_1}{p}).
    \end{equation}
\end{theorem}


\begin{proof}

    Since
    \begin{equation*}
        S=\hat{U}\hat{\Lambda}\hat{U}^T=
        \frac{1}{n}U\Lambda^{1/2}Z(I-\frac{1}{n}J)Z^T \Lambda^{1/2} U^T,
    \end{equation*}
    we have
    \begin{equation}\label{crucialEqInPCA1}
        \Lambda^{-1/2}U^T \hat{U}\hat{\Lambda}\hat{U}^T U\Lambda^{-1/2}=
        \frac{1}{n}Z(I-\frac{1}{n}J)Z^T. 
    \end{equation}
    First, we prove~\eqref{PCAtheorem101}.
    It follows from~\eqref{crucialEqInPCA1} that
    \begin{equation}\label{crucialEqInPCA3}
        \Lambda^{-1/2}_{(2)}\tilde{V}^T \hat{U}\hat{\Lambda}\hat{U}^T \tilde{V}\Lambda^{-1/2}_{(2)}=
        \frac{1}{n}\tilde{Z}_{(2)}(I-\frac{1}{n}J)\tilde{Z}_{(2)}^T.
    \end{equation}

    The left hand side of~\eqref{crucialEqInPCA3} equals to $C+D$, where 
    $C= \Lambda^{-1/2}_{(2)}\tilde{V}^T \hat{V}\hat{\Lambda}_{(1)}\hat{V}^T \tilde{V}\Lambda^{-1/2}_{(2)}$ 
    and 
    $D= \Lambda^{-1/2}_{(2)}\tilde{V}^T \hat{\tilde{V}}\hat{\Lambda}_{(2)}\hat{\tilde{V}}^T \tilde{V}\Lambda^{-1/2}_{(2)}$.
    Note that $\mathrm{Rank}(C)=r$, $\mathrm{Rank}(D)=n-1-r$ and $\mathrm{Rank}(C+D)=n-1$. By Bai Yin's law, we have that
    \begin{equation*}
        \lambda_1\big(\frac{1}{p}\tilde{Z}_{(2)}(I-\frac{1}{n}J)\tilde{Z}_{(2)}^T\big)\to 1,\quad
        \lambda_{n-1}\big(\frac{1}{p}\tilde{Z}_{(2)}(I-\frac{1}{n}J)\tilde{Z}_{(2)}^T\big)\to 1\quad  a.s..
    \end{equation*}
    By Lemma~\ref{lemmaRankLim}, $\lambda_{1}(C)\xrightarrow{a.s.}1$ and $\lambda_{r}(C)\xrightarrow{a.s.}1$. It follows that
    \begin{equation}\label{PCA27eq}
        \frac{n}{p}\hat{\Lambda}_{(1)}^{1/2}\hat{V}^T \tilde{V}\Lambda^{-1}_{(2)}\tilde{V}^T \hat{V}\hat{\Lambda}_{(1)}^{1/2}\xrightarrow{a.s.} I_r.
    \end{equation}
    When $\frac{p}{n\lambda_r}\to 0$, $\hat{\lambda}_i$'s are ratio consistent for $1\leq i\leq r$. That is, $\Lambda_{(1)}^{-1}\hat{\Lambda}_{(1)}\to I_r$ almost surely. Then it follows from~\eqref{PCA27eq} that
    \begin{equation}
        \frac{n}{p}\Lambda_{(1)}^{1/2}\hat{V}^T \tilde{V}\Lambda^{-1}_{(2)}\tilde{V}^T \hat{V}\Lambda_{(1)}^{1/2}\xrightarrow{a.s.} I_r.
    \end{equation}

    %Since $\Lambda_{(2)}$ is bounded from below and above,~\eqref{PCAINeedThis} holds.
    Notice that
    \begin{equation*}
        \begin{aligned}
         \frac{n}{p}   \mathrm{tr}\big(\Lambda_{(1)}^{1/2}\hat{V}^T \tilde{V}\Lambda^{-1}_{(2)}\tilde{V}^T \hat{V}\Lambda_{(1)}^{1/2}\big)&\geq
          \frac{n}{p}  \lambda_r\mathrm{tr}\big(\hat{V}^T \tilde{V}\Lambda^{-1}_{(2)}\tilde{V}^T \hat{V}\big)
            \geq
          \frac{n}{p}  e_r^T \hat{\Lambda}_{(1)}^{1/2}\hat{V}^T\tilde{V}\Lambda_{(2)}^{-1}\tilde{V}^T\tilde{V}\hat{\Lambda}_{(1)}^{1/2}e_1
        \end{aligned}
    \end{equation*}
    where $e_r=(\underbrace{0,\ldots,0}_{r-1},1)$. It follows that the medium term is bounded above and below asymptotically. Notice that
    \begin{equation*}
        \begin{aligned}
            \frac{n}{p}\lambda_r\mathrm{tr}\big(\hat{V}^T \tilde{V}\Lambda^{-1}_{(2)}\tilde{V}^T \hat{V}\big)
            &\asymp
            \frac{n}{p}\lambda_r\mathrm{tr}\big(\hat{V}^T \tilde{V}\tilde{V}^T \hat{V}\big)
            =\frac{n}{p}\lambda_r\frac{1}{2}\|VV^T -\hat{V}\hat{V}^T\|^2_F.
        \end{aligned}
    \end{equation*}
     Therefore
    $\|VV^T -\hat{V}\hat{V}^T\|^2_F\asymp \frac{p}{n\lambda_r}$ almost surely.

    Then we prove~\eqref{PCAtheorem102}. It follows from~\eqref{crucialEqInPCA1} that
    \begin{equation}\label{crucialEqInPCA2}
        \Lambda^{-1/2}_{(1)}V^T \hat{U}\hat{\Lambda}\hat{U}^T V\Lambda^{-1/2}_{(1)}=
        \frac{1}{n}\tilde{Z}_{(1)}(I-\frac{1}{n}J)\tilde{Z}_{(1)}^T\xrightarrow{a.s.} I_{r}.
    \end{equation}
But
    \begin{equation}
        \begin{aligned}
        \mathrm{tr}\big(\Lambda^{-1/2}_{(1)}V^T \hat{U}\hat{\Lambda}\hat{U}^T V\Lambda^{-1/2}_{(1)}\big)
            &\geq
        \mathrm{tr}\big(\Lambda^{-1/2}_{(1)}V^T \hat{V}\hat{\Lambda}_{(1)}\hat{V}^T V\Lambda^{-1/2}_{(1)}\big)
            \\
            &\geq
            \frac{\hat{\lambda}_r}{\lambda_1}\Big(r-\frac{1}{2}\|\hat{V}\hat{V}^T-VV^T\|^2_F\Big).
        \end{aligned}
    \end{equation}
When $\frac{p}{n\lambda_r}\to \infty$, $\hat{\lambda}_r\asymp p/n$.Then~\eqref{PCAtheorem102} holds.

\end{proof}



Suprisingly, from our proof we can see that the error of PCA can be estimated well!


\section{The new theory}
\begin{lemma}
    if 
    \begin{equation}
        \frac{\lambda_1(\Sigma_i)}{{[\mathrm{tr}(\Sigma_i^2)]}^{1/2}}\to 0,
    \end{equation}
    $i=1,2$. Then
    \begin{equation}
        \mathrm{tr}(\Sigma_i\Sigma_j\Sigma_l\Sigma_h)=o[\mathrm{tr}^2 \{{(\Sigma_1+\Sigma_2)}^2\}]\quad \textrm{for} \,\, i,j,l,h=1\,\,\textrm{or}\,\, 2.
    \end{equation}
\end{lemma}
\begin{proof}
    There are totally $6$ possibilities of 
    $\mathrm{tr}(\Sigma_i\Sigma_j\Sigma_l\Sigma_h)$:
    \begin{equation*}
        \begin{aligned}
    \mathrm{tr}(\Sigma_1\Sigma_1\Sigma_1\Sigma_1)\quad
    \mathrm{tr}(\Sigma_1\Sigma_1\Sigma_1\Sigma_2)\quad
    \mathrm{tr}(\Sigma_1\Sigma_1\Sigma_2\Sigma_2)\\
    \mathrm{tr}(\Sigma_1\Sigma_2\Sigma_1\Sigma_2)\quad
    \mathrm{tr}(\Sigma_1\Sigma_2\Sigma_2\Sigma_2)\quad
    \mathrm{tr}(\Sigma_2\Sigma_2\Sigma_2\Sigma_2)\\
        \end{aligned}
    \end{equation*}
    For $\mathrm{tr}(\Sigma_1\Sigma_2\Sigma_1\Sigma_2)$ we have
    \begin{equation*}
        \begin{aligned}
            \frac{\mathrm{tr}(\Sigma_1\Sigma_2\Sigma_1\Sigma_2)}{\mathrm{tr}^2 \{{(\Sigma_1+\Sigma_2)}^2\}}&=
            \frac{\mathrm{tr}(\Sigma_1^{1/2}\Sigma_2\Sigma_1\Sigma_2\Sigma_1^{1/2})}{\mathrm{tr}^2 \{{(\Sigma_1+\Sigma_2)}^2\}}\\
            &\leq
            \frac{\lambda_1(\Sigma_1)\mathrm{tr}(\Sigma_1^{1/2}\Sigma_2^2\Sigma_1^{1/2})}{\mathrm{tr}^2 \{{(\Sigma_1+\Sigma_2)}^2\}}\\
            &\leq
            \frac{\lambda_1^2(\Sigma_1)\mathrm{tr}(\Sigma_2^2)}{\mathrm{tr}^2 \{{(\Sigma_1+\Sigma_2)}^2\}}\\
            &\leq
            \frac{\lambda_1^2(\Sigma_1)\mathrm{tr}(\Sigma_2^2)}{\mathrm{tr} (\Sigma_1^2)\mathrm{tr} (\Sigma_2^2)}\to 0.
        \end{aligned}
    \end{equation*}
    The other cases can be proved similarly.
\end{proof}
Define
    \begin{equation}
        T(A)=\frac{\sum_{i\neq j}^{n_1}X_{1i}^T AA^T X_{1j}}{n_1(n_1-1)}+
        \frac{\sum_{i\neq j}^{n_2}X_{2i}^T AA^T X_{2j}}{n_2(n_2-1)}-
        2\frac{\sum_{i=1}^{n_1}\sum_{j=1}^{n_2}X_{1i}^T   AA^T X_{2j}}{n_1n_2}.
    \end{equation}
    Let
    \begin{equation}
        W(A)=\frac{2}{n_1(n_1-1)}\mathrm{tr}{(A^T \Sigma_1 A)}^2+
        \frac{2}{n_2(n_2-1)}\mathrm{tr}{(A^T \Sigma_2 A)}^2+
\frac{4}{n_1 n_2}\mathrm{tr}(A^T \Sigma_1 A A^T \Sigma_2 A).
    \end{equation}

\begin{theorem}
    Suppose $\hat{\tilde{V}}$ is a random $(p-\hat{r})\times p$ orthgonal matrix which is independent of the data, $\hat{r}\geq \max(r_1,r_2)$ and is bounded. $n_1/n_2\to \psi$. Then a sufficient  condition of 
    \begin{equation}\label{newTheoryResult}
        \frac{T(\hat{\tilde{V}})-\|\hat{\tilde{V}}^T(\mu_1-\mu_2)\|^2}
        {\sqrt{W(\hat{\tilde{V}})}}
        \xrightarrow{\mathcal{L}} N(0,1)
    \end{equation}
    is that
    \begin{equation}\label{newTheoryCondition}
        %\max\big(\lambda_1(\Sigma_1),\lambda_1(\Sigma_2)\big)\|VV^T-\hat{V}\hat{V}^T\|_F^2\to 0.
        \lambda_1(\Sigma_i)\mathrm{tr} (\hat{\tilde{V}}^T V_i V_i^T \hat{\tilde{V}})=o_P(\sqrt{p}),
    \end{equation}
    and
    \begin{equation}
        {(\mu_1-\mu_2)}^T \hat{\tilde{V}}\hat{\tilde{V}}^T \Sigma_i \hat{\tilde{V}}\hat{\tilde{V}}^T (\mu_1-\mu_2)=o_P(\frac{p}{n})
    \end{equation}
    $i=1,2$.
    If we further assume $\lambda_1(\Sigma)\asymp \lambda_r(\Sigma)$, $\Sigma_1=\Sigma_2$, $\mu_1=\mu_2$ and data are from normal distribution, then~\eqref{newTheoryCondition} is also neccessary.
\end{theorem}
\begin{proof}
    We only need to prove that~\eqref{newTheoryResult} holds conditioning on $\hat{\tilde{V}}$.
    Notice that $T(\hat{\tilde{V}})$ can be regard as the $T_{CQ}$ of data $\hat{\tilde{V}}^T X_{i,j}$.
    Since $\hat{\tilde{V}}^T X_{i,j}=\hat{\tilde{V}}^T U_i\Lambda^{1/2}_{i}Z_{i,j}$. Denote $\Sigma_i^*=\hat{\tilde{V}}^T\Sigma_i\hat{\tilde{V}}$, $i=1,2$.
    By Chen Qin's theorem, we only need to check condition (3.6) (for $\Sigma_i^*$) and (3.4) of Chen Qin's paper.
    Since 
    \begin{equation}
        \begin{aligned}
            \Sigma_i^*&=\hat{\tilde{V}}^T V_i \Lambda_{i,(1)}V_i^T \hat{\tilde{V}}+
    \hat{\tilde{V}}^T \tilde{V}_i \Lambda_{i,(2)}\tilde{V}_i^T\hat{\tilde{V}}\\
            &\overset{def}{=}A_1+A_2
        \end{aligned}
    \end{equation}

First step: deal with $A_2$.

     For $j=1,\ldots,p-\hat{r}$, we have 
    \begin{equation} 
        \begin{aligned}
            \lambda_j(A_2)&=
        \lambda_j(\Lambda_{i,(2)}^{1/2}\tilde{V}_i^T\hat{\tilde{V}}\hat{\tilde{V}}^T \tilde{V}_i\Lambda_{i,(2)}^{1/2})
           \\ 
            &=\lambda_j(\Lambda_{i,(2)}^{1/2}\tilde{V}_i^T \tilde{V}_i\Lambda_{i,(2)}^{1/2}-\Lambda_{i,(2)}^{1/2}\tilde{V}_i^T \hat{V}\hat{V}^T\tilde{V}_i\Lambda_{i,(2)}^{1/2})
        \end{aligned}
    \end{equation}
    Since the rank of $\Lambda_{i,(2)}^{1/2}\tilde{V}_i^T \hat{V}\hat{V}^T\tilde{V}_i\Lambda_{i,(2)}^{1/2}$ is at most $\hat{r}$, we have for $j=1,\ldots,p-2\hat{r}$ that 
    \begin{equation}
        \lambda_{j+\hat{r}}(\Lambda_{i,(2)}^{1/2}\tilde{V}_i^T \tilde{V}_i\Lambda_{i,(2)}^{1/2})
            \leq\lambda_j(A_2)\leq
            \lambda_j(\Lambda_{i,(2)}^{1/2}\tilde{V}_i^T \tilde{V}_i\Lambda_{i,(2)}^{1/2}).
    \end{equation}
    Note that $\lambda_j(\Lambda_{i,(2)}^{1/2}\tilde{V}_i^T \tilde{V}_i\Lambda_{i,(2)}^{1/2})=\lambda_j( \tilde{V}_i\Lambda_{i,(2)}\tilde{V}_i^T)=\lambda_{j+r_i}(\Sigma_i)$,
    $j=1,\ldots,p-r_i$. Hence we have
    \begin{equation}
        \lambda_{j+\hat{r}+r_i}(\Sigma_i)\leq \lambda_j(A_2)\leq \lambda_{j+r_i}(\Sigma_i),
    \end{equation}
    for $j=1,\ldots,p-2\hat{r}$.
Since $\mathrm{rank}(A_1)\leq r_i$, we have for $j=r_i+1,\ldots,p-2\hat{r}$ that
    \begin{equation}\label{mainLimitTheoremIne1}
        \lambda_{j+\hat{r}+r_i}(\Sigma_i)\leq \lambda_j(A_2)\leq\lambda_j(\Sigma_i^*)\leq \lambda_{j-r_i}(A_2)\leq \lambda_j(\Sigma_i).
    \end{equation}
    Specially, $\lambda_j(\Sigma_i^*)\asymp 1$, $j=r_i+1,\ldots,p-2\hat{r}$.

    By~\eqref{mainLimitTheoremIne1}, 
    \begin{equation}
        \sum_{j=\hat{r}+2r_i+1}^{p-\hat{r}+r_i}\lambda_j^2(\Sigma_i)\leq
\sum_{j=r_i+1}^{p-2\hat{r}}\lambda_j^2(\Sigma_i^*)\leq
\sum_{j=r_i+1}^{p-2\hat{r}}\lambda_j^2(\Sigma_i).
    \end{equation}
    Hence 
    \begin{equation}\label{mainLimitTheoremIne2}
        \frac{\sum_{j=r_i+1}^{p-2\hat{r}}\lambda_j^2(\Sigma_i^*)}{\sum_{j=r_i+1}^{p}\lambda_j^2(\Sigma_i)}\xrightarrow{P}1.
    \end{equation}

Below we will use the assumption of PCA rate.
    \begin{equation}
        \lambda_1(A_1)\leq \lambda_1(\Sigma_i)\mathrm{tr} (\hat{\tilde{V}}^T V_i V_i^T \hat{\tilde{V}}).
    \end{equation}

     And 
    \begin{equation}
        \lambda_1(\Sigma_i^*)\leq \lambda_1(A_1)+\lambda_1(A_2)\leq o_P(\sqrt{p})+\lambda_{r_i+1}(\Sigma_i)=o_P(\sqrt{p}).
    \end{equation}
    Since
    \begin{equation*}
        \begin{aligned}
            \sum_{j=r_i+1}^{p-2\hat{r}}\lambda_j^2(\Sigma_i^*)\leq \mathrm{tr}(\Sigma_i^{*2})
            \leq r_i \lambda_1^2(\Sigma_i^*)+
            \sum_{j=r_i+1}^{p-2\hat{r}}\lambda_j^2(\Sigma_i^*)+
            2\hat{r}\lambda_{p-2\hat{r}+1}^2(\Sigma_i^*).
        \end{aligned}
    \end{equation*}
    But $\sum_{j=r_i+1}^{p-2\hat{r}}\lambda_j^2(\Sigma_i^*)\asymp p$, $\lambda_1^2(\Sigma_i^*)=o_P(p)$, and $2\hat{r}\lambda_{p-2\hat{r}+1}^2(\Sigma_i^*)=O_P(1)$ since $\hat{r}$ is bounded.
    It follows that
    \begin{equation}
        \frac{\mathrm{tr}(\Sigma_i^{*2})}{\sum_{j=r_i+1}^{p-2\hat{r}}\lambda_j^2(\Sigma_i^*)}\xrightarrow{P}1.
    \end{equation}

    It follows that
    \begin{equation}
        \frac{\mathrm{tr}(\Sigma_i^{*2})}{\sum_{j=r_i+1}^{p}\lambda_j^2(\Sigma_i)}\xrightarrow{P}1.
    \end{equation}

    Notice that
    \begin{equation}
        \begin{aligned}
            \frac{\lambda_1(\Sigma_i^*)}{{[\mathrm{tr}(\Sigma_i^{*2})]}^{1/2}}\leq 
            \frac{o_P(\sqrt{p})}{{[\sum_{j=r_i+1}^{p-2\hat{r}}\lambda_j^2(\Sigma_i^*)]}^{1/2}}
            \asymp
            \frac{o_P(\sqrt{p})}{\sqrt{p}}\to 0.
        \end{aligned}
    \end{equation}
By lemma, the first condition of Chen and Qin is satisfied.


    \begin{equation*}
        \begin{aligned}
        \frac{{(\mu_1-\mu_2)}^T \hat{\tilde{V}}\Sigma_i^*\hat{\tilde{V}}^T (\mu_1-\mu_2)}{n^{-1}\mathrm{tr}\{{(\Sigma_1^*+\Sigma_2^*)}^2\}}
            &\leq
            \frac{{(\mu_1-\mu_2)}^T \hat{\tilde{V}}\Sigma_i^*\hat{\tilde{V}}^T (\mu_1-\mu_2)}{n^{-1}\mathrm{tr}(\Sigma_1^{*2})}\\
            &\asymp
            \frac{{(\mu_1-\mu_2)}^T \hat{\tilde{V}}\Sigma_i^*\hat{\tilde{V}}^T (\mu_1-\mu_2)}{n^{-1}p}\\
        \end{aligned}
    \end{equation*}

     Neccessaity is in the conditional sense. Suppose
\begin{equation}
        \frac{T(\hat{\tilde{V}})-\|\hat{\tilde{V}}^T(\mu_1-\mu_2)\|^2}
        {\sqrt{W(\hat{\tilde{V}})}}
        \xrightarrow{\mathcal{L}} N(0,1)
\end{equation}
Write $T(\hat{\tilde{V}})=T^{(1)}+T^{(2)}$, where
\begin{equation*}
\begin{aligned}
    T^{(1)}=&\frac{\sum_{i\neq j}^{n_1}{(X_{1i}-\mu_1)}^T\hat{\tilde{V}}\hat{\tilde{V}}^T(X_{1j}-\mu_1)}{n_1(n_1-1)}+\frac{\sum_{i\neq j}^{n_2}{(X_{2i}-\mu_2)}^T\hat{\tilde{V}}\hat{\tilde{V}}^T(X_{2j}-\mu_2)}{n_2(n_2-1)}
\\
    &-2\frac{\sum_{i=1}^{n_1}\sum_{j=1}^{n_2}{(X_{1i}-\mu_1)}^T\hat{\tilde{V}}\hat{\tilde{V}}^T(X_{2j}-\mu_2)}{n_1n_2}
\end{aligned}
\end{equation*}
and
\begin{equation*}
\begin{aligned}
    T^{(2)}=&2{(\bar{X_{1}}-\mu_1)}^T\hat{\tilde{V}}\hat{\tilde{V}}^T(\mu_1-\mu_2)+
    2{(\bar{X_{2}}-\mu_2)}^T\hat{\tilde{V}}\hat{\tilde{V}}^T(\mu_2-\mu_1)
\\
    &+\|\hat{\tilde{V}}(\mu_1-\mu_2)\|^2.
\end{aligned}
\end{equation*}
   
Since $W(\hat{\tilde{V}})\geq \frac{2}{n_1(n_1-1)}\mathrm{tr}(\Sigma_1^{*2})\asymp\frac{p}{n^2}$. 
We have
\begin{equation*}
    \begin{aligned}
    \mathrm{Var}\Big(\frac{T^{(2)}-\|\hat{\tilde{V}}(\mu_1-\mu_2)\|^2}{\sqrt{W(\hat{\tilde{V}})}}\Big)
        =&
        \frac{4}{W(\hat{\tilde{V}})}\Big(
        {(\mu_1-\mu_2)}^T\hat{\tilde{V}}\hat{\tilde{V}}^T       \frac{1}{n_1}\Sigma_1\hat{\tilde{V}}\hat{\tilde{V}}^T(\mu_1-\mu_2)\\
        &+
        {(\mu_1-\mu_2)}^T\hat{\tilde{V}}\hat{\tilde{V}}^T       \frac{1}{n_2}\Sigma_2\hat{\tilde{V}}\hat{\tilde{V}}^T(\mu_1-\mu_2)
        \Big)\\
        =&o_P(1)\frac{n^2}{p}\frac{1}{n}\frac{p}{n}\xrightarrow{P}0.
    \end{aligned}
\end{equation*}
And $\frac{T^{(2)}-\|\hat{\tilde{V}}(\mu_1-\mu_2)\|^2}{\sqrt{W(\hat{\tilde{V}})}}$ has mean zero, hence converges to $0$ in probability. By Slutsky's theorem, 
\begin{equation}
    \frac{T^{(1)}}
        {\sqrt{W(\hat{\tilde{V}})}}
        \xrightarrow{\mathcal{L}} N(0,1).
\end{equation}

It follows by previous theorem that
$\frac{\lambda_1(\Sigma^*)}{{[\mathrm{tr}\Sigma^{*2}]}^{1/2}}\to 0$.Then
\begin{equation}
    \frac{\sum_{i=1}^{r}\lambda_i^2(\Sigma^*)}{\mathrm{tr}\Sigma^{*2}}\leq \frac{r\lambda_1^2(\Sigma^*)}{\mathrm{tr}\Sigma^{*2}}\to 0.
\end{equation} 
    Then
\begin{equation}
    \frac{\sum_{i=1}^{r}\lambda_i^2(\Sigma^*)}{\sum_{i=r+1}^{p}\lambda_i^2(\Sigma^*)}\to 0,
\end{equation} 
which is equivalent to $\lambda$.
By~\eqref{mainLimitTheoremIne2}, $\sum_{i=1}^{r}\lambda_i^2(\Sigma^*)=o(p)$. Then
\begin{equation}
    \lambda_1(A_1)\leq \lambda_1^2(\Sigma^*)\leq \sum_{i=1}^{r}\lambda_i^2(\Sigma^*)=o(p)
\end{equation}
But
\begin{equation}
    \lambda_1(A_1)\asymp
    \lambda_1(\Sigma_i)\lambda_1(\hat{\tilde{V}}^T V V^T\hat{\tilde{V}})
    \asymp
    \lambda_1(\Sigma_i)\lambda_1(\hat{\tilde{V}}^T V V^T \hat{\tilde{V}})
\end{equation}
The first equivalence of above holds by assumption. The second equivalence holds because $\mathrm{rank}(\hat{\tilde{V}}^T V V^T \hat{\tilde{V}})\leq r$. The conclusion follows.
\end{proof}



The rest of the Appendix is devoted to the proof of propositions and theorems in the paper.
\begin{proof}[\textbf{Proof Of Proposition 1}]
Since $V$ and $\tilde{V}$ are orthogonal, we have
  $$\tilde{V}^T X_{ki}\sim N(\tilde{V}\mu_k,\sigma^2 I_{p-r}).$$
 By~\cite{Chen2010A}'s section 6.1, we have
\begin{equation*}
Var(T_1)=2\tau^2 p\sigma^4(1+o(1)).
\end{equation*}
Random sequences $\tilde{V}^T X_{ki}$ fulfill the condition of~\cite{Chen2010A}'s theorem 1, hence the conclusion follows.
\end{proof}



% consistency of variance estimator 1
\begin{proof}[\textbf{Proof Of Proposition 2}]
    By a standard orthogonal transformation, $\hat{\sigma}_{(1)}^2$ has the same law with respect to $\frac{1}{(n_1+n_2-2)(p-r)}\mathrm{tr}\sum_{k=1}^2\sum_{i=1}^{n_i-1}\hat{\tilde{V}}^T Y_{ki}Y_{ki}^T\hat{\tilde{V}}$,
    where $Y_{ki}=VDU^{*}_{ki}+Z_{ki}^*$. Here
    $U_{ki}^{*}$'s are i.i.d.\ random vectors with $r$ dimensional standard normal distribution and $Z_{ki}^{*}$'s are i.i.d.\ random vectors distributed as  $N_p(0,\sigma^2 I_p)$ which are independent of $U_{ki}^{*}$'s.
    Denote $U^{*}={(U^*_{11},\ldots,U^{*}_{1(n_1-1)},U^*_{21},\ldots,U^*_{2(n_2-1)})}^T$ and 
    $Z^*={(Z^*_{11},\ldots,Z^*_{1(n_1-1)},Z^*_{21},\ldots,Z^*_{2(n_2-1)})}^T$. Hence we have
        \begin{align*}
            \hat{\sigma}_{(1)}^2=&\frac{1}{(n_1+n_2-2)(p-r)}\|U^*DV^T\hat{\tilde{V}}\|_F^2+\frac{1}{(n_1+n_2-2)(p-r)}\|Z^*\hat{\tilde{V}}\|^2_F \\
            &+\frac{2}{(n_1+n_2-2)(p-r)}\mathrm{tr}\hat{\tilde{V}}^T Z^{*T}U^*DV^T\hat{\tilde{V}}
            \\
            \overset{\textrm{def}}{=}&R_1+R_2+R_3.
            %\frac{1}{(n-1)(p-r)}tr\hat{\tilde{V}}^T VDU^{*T}U^*DV^T\hat{\tilde{V}}^T+\frac{1}{(n-1)(p-r)}tr\hat{\tilde{V}}^TZ^{*T}Z^*\hat{\tilde{V}}\\
        \end{align*}

    By Lemma~\ref{maxEigen}, we have $\lambda_1(U^{*T}U^*)=O_P(n_1+n_2-2)$. Therefore,
    \begin{equation}
        \begin{aligned}
            R_1&=\frac{1}{(n_1+n_2-2)(p-r)}\mathrm{tr}\hat{\tilde{V}}^T VDU^{*T}U^*DV^T\hat{\tilde{V}}\\
            &=O_P(1)\frac{1}{p-r}\mathrm{tr}\hat{\tilde{V}}^T VD^2V^T\hat{\tilde{V}}\\
            &=O_P(1)\frac{1}{p-r}\mathrm{tr} D^2V^T(I-\hat{V}\hat{V}^T)V\\
            &\leq O_P(1)\frac{1}{p}\sqrt{\mathrm{tr}D^4}\sqrt{\mathrm{tr}{(V^T(I-\hat{V}\hat{V}^T)V)}^2}.
        \end{aligned}
    \end{equation}
    And
    \begin{align}
        \mathrm{tr} {(V^T(I-\hat{V}\hat{V}^T)V)}^2
            &\leq {(\mathrm{tr} V^T(I-\hat{V}\hat{V}^T)V)}^2 \label{eq:biti1}\\
            &=\frac{1}{4}\|VV^T-\hat{V}\hat{V}^T\|_F^4 \label{eq:biti2}\\
            &=O(\frac{p^2}{p^{2\beta}{(n_1+n_2)}^2}).\notag
    \end{align}
    The inequality~\eqref{eq:biti1} holds because $V^T(I-\hat{V}\hat{V}^T)V$ is positive semi-definite and equality~\eqref{eq:biti2} is by the fact that $\mathrm{tr} V^T(I-\hat{V}\hat{V}^T)V=\frac{1}{2}\|VV^T-\hat{V}\hat{V}^T\|^2_F$. 
    Since $\mathrm{tr}D^4=O_P(p^{2\beta})$, it follows that $R_1=O_P(\frac{1}{n_1+n_2})\xrightarrow{P}0$.

    We note that
    \begin{equation}
        \begin{aligned}
            &|R_2-\frac{1}{(n_1+n_2-2)(p-r)}\|Z^*\tilde{V}\|_F^2|\\
            &=
            \frac{1}{(n_1+n_2-2)(p-r)}|\mathrm{tr}Z^{*T}Z^*(\hat{\tilde{V}}\hat{\tilde{V}}^T-\tilde{V}\tilde{V}^T)|\\
            &=\frac{1}{(n_1+n_2-2)(p-r)}|\mathrm{tr}Z^{*T}Z^*(\hat{V}\hat{V}^T-VV^T)|\\
            &\leq \frac{1}{(n_1+n_2-2)(p-r)}\|Z^{*T}Z^*\|_F\|\hat{V}\hat{V}^T-VV^T\|_F.
        \end{aligned}
    \end{equation}
    By directly calculating expectation, it's easy to check that $\|Z^{*T}Z^*\|_F=O_P((n_1+n_2-2)p)$. Since $\beta\geq 1/2$  and $\frac{\sqrt{p}}{n}\to 0$, we have
    \begin{equation}
        \begin{aligned}
            \|VV^T -\hat{V}\hat{V}^T\|_F^2=O_P(\frac{p}{p^{\beta}(n_1+n_2)})=o(1).
    \end{aligned}
    \end{equation}
     Therefore, $|R_2-\frac{1}{(n_1+n_2-2)(p-r)}\|Z^*\tilde{V}\|_F^2|\xrightarrow{P}0$. But $\frac{1}{(n_1+n_2-2)(p-r)}\|Z^*\hat{V}\|_F^2\xrightarrow{P}\sigma^2$ by law of large numbers, which yields $R_2\xrightarrow{P}\sigma^2$.

    Finally, as $R_3\leq 2\sqrt{R_1}\sqrt{R_2}$ we have $R_3\xrightarrow{P}0$. It follows that $\hat{\sigma}^2_{(1)}$ is consistent.

% consistency of variance estimator 2
Next we proof the consistency of $\hat{\sigma}^2_{(2)}$.

We denote
\begin{equation}
\begin{aligned}
    S_{UU}=\frac{1}{n_1+n_2-2}\sum_{k=1}^2\sum_{i=1}^{n_k}(U_{ki}-\bar{U}_k){(U_{ki}-\bar{U}_k)}^T,
\\
    S_{ZZ}=\frac{1}{n_1+n_2-2}\sum_{k=1}^2\sum_{i=1}^{n_k}(Z_{ki}-\bar{Z}_k){(Z_{ki}-\bar{Z}_k)}^T,
\\
    S_{UZ}=\frac{1}{n_1+n_2-2}\sum_{k=1}^2\sum_{i=1}^{n_k}(U_{ki}-\bar{U}_k){(Z_{ki}-\bar{Z}_k)}^T,
\\
    S_{ZU}=\frac{1}{n_1+n_2-2}\sum_{k=1}^2\sum_{i=1}^{n_k}(Z_{ki}-\bar{Z}_k){(U_{ki}-\bar{U}_k)}^T.
\end{aligned}
\end{equation}
Then 
\begin{equation}
\begin{aligned}
    S&=\frac{1}{n_1+n_2-2}\sum_{k=1}^2\sum_{i=1}^{n_k}(X_{ki}-\bar{X}_k){(X_{ki}-\bar{X}_k)}^T\\
    &=\frac{1}{n_1+n_2-2}\sum_{k=1}^2\sum_{i=1}^{n_k}(VD(U_{ki}-\bar{U}_k)+Z_{ki}-\bar{Z}_k){(VD(U_{ki}-\bar{U}_k)+Z_{ki}-\bar{Z}_k)}^T\\
&=VDS_{UU}DV^T+VDS_{UZ}+S_{ZU}DV^T+S_{ZZ}\\
&=VD(S_{UU}DV^T+S_{UZ})+S_{ZU}DV^T+S_{ZZ}\\
    &\overset{\textrm{def}}{=}F_1+F_2+F_3.
\end{aligned}
\end{equation}
By Weyl's inequality, we have
\begin{equation}
    \lambda_{p-2r}{(F_1+F_2)}+\lambda_{2r+k}(S_{ZZ})
    \leq \lambda_{k}(S)\leq 
    \lambda_{2r+1}(F_1+F_2)+\lambda_{k-2r}(S_{ZZ})
\end{equation}
for $2r+1\leq k\leq p-2r$.
    But $rank(F_1+F_2)\leq 2r$, since $rank(F_k)\leq rank(D)=r$ for $k=1,2$. Thus the $k$th  eigenvalue of $F_1+F_2$ equals to zero, where $2r+1\leq k\leq p-2r$. In this way,
\begin{equation}
\lambda_{2r+k}(S_{ZZ})\leq \lambda_{k}(S)\leq \lambda_{k-2r}(S_{ZZ}).
\end{equation}
By above argument, we have upper bound
\begin{equation}
    \frac{1}{p-4r}\sum_{k=2r+1}^{p-2r}\lambda_k(S)\leq\frac{1}{p-4r}\sum_{k=1}^{p-4r}\lambda_{k}(S_{ZZ})\leq \frac{1}{p-4r}\mathrm{tr} S_{ZZ}
\end{equation} and lower bound
\begin{equation}
    \frac{1}{p-4r}\sum_{k=2r+1}^{p-2r}\lambda_k(S)\geq\frac{1}{p-4r}\sum_{k=4r+1}^{p}\lambda_{k}(S_{ZZ})\geq \frac{1}{p-4r}\mathrm{tr} S_{ZZ}-\frac{4r}{p-4r}\lambda_1(S_{ZZ}).
\end{equation}
Thus it suffices to  prove that
$\frac{1}{p}\mathrm{tr} S_{ZZ}\xrightarrow{P}\sigma^2$
and
$\frac{1}{p}\lambda_1(S_{ZZ})\xrightarrow{P}0$. 
We note that $(n_1+n_2-2)S_{ZZ}$ is distributed as $\textrm{Wishart}_p(n_1+n_2-2,\sigma^2 I_p)$. Hence $\frac{1}{p}\mathrm{tr}(S_{ZZ})\xrightarrow{P}\sigma^2$ by law of large numbers. And $\frac{1}{p}\lambda_1(S_{ZZ})=o_P(1)$ by Lemma~\ref{maxEigen}. Therefore the consistency of $\hat{\sigma}^2_{(2)}$ is proved.
\end{proof}


% same power with Chen's method
\begin{proof}[\textbf{Proof Of Theorem 1}]
    Our proof starts with the observation that the elements of $\mu_1-\mu_2$ is distributed as $N(0,\tau{p}^{-\frac{1}{2}}\psi)$. Hence 
    \begin{align}
        {(\mu_1-\mu_2)}^T \Sigma (\mu_1-\mu_2)=O(\|\mu_1-\mu_2\|^2)\notag
        =O_P(\tau p^{\frac{1}{2}})\notag
        =o_P(\tau\, \mathrm{tr}\Sigma^2).\notag
    \end{align}
Here the second equality holds by law of large number, the first and third equalities are due to boundedness of the eigenvalues of $\Sigma$.
    It follows that every subsequence has a further subsequence along which we have 
\begin{equation*}
{(\mu_1-\mu_2)}^T \Sigma (\mu_1-\mu_2)=o(\tau\, \mathrm{tr}\Sigma^2)
\end{equation*} 
        almost surely (a.s.). Let
\begin{equation*}
    \eta_n=\frac{T_{CQ}-\|\mu_1-\mu_2\|^2}{\sqrt{2\tau^2 \mathrm{tr}\Sigma^2}},
\end{equation*}
    then by Theorem 1 in~\cite{Chen2010A}, 
\begin{equation*}
    P(\eta_n\leq x | \mu_1,\mu_2)\to \Phi(x) \quad \textrm{a.s.}
\end{equation*}
along the further subsequence. Therefore,
\begin{equation*}
    P(\eta_n\leq x | \mu_1,\mu_2)\xrightarrow{P} \Phi(x).
\end{equation*}
We conclude from dominated convergence theorem that $\eta_n\xrightarrow{\mathcal{L}}N(0,1)$. What is left is to show that 
\begin{equation}\label{toBe0}
    \frac{T_{CQ}-T_2}{\sqrt{2\tau^2 \mathrm{tr}\Sigma^2}} \xrightarrow{P} 0.
\end{equation}
We note that
\begin{equation*}
\begin{aligned}
T_{CQ}-T_2&=
    \frac{\sum_{i\neq j}^{n_1}X_{1i}^T\hat{{V}}\hat{{V}}^T X_{1j}}{n_1(n_1-1)}+\frac{\sum_{i\neq j}^{n_2}X_{2i}^T\hat{{V}}\hat{{V}}^T X_{2j}}{n_2(n_2-1)}
-2\frac{\sum_{i=1}^{n_1}\sum_{j=1}^{n_2}X_{1i}^T\hat{{V}}\hat{{V}}^T X_{2j}}{n_1n_2}
\\
    &\overset{\textrm{def}}{=}P_1+P_2-2P_3.
\end{aligned}
\end{equation*}
And
\begin{equation*}
\begin{aligned}
    \frac{P_1}{\sqrt{2\tau^2\mathrm{tr}\Sigma^2}}=O(1)\frac{\sum_{i\neq j}^{n_1}X_{1i}^T\hat{V}\hat{V}^T X_{1j}}{n_1\sqrt{p}},
\end{aligned}
\end{equation*}
which can be further written by
\begin{equation*}
\begin{aligned}
\frac{\sum_{i\neq j}^{n_1}X_{1i}^T\hat{V}\hat{V}^T X_{1j}}{n_1\sqrt{p}}
    &=\frac{n_1(n_1-1)\bar{X}_1^T\hat{V}\hat{V}^T\bar{X}_1}{n_1\sqrt{p}}- \frac{\sum_{i=1}^{n_1}{(X_{1i}-\bar{X}_1)}^T\hat{V}\hat{V}^T(X_{1i}-\bar{X}_1)}{n_1\sqrt{p}}\\
    &\overset{\textrm{def}}{=}R_1-R_2.
\end{aligned}
\end{equation*}
Now we deal with $R_1$. Since 
    $\bar{X}_1|\mu_1\sim N(\mu_1,\frac{1}{n}\Sigma)$ and
    $\mu_1\sim N(0,\frac{\psi}{n_1\sqrt{p}}I_p)$,
we have $\bar{X}_1\sim N(0,\frac{1}{n_1}(\Sigma+\frac{1}{\sqrt{p}}\psi I_p))$. Hence we have $\hat{V}^T\bar{X}_1|S\sim N(0,\frac{1}{n}\hat{V}^T(\Sigma+\frac{1}{\sqrt{p}}\psi I_p)\hat{V})$ by the independence of $S$ and $(\mu_1,\bar{X}_1)$. Therefore,
\begin{equation*}
    \begin{aligned}
        E[\bar{X}_1^T\hat{V}\hat{V}^T\bar{X}_1]&=
    EE[\bar{X}_1^T\hat{V}\hat{V}^T\bar{X}_1|S]\\
        &=E[\frac{1}{n_1}\mathrm{tr} \hat{V}^T(\Sigma+\frac{1}{\sqrt{p}}\psi I_p)\hat{V}]\\
        &=O(\frac{1}{n_1}).
    \end{aligned}
\end{equation*}
    The last equality holds because the rank of $\hat{V}$ is at most $R$ which is fixed. It follows that $R_1\xrightarrow{P} 0$.
\begin{equation*}
\begin{aligned}
    R_2&=\frac{\mathrm{tr}[\hat{V}^T\sum_{i=1}^{n_1}(X_{1i}-\bar{X}_1){(X_{1i}-\bar{X}_1)}^T\hat{V}]}{n_1\sqrt{p}}\\
    &\leq R\frac{\lambda_1(\sum_{i=1}^{n_1}(X_{1i}-\bar{X}_1){(X_{1i}-\bar{X}_1)}^T)}{n_1\sqrt{p}}.
\end{aligned}
\end{equation*}
Lemma~\ref{maxEigen} implies that $\lambda_1(\sum_{i=1}^{n_1}(X_{1i}-\bar{X}_1){(X_{1i}-\bar{X}_1)}^T)=O_P(\max(n_1,p))$.
Therefore, by noting $p=o(n_1^2)$ we have $R_2\xrightarrow{P}0$. 
 It follows that $\frac{P_1}{\sqrt{2\tau^2\mathrm{tr}\Sigma^2}}\xrightarrow{P}0$.
 Similar arguments lead to $\frac{P_2}{\sqrt{2\tau^2\mathrm{tr}\Sigma^2}}\xrightarrow{P}0$.

\begin{align}
    \frac{P_3}{\sqrt{2\tau^2\mathrm{tr}\Sigma^2}}&=O(1)\frac{\sqrt{n_1n_2}\bar{X}_1^T\hat{V}\hat{V}^T\bar{X}_2}{\sqrt{p}}\notag
\\
    &\leq O(1)\sqrt{\frac{n_1\|\hat{V}^T\bar{X}_1\|^2}{\sqrt{p}}}\sqrt{\frac{{n_2}\|\hat{V}^T\bar{X}_2\|^2}{\sqrt{p}}},\notag
\end{align}
where the inequality is due to Cauchy inequality. By noting the relationship with  $R_1$, the right hand side converges to $0$ in probability. And the proof is completed.



\end{proof}


% proof of space estimation theorem

\begin{proof}[\textbf{Proof Of Theorem 2}]
    By~\cite{Chen2010A}'s Theorem 1, we have
    \begin{equation*}
        \frac{n_1 n_2 T_1}{\sqrt{2p}n\sigma^2}\xrightarrow{\mathcal{L}}N(0,1).
    \end{equation*}
    It suffices to prove
\begin{equation*}
\frac{n_1n_2(T_1-T_2)}{\sqrt{2p}n\sigma^2}\xrightarrow{P}0.
\end{equation*}
Since the test statistic is invariant under transformation $X_{1i}\mapsto X_{1i}+\mu$, $X_{2j}\mapsto X_{2j}+\mu$, without loss of generality, we assume $\mu_1=\mu_2=0$.
    By noting that $\hat{\tilde{V}}\hat{\tilde{V}}^T -\tilde{V}\tilde{V}^T =VV^T -\hat{V}\hat{V}^T $ we have ${n_1n_2(T_1-T_2)}/{(\sqrt{2p}n\sigma^2)}=P_1+P_2-2P_3$, where

\begin{equation*}
        P_1=\frac{n_2\sum_{i\neq j}X_{1i}^T(VV^T-\hat{V}\hat{V}^T)X_{1j}}{\sqrt{2p}(n_1+n_2)(n_1-1)\sigma^2},
\end{equation*}
\begin{equation*}
        P_2=\frac{n_1\sum_{i\neq j}X_{2i}^T(VV^T-\hat{V}\hat{V}^T)X_{2j}}{\sqrt{2p}(n_1+n_2)(n_2-1)\sigma^2},
\end{equation*}
\begin{equation*}
        P_3=\frac{\sum_{i=1}^{n_1}\sum_{j=1}^{n_2}X_{1i}^T(VV^T-\hat{V}\hat{V}^T)X_{2j}}{\sqrt{2p}(n_1+n_2)\sigma^2}.
\end{equation*}
Write
\begin{equation*}
    \begin{aligned}
        P_1=O(1)\frac{\sum_{i\neq j}X_{1i}^T(VV^T-\hat{V}\hat{V}^T)X_{1j}}{n_1\sqrt{p}}
        =O(1)(R_1-R_2),
    \end{aligned}
\end{equation*}
where
\begin{equation*}
    R_1=\frac{n_1}{\sqrt{p}}(\bar{X}_1^T VV^T\bar{X}_1-\bar{X}_1^T \hat{V}\hat{V}^T\bar{X}_1)
\end{equation*}
and 
\begin{equation*}
    R_2=\frac{1}{n_1\sqrt{p}}\sum_{i=1}^{n_1}(X_{1i}^T VV^T X_{1i}-X_{1i}^T \hat{V}\hat{V}^T X_{1i}).
\end{equation*}
To deal with $R_1$, we further decompose $\bar{X}_1^T \hat{V}\hat{V}^T\bar{X}_1$  into $3$ parts
    \begin{equation*}
    \begin{aligned} 
        \bar{X}_1^T \hat{V}\hat{V}^T\bar{X}_1=&
        \bar{X}_1^T (VV^T+\tilde{V}\tilde{V}^T) \hat{V}\hat{V}^T (VV^T+\tilde{V}\tilde{V}^T)\bar{X}_1\\
        =&\bar{X}_1^T VV^T \hat{V}\hat{V}^T VV^T \bar{X}_1
        +2\bar{X}_1^T \tilde{V}\tilde{V}^T \hat{V}\hat{V}^T VV^T \bar{X}_1\\
        &+\bar{X}_1^T VV^T \hat{V}\hat{V}^T VV^T \bar{X}_1.\\
    \end{aligned}
    \end{equation*} 
    The above decomposition is a technique we will use many times. $R_1$ can thus be written by $R_1=Q_1-2Q_2-Q_3$, where
    \begin{equation*}
        Q_1=\frac{n_1}{\sqrt{p}}\bar{X}_1^T V(I_r-V^T\hat{V}\hat{V}^T V)V^T \bar{X}_1,
    \end{equation*}
    \begin{equation*}
        Q_2=\frac{n_1}{\sqrt{p}}\bar{X}_1^T \tilde{V}\tilde{V}^T\hat{V}\hat{V}^T VV^T \bar{X}_1,
    \end{equation*}
    \begin{equation*}
        Q_3=\frac{n_1}{\sqrt{p}}\bar{X}_1^T \tilde{V}\tilde{V}^T\hat{V}\hat{V}^T \tilde{V}\tilde{V}^T \bar{X}_1.
    \end{equation*}
    It's clear that $V^T \bar{X}_1 \sim N_r(0,\frac{1}{n}(\Lambda+\sigma^2 I_p))$, $\tilde{V}^T \bar{X}_1 \sim N_{p-r}(0,\frac{\sigma^2 }{n}I_p)$ and $S$ are mutually independent.  $\|V^T\bar{X}_1\|^2=O_P(\frac{p^{\beta}}{n_1})$.  $I_r-V^T \hat{V}\hat{V}^T V$ is a positive semi-difinite matrix which only relies  on $S$. Combining these observations, we have
    

    \begin{equation}\label{myQ1}
        \begin{aligned}
            |Q_1|&\leq \frac{n_1}{\sqrt{p}}\|I_r-V^T \hat{V}\hat{V}^T V\| \|V^T\bar{X}_1\|^2\\
            &\leq \frac{n_1}{\sqrt{p}}\mathrm{tr}(I_r-V^T \hat{V}\hat{V}^T V) \|V^T\bar{X}_1\|^2\\
            &=\frac{n_1}{2\sqrt{p}}\|VV^T -\hat{V}\hat{V}^T\|^2_F \|V^T\bar{X}_1\|^2\\
            &=O_P(1)\frac{n_1}{\sqrt{p}}\frac{p}{p^{\beta}n_1}\frac{p^\beta}{n_1}\xrightarrow{P}0.
        \end{aligned}
    \end{equation}

We next prove $Q_2\xrightarrow{L^2}0$.
\begin{align}
        E(Q_2^2)&=\frac{n_1^2}{p}E(\bar{X}_1^T \tilde{V} \tilde{V}^T \hat{V}\hat{V}^T VV^T \bar{X}_1\bar{X}_1^T VV^T \hat{V}\hat{V}^T \tilde{V}\tilde{V}^T \bar{X}_1)\label{myQ2}\\
        &=\frac{n_1^2}{p}E(\bar{X}_1^T \tilde{V} \tilde{V}^T \hat{V}\hat{V}^T V\frac{1}{n_1}(\Lambda+\sigma^2I_r)V^T \hat{V}\hat{V}^T \tilde{V}\tilde{V}^T \bar{X}_1)\notag\\
        &=O(1)\frac{n_1p^{\beta}}{p}\mathrm{E}(\bar{X}_1^T \tilde{V} \tilde{V}^T \hat{V}\hat{V}^T VV^T \hat{V}\hat{V}^T \tilde{V}\tilde{V}^T \bar{X}_1)\notag\\
        &=O(1)\frac{n_1p^{\beta}}{p}\mathrm{E}\, \mathrm{tr}(\tilde{V}^T \hat{V}\hat{V}^T VV^T \hat{V}\hat{V}^T \tilde{V}\tilde{V}^T \bar{X}_1\bar{X}_1^T \tilde{V} )\notag\\
        &=O(1)\frac{n_1p^{\beta}}{p}\frac{\sigma^2}{n_1}\mathrm{E}\, \mathrm{tr}(\tilde{V}^T \hat{V}\hat{V}^T VV^T \hat{V}\hat{V}^T \tilde{V})\notag\\
        &\leq O(1)\frac{p^{\beta}}{p}\mathrm{E}\, \mathrm{tr}(\tilde{V}^T \hat{V}\hat{V}^T I_p \hat{V}\hat{V}^T \tilde{V})\notag\\
        &= O(1)\frac{p^{\beta}}{p}\mathrm{E}\, \mathrm{tr}(\tilde{V}^T \hat{V}\hat{V}^T \tilde{V})\notag\\
        &= O(1)\frac{p^{\beta}}{p}\mathrm{E}\, \mathrm{tr}( \hat{V}\hat{V}^T (I_p-VV^T ))\notag\\
        &= O(1)\frac{p^{\beta}}{p}\frac{1}{2}\mathrm{E}\|VV^T -\hat{V}\hat{V}^T \|_F^2\notag\\
        &= O(\frac{1}{n_1})\to 0.\notag
\end{align}
Similarly,
\begin{align}
        E(Q_3)&=\frac{n_1}{\sqrt{p}}\frac{\sigma^2}{n_1}E\, \mathrm{tr}(\tilde{V}^T \hat{V}\hat{V}^T \tilde{V})\label{myQ3}\\
        &\leq \frac{r\sigma^2}{\sqrt{p}}\to 0.   \notag
\end{align}
By~\eqref{myQ1},~\eqref{myQ2} and~\eqref{myQ3}, $R_1\xrightarrow{P}0$. Next we deal with $R_2$. Write $R_2=W_1-2W_2-W_3$, where

    \begin{equation*}
        W_1=\frac{1}{n_1\sqrt{p}}\sum_{i=1}^{n_1}X_{1i}^T V(I_r-V^T\hat{V}\hat{V}^T V)V^T X_{1i},
    \end{equation*}
    \begin{equation*}
        W_2=\frac{1}{n_1\sqrt{p}}\sum_{i=1}^{n_1}X_{1i}^T \tilde{V}\tilde{V}^T\hat{V}\hat{V}^T VV^T X_{1i},
    \end{equation*}
    \begin{equation*}
        W_3=\frac{1}{n_1\sqrt{p}}\sum_{i=1}^{n_1}X_{1i}^T \tilde{V}\tilde{V}^T\hat{V}\hat{V}^T \tilde{V}\tilde{V}^T X_{1i}.
    \end{equation*}
It is seen that independence property does not holds anymore compared with the case we deal with $R_1$. Nevertheless, the form is of sum, which makes it possible to apply law of large numbers.

We note that
    \begin{equation}\label{803W1}
        \begin{aligned}
            W_1&=\frac{1}{n_1\sqrt{p}}\mathrm{tr} ((I_r-V^T\hat{V}\hat{V}^T V)\sum_{i=1}^{n_1}V^T X_{1i}X_{1i}^T V).
        \end{aligned}
    \end{equation}
By law of large numbers, 
\begin{equation*}
\frac{1}{n_1}{(\Lambda+\sigma^2 I_{r})}^{-\frac{1}{2}}\sum_{i=1}^{n_1}V^T X_{1i}X_{1i}^T V {(\Lambda+\sigma^2 I_{r})}^{-\frac{1}{2}}\xrightarrow{P}I_{r}.
\end{equation*}
Hence $\lambda_1(\sum_{i=1}^{n_1}V^T X_{1i}X_{1i}^T V )=O_P(n_1 p^{\beta})$. Substituting it into~\eqref{803W1} yields

\begin{equation}\label{myW1}
    \begin{aligned}
        W_1&\leq O_P(1)\frac{p^{\beta}}{\sqrt{p}}\mathrm{tr}(I_r-V^T \hat{V}\hat{V}^T V)\\
        &=O_P(1)\frac{p^{\beta}}{\sqrt{p}}\|VV^T -\hat{V}\hat{V}^T\|^2_F\\
        &=O_P(1)\frac{p^\beta}{\sqrt{p}}\frac{p}{p^\beta n_1}\\
        &=O_P(\frac{\sqrt{p}}{n_1})\xrightarrow{P}0.
    \end{aligned}
\end{equation}
 Next we deal with $W_2$. By cauchy inequality, we have
\begin{equation*}
    \begin{aligned}
        W_2 &=\frac{1}{n_2 \sqrt{p}}\mathrm{tr} \tilde{V}^T \hat{V}\hat{V}^T V (\sum_{i=1}^{n_1}V^T X_{1i}X_{1i}^T \tilde{V})\\
        &\leq \frac{1}{n_2 \sqrt{p}}\sqrt{\mathrm{tr}(\tilde{V}^T \hat{V}\hat{V}^T VV^T \hat{V}\hat{V}^T \tilde{V})}\sqrt{\mathrm{tr}(VZZ^T \tilde{V}\tilde{V}^T ZZ^T V^T)},
    \end{aligned}
\end{equation*}
where $Z=(X_{11},\ldots,X_{1n_1})$. Note that
\begin{equation*}
    \begin{aligned}
        \mathrm{tr}(\tilde{V}^T \hat{V}\hat{V}^T VV^T \hat{V}\hat{V}^T \tilde{V})&\leq \mathrm{tr} (\tilde{V}^T\hat{V}\hat{V}^T\tilde{V})=\frac{1}{2}\|VV^T-\hat{V}\hat{V}^T\|^2_F
    \end{aligned}
\end{equation*}
and
\begin{equation*}
    \begin{aligned}
        \mathrm{tr}(VZZ^T \tilde{V}\tilde{V}^T ZZ^T V^T)&\leq
        \lambda_1 (Z^T\tilde{V}\tilde{V}^T Z)\mathrm{tr}(VZZ^T V^T).
    \end{aligned}
\end{equation*}
$Z^T\tilde{V}\tilde{V}^T Z$ is distributed as $\textrm{Wishart}_{n}(p-r,\sigma^2 I_{n_1})$, hence $\lambda_1 (Z^T\tilde{V}\tilde{V}^T Z)=O_P(\max(n_1,p))$ by Lemma~\ref{maxEigen}. Again by law of large numbers $\mathrm{tr}(VZZ^T V^T)=O(p^{\beta} n_1)$. Combining the argument above yields
\begin{equation}\label{myW2}
    \begin{aligned}
        W_2&=O_P(1)\frac{1}{n_1\sqrt{p}}\sqrt{\frac{p}{p^{\beta}n_1}}\sqrt{\max(n_1,p)p^{\beta}n_1}\\
        &=O_P(\frac{\sqrt{\max(n_1,p)}}{n_1})\xrightarrow{P}0.
    \end{aligned}
\end{equation}

To deal with $W_3$, note that $\tilde{V}^T\sum_{i=1}^{n_1}X_{1i}X_{1i}^T\tilde{V}$ is of distribution $\textrm{Wishart}_{p-r}(n,\sigma^2 I_{p-r})$. By Lemma~\ref{maxEigen}, $\lambda_1(\tilde{V}^T\sum_{i=1}^{n_1}X_{1i}X_{1i}^T\tilde{V})=O_P(\max(p,n_1))$. 
Hence

\begin{equation}\label{myW3}
    \begin{aligned}
        W_3&=\frac{1}{n_1\sqrt{p}}\mathrm{tr}(\hat{V}^T\tilde{V}\tilde{V}^T\sum_{i=1}^{n_1}X_{1i}X_{1i}^T\tilde{V}\tilde{V}^T \hat{V})\\
        &\leq \frac{O_{P}(\max(n_1,p))}{n_1\sqrt{p}}\mathrm{tr} \hat{V}^T\tilde{V}\tilde{V}^T\hat{V}\xrightarrow{P}0,
    \end{aligned}
\end{equation}
due to $\mathrm{tr} \hat{V}^T\tilde{V}\tilde{V}^T\hat{V}=O_P(1)$. Combining~\eqref{myW1},~\eqref{myW2} and~\eqref{myW3} gives $R_2\xrightarrow{P}0$. So far, it has been proved that $P_1\xrightarrow{P}0$. By symmetry, $P_2\xrightarrow{P}0$. 

We proceed to deal with $P_3$. Note that $P_3=O(1)(M_1-M_2-M_3-M_4)$, where
    \begin{equation*}
        M_1=\frac{n_1}{\sqrt{p}}\bar{X}_1^T V(I_r-V^T\hat{V}\hat{V}^T V)V^T \bar{X}_2,
    \end{equation*}
    \begin{equation*}
        M_2=\frac{n_1}{\sqrt{p}}\bar{X}_1^T \tilde{V}\tilde{V}^T\hat{V}\hat{V}^T VV^T \bar{X}_2,
    \end{equation*}
    \begin{equation*}
        M_3=\frac{n_1}{\sqrt{p}}\bar{X}_1^T {V}{V}^T\hat{V}\hat{V}^T \tilde{V}\tilde{V}^T \bar{X}_2,
    \end{equation*}
    \begin{equation*}
        M_4=\frac{n_1}{\sqrt{p}}\bar{X}_1^T \tilde{V}\tilde{V}^T\hat{V}\hat{V}^T \tilde{V}\tilde{V}^T \bar{X}_2.
    \end{equation*}
Since $\|V^T\bar{X}_i\|^2=O_P(\frac{p^{\beta}}{n_1})$ for $i=1,2$, we have
    \begin{equation}\label{myM1}
        \begin{aligned}
            |M_1|&\leq \frac{n_1}{\sqrt{p}}\|I_r-V^T \hat{V}\hat{V}^T V\| \|V^T\bar{X}_1\|\|V^T\bar{X}_2\|\\
            &\leq \frac{n_1}{2\sqrt{p}}\|VV^T -\hat{V}\hat{V}^T\|^2_F \|V^T\bar{X}_1\|\|V^T\bar{X}_2\|\\
            &=O_P(1)\frac{n_1}{\sqrt{p}}\frac{p}{p^{\beta}n_1}\frac{p^\beta}{n_1}\xrightarrow{P}0.
        \end{aligned}
    \end{equation}
The joint distribution of ($V^T\bar{X}_2$, $\tilde{V}^T\bar{X}_1$,$S$) is identity to that of ($V^T\bar{X}_1$, $\tilde{V}^T\bar{X}_1$,$S$). Therefore, $M_2$ has the same distribution with $Q_2$ and converges to $0$ in probability. The same reasoning yields $M_3\xrightarrow{P}0$.

    Applying cauchy inequality gives
    \begin{equation*}
        \begin{aligned}
            M_4\leq \sqrt{\frac{n_1}{\sqrt{p}}\bar{X}_1^T\tilde{V}\tilde{V}^T \hat{V}\hat{V}^T \tilde{V}\tilde{V}^T\bar{X}_1}
            \sqrt{\frac{n_1}{\sqrt{p}}\bar{X}_2^T\tilde{V}\tilde{V}^T \hat{V}\hat{V}^T \tilde{V}\tilde{V}^T\bar{X}_2}.
        \end{aligned}
    \end{equation*}
    Then $M_4\xrightarrow{P} 0$ by the same reason as $Q_3\xrightarrow{P} 0$, and consequently $P_3\xrightarrow{P}0$. This finishes the proof.

\end{proof}

\begin{proof}[\textbf{Proof Of Theorem 3}]
    By~\cite{Chen2010A}'s Theorem 1, we have
    \begin{equation*}
        \frac{n_1 n_2 (T_1-\|\tilde{V}(\mu_1-\mu_2)\|^2)}{\sqrt{2p}(n_1+n_2)\sigma^2}\xrightarrow{\mathcal{L}}N(0,1).
    \end{equation*}
    It suffices to prove
\begin{equation*}
\frac{n_1n_2(T_1-T_2)}{\sqrt{2p}(n_1+n_2)\sigma^2}\xrightarrow{P}0.
\end{equation*}
Write $T_1=T_1^{(1)}+T_1^{(2)}$, where
\begin{equation*}
\begin{aligned}
    T_1^{(1)}=&\frac{\sum_{i\neq j}^{n_1}{(X_{1i}-\mu_1)}^T\tilde{V}\tilde{V}^T(X_{1j}-\mu_1)}{n_1(n_1-1)}+\frac{\sum_{i\neq j}^{n_2}{(X_{2i}-\mu_2)}^T\tilde{V}\tilde{V}^T(X_{2j}-\mu_2)}{n_2(n_2-1)}
\\
    &-2\frac{\sum_{i=1}^{n_1}\sum_{j=1}^{n_2}{(X_{1i}-\mu_1)}^T\tilde{V}\tilde{V}^T(X_{2j}-\mu_2)}{n_1n_2}
\end{aligned}
\end{equation*}
and
\begin{equation*}
\begin{aligned}
    T_1^{(2)}=&2{(\bar{X_{1}}-\mu_1)}^T\tilde{V}\tilde{V}^T(\mu_1-\mu_2)+{(\bar{X_{2}}-\mu_2)}^T\tilde{V}\tilde{V}^T(\mu_2-\mu_1)
\\
&+\|\tilde{V}(\mu_1-\mu_2)\|^2.
\end{aligned}
\end{equation*}
    Similarly, $T_2=T_2^{(1)}+T_2^{(2)}$. By Theorem~\ref{myPanpan}, it holds that
    \begin{equation}\label{yintianHaha}
        \begin{aligned}
            \frac{n_1n_2(T_2^{(1)}-T_1^{(1)})}{\sqrt{2p}(n_1+n_2)\sigma^2}\xrightarrow{P}0.
        \end{aligned}
    \end{equation}
We are left with the task of dealing with $T_{1}^{(2)}$ and $T_{2}^{(2)}$. Note that
\begin{align}
    \frac{n_1n_2(T_2^{(2)}-T_1^{(2)})}{\sqrt{2p}(n_1+n_2)\sigma^2}=&O(1)\frac{n_1}{\sqrt{p}}{(\bar{X_{1}}-\mu_1)}^T(VV^T -\hat{V}\hat{V}^T)(\mu_1-\mu_2)+
\label{yumenHaha1}\\
    &O(1)\frac{n_2}{\sqrt{p}}{(\bar{X_{2}}-\mu_2)}^T(VV^T-\hat{V}\hat{V}^T)(\mu_2-\mu_1)+
\label{yumenHaha2}\\
    &O(1)\frac{n_1+n_2}{\sqrt{p}}{(\mu_1-\mu_2)}^T(VV^T-\hat{V}\hat{V}^T)(\mu_1-\mu_2).
\label{yumenHaha3}
\end{align}

    We proceed to deal with the~\eqref{yumenHaha1}.
\begin{align}
    &E|\frac{n_1}{\sqrt{p}}{(\bar{X_{1}}-\mu_1)}^T(VV^T-\hat{V}\hat{V}^T)(\mu_1-\mu_2)|^2
\label{theStartHaha}
\\
    &=\frac{n_1^2}{p}E{(\mu_1-\mu_2)}^T(VV^T-\hat{V}\hat{V}^T)\frac{\Sigma}{n_1}(VV^T-\hat{V}\hat{V}^T)(\mu_1-\mu_2)
    \notag\\
    &=O(n_1p^{\beta-1})E{(\mu_1-\mu_2)}^T{(VV^T-\hat{V}\hat{V}^T)}^2(\mu_1-\mu_2)
    \notag\\
&\leq
O(n_1p^{\beta-1})\|\mu_1-\mu_2\|^2E\|VV^T-\hat{V}\hat{V}^T\|^2_F
    \notag\\
    &=O(n_1p^{\beta-1})\frac{p}{p^{\beta}n_1}\|\mu_1-\mu_2\|^2
    \notag\\
    &=O(\|\mu_1-\mu_2\|^2).
    \notag
\end{align}
Combining with conditions $\frac{(n_1+n_2)}{\sqrt{p}}\|\mu_1-\mu_2\|^2=O(1)$ and $\frac{\sqrt{p}}{n_1+n_2}\to 0$, it follows that~\eqref{theStartHaha} converges to $0$ in probability.
    By symmetry,~\eqref{yumenHaha2} also converges to $0$ in probability. It remains to deal with~\eqref{yumenHaha3}. But
\begin{align}
    &E|\frac{n_1+n_2}{\sqrt{p}}{(\mu_1-\mu_2)}^T(VV^T-\hat{V}\hat{V}^T )(\mu_1-\mu_2)|
\label{mmmHaha}
\\
&\leq
\frac{n_1+n_2}{\sqrt{p}}\|\mu_1-\mu_2\|^2 E\|VV^T-\hat{V}\hat{V}^T\|
\notag
\\
    &\leq    \frac{n_1+n_2}{\sqrt{p}}\|\mu_1-\mu_2\|^2 \sqrt{E\|VV^T-\hat{V}\hat{V}^T\|^2}
\notag
\\
    &\leq O(1)\sqrt{E\|VV^T-\hat{V}\hat{V}^T\|^2_F}.
\notag
\end{align}
Since $E\|VV^T-\hat{V}\hat{V}^T\|^2_F=O(\frac{p}{p^{\beta}n_1})$ and $\beta\geq 1/2$,~\eqref{mmmHaha} converges to $0$.

It follows that
\begin{equation*}
\begin{aligned}
\frac{n_1n_2(T_2^{(2)}-T_1^{(2)})}{\sqrt{2p}(n_1+n_2)\sigma^2}\xrightarrow{P}0.
\end{aligned}
\end{equation*}
Together with~\eqref{yintianHaha}, the theorem follows.
\end{proof}


\begin{proof}[\textbf{Proof Of Theorem 4}]
    The theorem follows by Theorem~\ref{oracleTheorem}, Theorem~\ref{spaceEstimation} and Theorem~\ref{varianceEstimation}.
\end{proof}


\begin{proof}[\textbf{Proof Of Theorem 4}]
    The proof is based on Theorem~\ref{myPanpan} and procedure runs almost the same. The notation is parallel to Theorem~\ref{myPanpan}. Denote by $O_{m\times n}$ the $m\times n$ matrix with all elements equal to $0$. First we consider term
\begin{equation*}
    R_1'=\frac{n_1}{\sqrt{p}}(\bar{X}_1^T VV^T\bar{X}_1-\bar{X}_1^T \hat{V}\hat{V}^T\bar{X}_1),
\end{equation*}
and
\begin{equation*}
    R_2'=\frac{1}{n_1\sqrt{p}}\sum_{i=1}^{n_1}(X_{1i}^T VV^T X_{1i}-X_{1i}^T \hat{V}\hat{V}^T X_{1i}).
\end{equation*}
We note that $VV^T\geq V_1 V_1^T$. Define $V_{2\ominus 1}V_{2\ominus 1}^T=VV^T-V_1 V_1^T$. From the fact that $rank(V)\leq r_1+r_2$ and $V_{2\ominus 1}^T V_1=O_{r_2\times r_1}$, it can easily deduced that $\frac{n_1}{\sqrt{p}}\bar{X}_1^T VV^T\bar{X}_1-\frac{n_1}{\sqrt{p}}\bar{X}_1^T V_1V_1^T\bar{X}_1\xrightarrow{P}0$. Hence to proof $R_1'\xrightarrow{P}0$, we only need to consider
\begin{equation*}
\frac{n_1}{\sqrt{p}}(\bar{X}_1^T V_1V_1^T \bar{X}_1-\bar{X}_1^T \hat{V}\hat{V}^T \bar{X}_1),
\end{equation*}
which can be further decomposed into three parts

    \begin{equation*}
        Q_1'=\frac{n_1}{\sqrt{p}}\bar{X}_1^T V_1(I_{r_1}-V_1^T\hat{V}\hat{V}^T V_1)V_1^T \bar{X}_1,
    \end{equation*}
    \begin{equation*}
        Q_2'=\frac{n_1}{\sqrt{p}}\bar{X}_1^T \tilde{V}_1\tilde{V}_1^T\hat{V}\hat{V}^T V_1V_1^T \bar{X}_1,
    \end{equation*}
    \begin{equation*}
        Q_3'=\frac{n_1}{\sqrt{p}}\bar{X}_1^T \tilde{V}_1\tilde{V}_1^T\hat{V}\hat{V}^T \tilde{V}_1\tilde{V}_1^T \bar{X}_1.
    \end{equation*}
We note that $Q_1'\leq n_1\bar{X}_1^T V_1(I_{r_1}-V_1^T\hat{V}_1\hat{V}_1^T V_1)V_1^T \bar{X}_1/\sqrt{p}$ which converges to $0$ by Theorem~\ref{myPanpan}.  $Q_2'$ can be written as

    \begin{equation*}
    \begin{aligned}
        Q_2'=\frac{n_1}{\sqrt{p}}\bar{X}_1^T \tilde{V}_1\tilde{V}_1^T\hat{V}_1\hat{V}_1^T V_1V_1^T \bar{X}_1
        +\frac{n_1}{\sqrt{p}}\bar{X}_1^T \tilde{V}_1\tilde{V}_1^T\hat{V}_{2\ominus 1}\hat{V}_{2\ominus 1}^T V_1V_1^T \bar{X}_1.
        \end{aligned}
    \end{equation*}
    The first term converges to $0$ by Theorem~\ref{myPanpan}.  For the second term we have
    \begin{equation*}
    \begin{aligned}
&\frac{n_1}{\sqrt{p}}\bar{X}_1^T \tilde{V}_1\tilde{V}_1^T\hat{V}_{2\ominus 1}\hat{V}_{2\ominus 1}^T V_1V_1^T \bar{X}_1\\
&\leq
\sqrt{\frac{n_1}{\sqrt{p}}\bar{X}_1^T \tilde{V}_1\tilde{V}_1^T\hat{V}_{2\ominus 1}\hat{V}_{2\ominus 1}^T \tilde{V}_1\tilde{V}_1^T \bar{X}_1}\sqrt{\frac{n_1}{\sqrt{p}}\bar{X}_1^T {V}_1{V}_1^T\hat{V}_{2\ominus 1}\hat{V}_{2\ominus 1}^T V_1V_1^T \bar{X}_1}.
    \end{aligned}
    \end{equation*}
The first term converges to $0$ because $\hat{V}_{2\ominus 1}^T \tilde{V}_1\tilde{V}_1^T \bar{X}_1|(S_1,S_2)\sim N(0,\frac{\sigma_1^2}{n_1}\hat{V}_{2\ominus 1}^T \tilde{V}_1\tilde{V}_1^T\hat{V}_{2\ominus 1})$ whose conditional variance is dominated by $\frac{\sigma^2_1}{n_1}$. And for the second term, we have
\begin{equation*}
\begin{aligned}
\frac{n_1}{\sqrt{p}}\bar{X}_1^T {V}_1{V}_1^T\hat{V}_{2\ominus 1}\hat{V}_{2\ominus 1}^T V_1V_1^T \bar{X}_1
&\leq
\frac{n_1}{\sqrt{p}}\bar{X}_1^T {V}_1{V}_1^T\hat{\tilde{V}}_{1}\hat{\tilde{V}}_{1}^T V_1V_1^T \bar{X}_1\\
&\leq
\frac{n_1}{\sqrt{p}}\|{V}_1^T\hat{\tilde{V}}_{1}\hat{\tilde{V}}_{1}^T V_1\|\|V_1^T \bar{X}_1\|^2\\
&\leq O(1)\frac{n_1}{\sqrt{p}}\|V_1 V_1^T -\hat{V}_1\hat{V}_1^T\|^2_F \frac{p^{\beta}}{n_1}\\
&=O(\frac{\sqrt{p}}{n_1})\xrightarrow{P}0.
\end{aligned}
\end{equation*}
$Q_3'=
\frac{n_1}{\sqrt{p}}\bar{X}_1^T \tilde{V}_1\tilde{V}_1^T\hat{V}_{2\ominus 1}\hat{V}_{2\ominus 1}^T \tilde{V}_1\tilde{V}_1^T \bar{X}_1+
\frac{n_1}{\sqrt{p}}\bar{X}_1^T \tilde{V}_1\tilde{V}_1^T\hat{V}_{1}\hat{V}_{1}^T \tilde{V}_1\tilde{V}_1^T \bar{X}_1$
    converges to $0$ in probability by combining Theorem~\ref{myPanpan} and the argument above. Hence $R_1'\xrightarrow{P}0$.

To prove $R_2'\xrightarrow{P}0$, we note that
 $\frac{1}{n_1\sqrt{p}}\sum_{i=1}^{n_1}X_{1i}^T VV^T X_{1i}-
\frac{1}{n_1\sqrt{p}}\sum_{i=1}^{n_1}X_{1i}^T V_1V_1^T X_{1i}\xrightarrow{P}0$.
    Hence it suffices to consider the following three parts
    \begin{equation*}
        W_1'=\frac{1}{n_1\sqrt{p}}\sum_{i=1}^{n_1}X_{1i}^T V_1(I_{r_1}-V_1^T\hat{V}\hat{V}^T V_1)V_1^T X_{1i},
    \end{equation*}
    \begin{equation*}
        W_2'=\frac{1}{n_1\sqrt{p}}\sum_{i=1}^{n_1}X_{1i}^T \tilde{V}_1\tilde{V}_1^T\hat{V}\hat{V}^T V_1V_1^T X_{1i}
    \end{equation*}
    and
    \begin{equation*}
        W_3'=\frac{1}{n_1\sqrt{p}}\sum_{i=1}^{n_1}X_{1i}^T \tilde{V}_1\tilde{V}_1^T\hat{V}\hat{V}^T \tilde{V}_1\tilde{V}_1^T X_{1i}.
    \end{equation*}
$W_1'\xrightarrow{P}0$ by $\hat{V}\hat{V}^T\leq \hat{V}_1\hat{V}_1^T$ and Theorem~\ref{myPanpan}.
\begin{equation*}
\begin{aligned}
        W_2'=\frac{1}{n_1\sqrt{p}}\sum_{i=1}^{n_1}X_{1i}^T \tilde{V}_1\tilde{V}_1^T\hat{V}_1\hat{V}_1^T V_1V_1^T X_{1i}+\frac{1}{n_1\sqrt{p}}\sum_{i=1}^{n_1}X_{1i}^T \tilde{V}_1\tilde{V}_1^T\hat{V}_{2\ominus 1}\hat{V}_{2\ominus 1}^T V_1V_1^T X_{1i}.
\end{aligned}
\end{equation*}
The first term converges to $0$ by Theorem~\ref{myPanpan}. As for the second term, we can use the similar technique as we deal with $W_2$ and the fact $\hat{V}_{2\ominus 1}\hat{V}_{2\ominus 1}^T\leq \hat{\tilde{V}}\hat{\tilde{V}}^T$.
The proof of $W_3'\xrightarrow{P}0$ runs the same as the case of $W_3$.

Finally it suffices to proof following four terms converges to $0$ in probability

    \begin{equation*}
        M_1'=\frac{n_1}{\sqrt{p}}\bar{X}_1^T V(I_{r_1+r_2}-V^T\hat{V}\hat{V}^T V)V^T \bar{X}_2,
    \end{equation*}
    \begin{equation*}
        M_2'=\frac{n_1}{\sqrt{p}}\bar{X}_1^T \tilde{V}\tilde{V}^T\hat{V}\hat{V}^T VV^T \bar{X}_2,
    \end{equation*}
    \begin{equation*}
        M_3'=\frac{n_1}{\sqrt{p}}\bar{X}_1^T {V}{V}^T\hat{V}\hat{V}^T \tilde{V}\tilde{V}^T \bar{X}_2,
    \end{equation*}
    and
    \begin{equation*}
        M_4'=\frac{n_1}{\sqrt{p}}\bar{X}_1^T \tilde{V}\tilde{V}^T\hat{V}\hat{V}^T \tilde{V}\tilde{V}^T \bar{X}_2.
    \end{equation*}

We note that 
\begin{equation*}
M_1'\leq\sqrt{\frac{n_1}{\sqrt{p}}\bar{X}_1^T V(I_{r_1+r_2}-V^T\hat{V}\hat{V}^T V)V^T \bar{X}_1}\sqrt{\frac{n_1}{\sqrt{p}}\bar{X}_1^T V(I_{r_1+r_2}-V^T\hat{V}\hat{V}^T V)V^T \bar{X}_2}.
\end{equation*}
Hence to prove $M_1'\xrightarrow{P} 0$, it suffices to prove $\frac{n_1}{\sqrt{p}}\bar{X}_1^T V(I_{r_1+r_2}-V^T\hat{V}\hat{V}^T V)V^T \bar{X}_1\xrightarrow{P}0$. 
We note that
\begin{equation*}
\begin{aligned}
&\frac{n_1}{\sqrt{p}}\bar{X}_1^T V(I_{r_1+r_2}-V^T\hat{V}\hat{V}^T V)V^T \bar{X}_1\\
&=(\frac{n_1}{\sqrt{p}}\bar{X}_1^T V_1V_1^T \bar{X}_1-\frac{n_1}{\sqrt{p}}\bar{X}_1^T V_1V_1^T \hat{V}\hat{V}^T V_1 V_1^T \bar{X}_1)
\\
&+\frac{n_1}{\sqrt{p}}\bar{X}_1^T V_{2\ominus 1}V_{2\ominus 1}^T \bar{X}_1-
2\frac{n_1}{\sqrt{p}}\bar{X}_1^T V_{2\ominus 1}V_{2\ominus 1}^T \hat{V}\hat{V}^T V_1 V_1^T \bar{X}_1
\\
&-\frac{n_1}{\sqrt{p}}\bar{X}_1^T V_{2\ominus 1}V_{2\ominus 1}^T \hat{V}\hat{V}^T V_{2\ominus 1}V_{2\ominus 1}^T\bar{X}_1
\\
&=L_1+L_2-2L_3-L_4.
\end{aligned}
\end{equation*}
Check that $L_1=Q_1'$. It's also clear that $L_2\xrightarrow{P}0$. To deal with $L_3$, we note that
\begin{equation}\label{xiaojiqiao1}
\begin{aligned}
    &|\frac{n_1}{\sqrt{p}}\bar{X}_1^T V_{2\ominus 1}V_{2\ominus 1}^T \hat{V}_1\hat{V}_1^T V_1 V_1^T \bar{X}_1|
\\
    &\leq \frac{n_1}{\sqrt{p}}\|V_{2\ominus 1}^T\bar{X}_1\|\|V^T_{2\ominus 1}\hat{V}_1\|_F\|\hat{V}_1^T V_1{V}_1^T\bar{X}_1\|\\
&\leq
\frac{n_1}{\sqrt{p}}O(\frac{1}{\sqrt{n_1}})\sqrt{tr(\hat{V}_1^T V_{2\ominus 1}V_{2\ominus 1}^T\hat{V}_1)}O(\frac{\sqrt{p^{\beta}}}{\sqrt{n_1}}).
\end{aligned}
\end{equation}
But 
\begin{equation*}
tr(\hat{V}_1^T V_{2\ominus 1}V_{2\ominus 1}^T\hat{V}_1)\leq tr(\hat{V}_1^T \tilde{V}_1\tilde{V}_1^T\hat{V}_1)=\|V_1V_1^T- \hat{V}_1\hat{V}_1^T\|_F^2=O_P(\frac{p}{p^{\beta}n_1}).
\end{equation*}
And
\begin{equation}\label{xiaojiqiao2}
\begin{aligned}
|\frac{n_1}{\sqrt{p}}\bar{X}_1^T V_{2\ominus 1}V_{2\ominus 1}^T \hat{V}_{2\ominus 1}\hat{V}_{2\ominus 1}^T V_1 V_1^T \bar{X}_1|
&\leq \frac{n_1}{\sqrt{p}}\|V^T_{2\ominus 1}\hat{V}_{2\ominus 1}V_{2\ominus 1}^T\bar{X}_1\|\|\hat{V}_{2\ominus 1}^T V_1\|_F \|{V}_1^T\bar{X}_1\|\\
&\leq
\frac{n_1}{\sqrt{p}}O(\frac{1}{\sqrt{n_1}})\sqrt{tr(\hat{V}_{2\ominus 1}^T V_{1}V_{1}^T\hat{V}_{2\ominus 1})}O(\frac{\sqrt{p^{\beta}}}{\sqrt{n_1}}).
\end{aligned}
\end{equation}
But
\begin{equation*}
tr(\hat{V}_{2\ominus 1}^T V_{1}V_{1}^T\hat{V}_{2\ominus 1})\leq tr(\hat{\tilde{V}}_1^T {V}_1{V}_1^T\hat{\tilde{V}}_1)=\|V_1V_1^T- \hat{V}_1\hat{V}_1^T\|_F^2=O_P(\frac{p}{p^{\beta}n_1}).
\end{equation*}
Therefore $L_3\xrightarrow{P}0$. And $L_4\xrightarrow{P}0$ for trival reason.
 $M_2'$ can be similarly treated by technique~\eqref{xiaojiqiao1} and~\eqref{xiaojiqiao2}. Since $\sqrt{n_1}\hat{V}^T\tilde{V}\tilde{V}^T\bar{X}_2$ is bounded in probability, we have $M_4'\xrightarrow{P} 0$, which completes the proof.

    \end{proof}
\begin{proof}[\textbf{Proof Of Theorem 5}]
    The theorem follows by the same method as Theorem~\ref{spaceEstimation}.
\end{proof}



\section*{Acknowledgements}
This work was supported by the National Natural Science Foundation of China under Grant No. 11471035, 11471030.


\section*{References}

\bibliography{mybibfile}

\end{document}
