
% consistency of variance estimator 1
\begin{proof}[\textbf{Proof Of Proposition 2}]
    By a standard orthogonal transformation, $\hat{\sigma}_{(1)}^2$ has the same law with respect to $\frac{1}{(n_1+n_2-2)(p-r)}\mathrm{tr}\sum_{k=1}^2\sum_{i=1}^{n_i-1}\hat{\tilde{V}}^T Y_{ki}Y_{ki}^T\hat{\tilde{V}}$,
    where $Y_{ki}=VDU^{*}_{ki}+Z_{ki}^*$. Here
    $U_{ki}^{*}$'s are i.i.d.\ random vectors with $r$ dimensional standard normal distribution and $Z_{ki}^{*}$'s are i.i.d.\ random vectors distributed as  $N_p(0,\sigma^2 I_p)$ which are independent of $U_{ki}^{*}$'s.
    Denote $U^{*}={(U^*_{11},\ldots,U^{*}_{1(n_1-1)},U^*_{21},\ldots,U^*_{2(n_2-1)})}^T$ and 
    $Z^*={(Z^*_{11},\ldots,Z^*_{1(n_1-1)},Z^*_{21},\ldots,Z^*_{2(n_2-1)})}^T$. Hence we have
        \begin{align*}
            \hat{\sigma}_{(1)}^2=&\frac{1}{(n_1+n_2-2)(p-r)}\|U^*DV^T\hat{\tilde{V}}\|_F^2+\frac{1}{(n_1+n_2-2)(p-r)}\|Z^*\hat{\tilde{V}}\|^2_F \\
            &+\frac{2}{(n_1+n_2-2)(p-r)}\mathrm{tr}\hat{\tilde{V}}^T Z^{*T}U^*DV^T\hat{\tilde{V}}
            \\
            \overset{\textrm{def}}{=}&R_1+R_2+R_3.
            %\frac{1}{(n-1)(p-r)}tr\hat{\tilde{V}}^T VDU^{*T}U^*DV^T\hat{\tilde{V}}^T+\frac{1}{(n-1)(p-r)}tr\hat{\tilde{V}}^TZ^{*T}Z^*\hat{\tilde{V}}\\
        \end{align*}

    By Lemma~\ref{maxEigen}, we have $\lambda_1(U^{*T}U^*)=O_P(n_1+n_2-2)$. Therefore,
    \begin{equation}
        \begin{aligned}
            R_1&=\frac{1}{(n_1+n_2-2)(p-r)}\mathrm{tr}\hat{\tilde{V}}^T VDU^{*T}U^*DV^T\hat{\tilde{V}}\\
            &=O_P(1)\frac{1}{p-r}\mathrm{tr}\hat{\tilde{V}}^T VD^2V^T\hat{\tilde{V}}\\
            &=O_P(1)\frac{1}{p-r}\mathrm{tr} D^2V^T(I-\hat{V}\hat{V}^T)V\\
            &\leq O_P(1)\frac{1}{p}\sqrt{\mathrm{tr}D^4}\sqrt{\mathrm{tr}{(V^T(I-\hat{V}\hat{V}^T)V)}^2}.
        \end{aligned}
    \end{equation}
    And
    \begin{align}
        \mathrm{tr} {(V^T(I-\hat{V}\hat{V}^T)V)}^2
            &\leq {(\mathrm{tr} V^T(I-\hat{V}\hat{V}^T)V)}^2 \label{eq:biti1}\\
            &=\frac{1}{4}\|VV^T-\hat{V}\hat{V}^T\|_F^4 \label{eq:biti2}\\
            &=O(\frac{p^2}{p^{2\beta}{(n_1+n_2)}^2}).\notag
    \end{align}
    The inequality~\eqref{eq:biti1} holds because $V^T(I-\hat{V}\hat{V}^T)V$ is positive semi-definite and equality~\eqref{eq:biti2} is by the fact that $\mathrm{tr} V^T(I-\hat{V}\hat{V}^T)V=\frac{1}{2}\|VV^T-\hat{V}\hat{V}^T\|^2_F$. 
    Since $\mathrm{tr}D^4=O_P(p^{2\beta})$, it follows that $R_1=O_P(\frac{1}{n_1+n_2})\xrightarrow{P}0$.

    We note that
    \begin{equation}
        \begin{aligned}
            &|R_2-\frac{1}{(n_1+n_2-2)(p-r)}\|Z^*\tilde{V}\|_F^2|\\
            &=
            \frac{1}{(n_1+n_2-2)(p-r)}|\mathrm{tr}Z^{*T}Z^*(\hat{\tilde{V}}\hat{\tilde{V}}^T-\tilde{V}\tilde{V}^T)|\\
            &=\frac{1}{(n_1+n_2-2)(p-r)}|\mathrm{tr}Z^{*T}Z^*(\hat{V}\hat{V}^T-VV^T)|\\
            &\leq \frac{1}{(n_1+n_2-2)(p-r)}\|Z^{*T}Z^*\|_F\|\hat{V}\hat{V}^T-VV^T\|_F.
        \end{aligned}
    \end{equation}
    By directly calculating expectation, it's easy to check that $\|Z^{*T}Z^*\|_F=O_P((n_1+n_2-2)p)$. Since $\beta\geq 1/2$  and $\frac{\sqrt{p}}{n}\to 0$, we have
    \begin{equation}
        \begin{aligned}
            \|VV^T -\hat{V}\hat{V}^T\|_F^2=O_P(\frac{p}{p^{\beta}(n_1+n_2)})=o(1).
    \end{aligned}
    \end{equation}
     Therefore, $|R_2-\frac{1}{(n_1+n_2-2)(p-r)}\|Z^*\tilde{V}\|_F^2|\xrightarrow{P}0$. But $\frac{1}{(n_1+n_2-2)(p-r)}\|Z^*\hat{V}\|_F^2\xrightarrow{P}\sigma^2$ by law of large numbers, which yields $R_2\xrightarrow{P}\sigma^2$.

    Finally, as $R_3\leq 2\sqrt{R_1}\sqrt{R_2}$ we have $R_3\xrightarrow{P}0$. It follows that $\hat{\sigma}^2_{(1)}$ is consistent.

% consistency of variance estimator 2
Next we proof the consistency of $\hat{\sigma}^2_{(2)}$.

We denote
\begin{equation}
\begin{aligned}
    S_{UU}=\frac{1}{n_1+n_2-2}\sum_{k=1}^2\sum_{i=1}^{n_k}(U_{ki}-\bar{U}_k){(U_{ki}-\bar{U}_k)}^T,
\\
    S_{ZZ}=\frac{1}{n_1+n_2-2}\sum_{k=1}^2\sum_{i=1}^{n_k}(Z_{ki}-\bar{Z}_k){(Z_{ki}-\bar{Z}_k)}^T,
\\
    S_{UZ}=\frac{1}{n_1+n_2-2}\sum_{k=1}^2\sum_{i=1}^{n_k}(U_{ki}-\bar{U}_k){(Z_{ki}-\bar{Z}_k)}^T,
\\
    S_{ZU}=\frac{1}{n_1+n_2-2}\sum_{k=1}^2\sum_{i=1}^{n_k}(Z_{ki}-\bar{Z}_k){(U_{ki}-\bar{U}_k)}^T.
\end{aligned}
\end{equation}
Then 
\begin{equation}
\begin{aligned}
    S&=\frac{1}{n_1+n_2-2}\sum_{k=1}^2\sum_{i=1}^{n_k}(X_{ki}-\bar{X}_k){(X_{ki}-\bar{X}_k)}^T\\
    &=\frac{1}{n_1+n_2-2}\sum_{k=1}^2\sum_{i=1}^{n_k}(VD(U_{ki}-\bar{U}_k)+Z_{ki}-\bar{Z}_k){(VD(U_{ki}-\bar{U}_k)+Z_{ki}-\bar{Z}_k)}^T\\
&=VDS_{UU}DV^T+VDS_{UZ}+S_{ZU}DV^T+S_{ZZ}\\
&=VD(S_{UU}DV^T+S_{UZ})+S_{ZU}DV^T+S_{ZZ}\\
    &\overset{\textrm{def}}{=}F_1+F_2+F_3.
\end{aligned}
\end{equation}
By Weyl's inequality, we have
\begin{equation}
    \lambda_{p-2r}{(F_1+F_2)}+\lambda_{2r+k}(S_{ZZ})
    \leq \lambda_{k}(S)\leq 
    \lambda_{2r+1}(F_1+F_2)+\lambda_{k-2r}(S_{ZZ})
\end{equation}
for $2r+1\leq k\leq p-2r$.
    But $rank(F_1+F_2)\leq 2r$, since $rank(F_k)\leq rank(D)=r$ for $k=1,2$. Thus the $k$th  eigenvalue of $F_1+F_2$ equals to zero, where $2r+1\leq k\leq p-2r$. In this way,
\begin{equation}
\lambda_{2r+k}(S_{ZZ})\leq \lambda_{k}(S)\leq \lambda_{k-2r}(S_{ZZ}).
\end{equation}
By above argument, we have upper bound
\begin{equation}
    \frac{1}{p-4r}\sum_{k=2r+1}^{p-2r}\lambda_k(S)\leq\frac{1}{p-4r}\sum_{k=1}^{p-4r}\lambda_{k}(S_{ZZ})\leq \frac{1}{p-4r}\mathrm{tr} S_{ZZ}
\end{equation} and lower bound
\begin{equation}
    \frac{1}{p-4r}\sum_{k=2r+1}^{p-2r}\lambda_k(S)\geq\frac{1}{p-4r}\sum_{k=4r+1}^{p}\lambda_{k}(S_{ZZ})\geq \frac{1}{p-4r}\mathrm{tr} S_{ZZ}-\frac{4r}{p-4r}\lambda_1(S_{ZZ}).
\end{equation}
Thus it suffices to  prove that
$\frac{1}{p}\mathrm{tr} S_{ZZ}\xrightarrow{P}\sigma^2$
and
$\frac{1}{p}\lambda_1(S_{ZZ})\xrightarrow{P}0$. 
We note that $(n_1+n_2-2)S_{ZZ}$ is distributed as $\textrm{Wishart}_p(n_1+n_2-2,\sigma^2 I_p)$. Hence $\frac{1}{p}\mathrm{tr}(S_{ZZ})\xrightarrow{P}\sigma^2$ by law of large numbers. And $\frac{1}{p}\lambda_1(S_{ZZ})=o_P(1)$ by Lemma~\ref{maxEigen}. Therefore the consistency of $\hat{\sigma}^2_{(2)}$ is proved.
\end{proof}
