
\begin{proof}[\textbf{Proof Of Theorem 2}]
    By~\cite{Chen2010A}'s Theorem 1, we have
    \begin{equation*}
        \frac{n_1 n_2 T_1}{\sqrt{2p}n\sigma^2}\xrightarrow{\mathcal{L}}N(0,1).
    \end{equation*}
    It suffices to prove
\begin{equation*}
\frac{n_1n_2(T_1-T_2)}{\sqrt{2p}n\sigma^2}\xrightarrow{P}0.
\end{equation*}
Since the test statistic is invariant under transformation $X_{1i}\mapsto X_{1i}+\mu$, $X_{2j}\mapsto X_{2j}+\mu$, without loss of generality, we assume $\mu_1=\mu_2=0$.
    By noting that $\hat{\tilde{V}}\hat{\tilde{V}}^T -\tilde{V}\tilde{V}^T =VV^T -\hat{V}\hat{V}^T $ we have ${n_1n_2(T_1-T_2)}/{(\sqrt{2p}n\sigma^2)}=P_1+P_2-2P_3$, where

\begin{equation*}
        P_1=\frac{n_2\sum_{i\neq j}X_{1i}^T(VV^T-\hat{V}\hat{V}^T)X_{1j}}{\sqrt{2p}(n_1+n_2)(n_1-1)\sigma^2},
\end{equation*}
\begin{equation*}
        P_2=\frac{n_1\sum_{i\neq j}X_{2i}^T(VV^T-\hat{V}\hat{V}^T)X_{2j}}{\sqrt{2p}(n_1+n_2)(n_2-1)\sigma^2},
\end{equation*}
\begin{equation*}
        P_3=\frac{\sum_{i=1}^{n_1}\sum_{j=1}^{n_2}X_{1i}^T(VV^T-\hat{V}\hat{V}^T)X_{2j}}{\sqrt{2p}(n_1+n_2)\sigma^2}.
\end{equation*}
Write
\begin{equation*}
    \begin{aligned}
        P_1=O(1)\frac{\sum_{i\neq j}X_{1i}^T(VV^T-\hat{V}\hat{V}^T)X_{1j}}{n_1\sqrt{p}}
        =O(1)(R_1-R_2),
    \end{aligned}
\end{equation*}
where
\begin{equation*}
    R_1=\frac{n_1}{\sqrt{p}}(\bar{X}_1^T VV^T\bar{X}_1-\bar{X}_1^T \hat{V}\hat{V}^T\bar{X}_1)
\end{equation*}
and 
\begin{equation*}
    R_2=\frac{1}{n_1\sqrt{p}}\sum_{i=1}^{n_1}(X_{1i}^T VV^T X_{1i}-X_{1i}^T \hat{V}\hat{V}^T X_{1i}).
\end{equation*}
To deal with $R_1$, we further decompose $\bar{X}_1^T \hat{V}\hat{V}^T\bar{X}_1$  into $3$ parts
    \begin{equation*}
    \begin{aligned} 
        \bar{X}_1^T \hat{V}\hat{V}^T\bar{X}_1=&
        \bar{X}_1^T (VV^T+\tilde{V}\tilde{V}^T) \hat{V}\hat{V}^T (VV^T+\tilde{V}\tilde{V}^T)\bar{X}_1\\
        =&\bar{X}_1^T VV^T \hat{V}\hat{V}^T VV^T \bar{X}_1
        +2\bar{X}_1^T \tilde{V}\tilde{V}^T \hat{V}\hat{V}^T VV^T \bar{X}_1\\
        &+\bar{X}_1^T VV^T \hat{V}\hat{V}^T VV^T \bar{X}_1.\\
    \end{aligned}
    \end{equation*} 
    The above decomposition is a technique we will use many times. $R_1$ can thus be written by $R_1=Q_1-2Q_2-Q_3$, where
    \begin{equation*}
        Q_1=\frac{n_1}{\sqrt{p}}\bar{X}_1^T V(I_r-V^T\hat{V}\hat{V}^T V)V^T \bar{X}_1,
    \end{equation*}
    \begin{equation*}
        Q_2=\frac{n_1}{\sqrt{p}}\bar{X}_1^T \tilde{V}\tilde{V}^T\hat{V}\hat{V}^T VV^T \bar{X}_1,
    \end{equation*}
    \begin{equation*}
        Q_3=\frac{n_1}{\sqrt{p}}\bar{X}_1^T \tilde{V}\tilde{V}^T\hat{V}\hat{V}^T \tilde{V}\tilde{V}^T \bar{X}_1.
    \end{equation*}
    It's clear that $V^T \bar{X}_1 \sim N_r(0,\frac{1}{n}(\Lambda+\sigma^2 I_p))$, $\tilde{V}^T \bar{X}_1 \sim N_{p-r}(0,\frac{\sigma^2 }{n}I_p)$ and $S$ are mutually independent.  $\|V^T\bar{X}_1\|^2=O_P(\frac{p^{\beta}}{n_1})$.  $I_r-V^T \hat{V}\hat{V}^T V$ is a positive semi-difinite matrix which only relies  on $S$. Combining these observations, we have
    

    \begin{equation}\label{myQ1}
        \begin{aligned}
            |Q_1|&\leq \frac{n_1}{\sqrt{p}}\|I_r-V^T \hat{V}\hat{V}^T V\| \|V^T\bar{X}_1\|^2\\
            &\leq \frac{n_1}{\sqrt{p}}\mathrm{tr}(I_r-V^T \hat{V}\hat{V}^T V) \|V^T\bar{X}_1\|^2\\
            &=\frac{n_1}{2\sqrt{p}}\|VV^T -\hat{V}\hat{V}^T\|^2_F \|V^T\bar{X}_1\|^2\\
            &=O_P(1)\frac{n_1}{\sqrt{p}}\frac{p}{p^{\beta}n_1}\frac{p^\beta}{n_1}\xrightarrow{P}0.
        \end{aligned}
    \end{equation}

We next prove $Q_2\xrightarrow{L^2}0$.
\begin{align}
        E(Q_2^2)&=\frac{n_1^2}{p}E(\bar{X}_1^T \tilde{V} \tilde{V}^T \hat{V}\hat{V}^T VV^T \bar{X}_1\bar{X}_1^T VV^T \hat{V}\hat{V}^T \tilde{V}\tilde{V}^T \bar{X}_1)\label{myQ2}\\
        &=\frac{n_1^2}{p}E(\bar{X}_1^T \tilde{V} \tilde{V}^T \hat{V}\hat{V}^T V\frac{1}{n_1}(\Lambda+\sigma^2I_r)V^T \hat{V}\hat{V}^T \tilde{V}\tilde{V}^T \bar{X}_1)\notag\\
        &=O(1)\frac{n_1p^{\beta}}{p}\mathrm{E}(\bar{X}_1^T \tilde{V} \tilde{V}^T \hat{V}\hat{V}^T VV^T \hat{V}\hat{V}^T \tilde{V}\tilde{V}^T \bar{X}_1)\notag\\
        &=O(1)\frac{n_1p^{\beta}}{p}\mathrm{E}\, \mathrm{tr}(\tilde{V}^T \hat{V}\hat{V}^T VV^T \hat{V}\hat{V}^T \tilde{V}\tilde{V}^T \bar{X}_1\bar{X}_1^T \tilde{V} )\notag\\
        &=O(1)\frac{n_1p^{\beta}}{p}\frac{\sigma^2}{n_1}\mathrm{E}\, \mathrm{tr}(\tilde{V}^T \hat{V}\hat{V}^T VV^T \hat{V}\hat{V}^T \tilde{V})\notag\\
        &\leq O(1)\frac{p^{\beta}}{p}\mathrm{E}\, \mathrm{tr}(\tilde{V}^T \hat{V}\hat{V}^T I_p \hat{V}\hat{V}^T \tilde{V})\notag\\
        &= O(1)\frac{p^{\beta}}{p}\mathrm{E}\, \mathrm{tr}(\tilde{V}^T \hat{V}\hat{V}^T \tilde{V})\notag\\
        &= O(1)\frac{p^{\beta}}{p}\mathrm{E}\, \mathrm{tr}( \hat{V}\hat{V}^T (I_p-VV^T ))\notag\\
        &= O(1)\frac{p^{\beta}}{p}\frac{1}{2}\mathrm{E}\|VV^T -\hat{V}\hat{V}^T \|_F^2\notag\\
        &= O(\frac{1}{n_1})\to 0.\notag
\end{align}
Similarly,
\begin{align}
        E(Q_3)&=\frac{n_1}{\sqrt{p}}\frac{\sigma^2}{n_1}E\, \mathrm{tr}(\tilde{V}^T \hat{V}\hat{V}^T \tilde{V})\label{myQ3}\\
        &\leq \frac{r\sigma^2}{\sqrt{p}}\to 0.   \notag
\end{align}
By~\eqref{myQ1},~\eqref{myQ2} and~\eqref{myQ3}, $R_1\xrightarrow{P}0$. Next we deal with $R_2$. Write $R_2=W_1-2W_2-W_3$, where

    \begin{equation*}
        W_1=\frac{1}{n_1\sqrt{p}}\sum_{i=1}^{n_1}X_{1i}^T V(I_r-V^T\hat{V}\hat{V}^T V)V^T X_{1i},
    \end{equation*}
    \begin{equation*}
        W_2=\frac{1}{n_1\sqrt{p}}\sum_{i=1}^{n_1}X_{1i}^T \tilde{V}\tilde{V}^T\hat{V}\hat{V}^T VV^T X_{1i},
    \end{equation*}
    \begin{equation*}
        W_3=\frac{1}{n_1\sqrt{p}}\sum_{i=1}^{n_1}X_{1i}^T \tilde{V}\tilde{V}^T\hat{V}\hat{V}^T \tilde{V}\tilde{V}^T X_{1i}.
    \end{equation*}
It is seen that independence property does not holds anymore compared with the case we deal with $R_1$. Nevertheless, the form is of sum, which makes it possible to apply law of large numbers.

We note that
    \begin{equation}\label{803W1}
        \begin{aligned}
            W_1&=\frac{1}{n_1\sqrt{p}}\mathrm{tr} ((I_r-V^T\hat{V}\hat{V}^T V)\sum_{i=1}^{n_1}V^T X_{1i}X_{1i}^T V).
        \end{aligned}
    \end{equation}
By law of large numbers, 
\begin{equation*}
\frac{1}{n_1}{(\Lambda+\sigma^2 I_{r})}^{-\frac{1}{2}}\sum_{i=1}^{n_1}V^T X_{1i}X_{1i}^T V {(\Lambda+\sigma^2 I_{r})}^{-\frac{1}{2}}\xrightarrow{P}I_{r}.
\end{equation*}
Hence $\lambda_1(\sum_{i=1}^{n_1}V^T X_{1i}X_{1i}^T V )=O_P(n_1 p^{\beta})$. Substituting it into~\eqref{803W1} yields

\begin{equation}\label{myW1}
    \begin{aligned}
        W_1&\leq O_P(1)\frac{p^{\beta}}{\sqrt{p}}\mathrm{tr}(I_r-V^T \hat{V}\hat{V}^T V)\\
        &=O_P(1)\frac{p^{\beta}}{\sqrt{p}}\|VV^T -\hat{V}\hat{V}^T\|^2_F\\
        &=O_P(1)\frac{p^\beta}{\sqrt{p}}\frac{p}{p^\beta n_1}\\
        &=O_P(\frac{\sqrt{p}}{n_1})\xrightarrow{P}0.
    \end{aligned}
\end{equation}
 Next we deal with $W_2$. By cauchy inequality, we have
\begin{equation*}
    \begin{aligned}
        W_2 &=\frac{1}{n_2 \sqrt{p}}\mathrm{tr} \tilde{V}^T \hat{V}\hat{V}^T V (\sum_{i=1}^{n_1}V^T X_{1i}X_{1i}^T \tilde{V})\\
        &\leq \frac{1}{n_2 \sqrt{p}}\sqrt{\mathrm{tr}(\tilde{V}^T \hat{V}\hat{V}^T VV^T \hat{V}\hat{V}^T \tilde{V})}\sqrt{\mathrm{tr}(VZZ^T \tilde{V}\tilde{V}^T ZZ^T V^T)},
    \end{aligned}
\end{equation*}
where $Z=(X_{11},\ldots,X_{1n_1})$. Note that
\begin{equation*}
    \begin{aligned}
        \mathrm{tr}(\tilde{V}^T \hat{V}\hat{V}^T VV^T \hat{V}\hat{V}^T \tilde{V})&\leq \mathrm{tr} (\tilde{V}^T\hat{V}\hat{V}^T\tilde{V})=\frac{1}{2}\|VV^T-\hat{V}\hat{V}^T\|^2_F
    \end{aligned}
\end{equation*}
and
\begin{equation*}
    \begin{aligned}
        \mathrm{tr}(VZZ^T \tilde{V}\tilde{V}^T ZZ^T V^T)&\leq
        \lambda_1 (Z^T\tilde{V}\tilde{V}^T Z)\mathrm{tr}(VZZ^T V^T).
    \end{aligned}
\end{equation*}
$Z^T\tilde{V}\tilde{V}^T Z$ is distributed as $\textrm{Wishart}_{n}(p-r,\sigma^2 I_{n_1})$, hence $\lambda_1 (Z^T\tilde{V}\tilde{V}^T Z)=O_P(\max(n_1,p))$ by Lemma~\ref{maxEigen}. Again by law of large numbers $\mathrm{tr}(VZZ^T V^T)=O(p^{\beta} n_1)$. Combining the argument above yields
\begin{equation}\label{myW2}
    \begin{aligned}
        W_2&=O_P(1)\frac{1}{n_1\sqrt{p}}\sqrt{\frac{p}{p^{\beta}n_1}}\sqrt{\max(n_1,p)p^{\beta}n_1}\\
        &=O_P(\frac{\sqrt{\max(n_1,p)}}{n_1})\xrightarrow{P}0.
    \end{aligned}
\end{equation}

To deal with $W_3$, note that $\tilde{V}^T\sum_{i=1}^{n_1}X_{1i}X_{1i}^T\tilde{V}$ is of distribution $\textrm{Wishart}_{p-r}(n,\sigma^2 I_{p-r})$. By Lemma~\ref{maxEigen}, $\lambda_1(\tilde{V}^T\sum_{i=1}^{n_1}X_{1i}X_{1i}^T\tilde{V})=O_P(\max(p,n_1))$. 
Hence

\begin{equation}\label{myW3}
    \begin{aligned}
        W_3&=\frac{1}{n_1\sqrt{p}}\mathrm{tr}(\hat{V}^T\tilde{V}\tilde{V}^T\sum_{i=1}^{n_1}X_{1i}X_{1i}^T\tilde{V}\tilde{V}^T \hat{V})\\
        &\leq \frac{O_{P}(\max(n_1,p))}{n_1\sqrt{p}}\mathrm{tr} \hat{V}^T\tilde{V}\tilde{V}^T\hat{V}\xrightarrow{P}0,
    \end{aligned}
\end{equation}
due to $\mathrm{tr} \hat{V}^T\tilde{V}\tilde{V}^T\hat{V}=O_P(1)$. Combining~\eqref{myW1},~\eqref{myW2} and~\eqref{myW3} gives $R_2\xrightarrow{P}0$. So far, it has been proved that $P_1\xrightarrow{P}0$. By symmetry, $P_2\xrightarrow{P}0$. 

We proceed to deal with $P_3$. Note that $P_3=O(1)(M_1-M_2-M_3-M_4)$, where
    \begin{equation*}
        M_1=\frac{n_1}{\sqrt{p}}\bar{X}_1^T V(I_r-V^T\hat{V}\hat{V}^T V)V^T \bar{X}_2,
    \end{equation*}
    \begin{equation*}
        M_2=\frac{n_1}{\sqrt{p}}\bar{X}_1^T \tilde{V}\tilde{V}^T\hat{V}\hat{V}^T VV^T \bar{X}_2,
    \end{equation*}
    \begin{equation*}
        M_3=\frac{n_1}{\sqrt{p}}\bar{X}_1^T {V}{V}^T\hat{V}\hat{V}^T \tilde{V}\tilde{V}^T \bar{X}_2,
    \end{equation*}
    \begin{equation*}
        M_4=\frac{n_1}{\sqrt{p}}\bar{X}_1^T \tilde{V}\tilde{V}^T\hat{V}\hat{V}^T \tilde{V}\tilde{V}^T \bar{X}_2.
    \end{equation*}
Since $\|V^T\bar{X}_i\|^2=O_P(\frac{p^{\beta}}{n_1})$ for $i=1,2$, we have
    \begin{equation}\label{myM1}
        \begin{aligned}
            |M_1|&\leq \frac{n_1}{\sqrt{p}}\|I_r-V^T \hat{V}\hat{V}^T V\| \|V^T\bar{X}_1\|\|V^T\bar{X}_2\|\\
            &\leq \frac{n_1}{2\sqrt{p}}\|VV^T -\hat{V}\hat{V}^T\|^2_F \|V^T\bar{X}_1\|\|V^T\bar{X}_2\|\\
            &=O_P(1)\frac{n_1}{\sqrt{p}}\frac{p}{p^{\beta}n_1}\frac{p^\beta}{n_1}\xrightarrow{P}0.
        \end{aligned}
    \end{equation}
The joint distribution of ($V^T\bar{X}_2$, $\tilde{V}^T\bar{X}_1$,$S$) is identity to that of ($V^T\bar{X}_1$, $\tilde{V}^T\bar{X}_1$,$S$). Therefore, $M_2$ has the same distribution with $Q_2$ and converges to $0$ in probability. The same reasoning yields $M_3\xrightarrow{P}0$.

    Applying cauchy inequality gives
    \begin{equation*}
        \begin{aligned}
            M_4\leq \sqrt{\frac{n_1}{\sqrt{p}}\bar{X}_1^T\tilde{V}\tilde{V}^T \hat{V}\hat{V}^T \tilde{V}\tilde{V}^T\bar{X}_1}
            \sqrt{\frac{n_1}{\sqrt{p}}\bar{X}_2^T\tilde{V}\tilde{V}^T \hat{V}\hat{V}^T \tilde{V}\tilde{V}^T\bar{X}_2}.
        \end{aligned}
    \end{equation*}
    Then $M_4\xrightarrow{P} 0$ by the same reason as $Q_3\xrightarrow{P} 0$, and consequently $P_3\xrightarrow{P}0$. This finishes the proof.

\end{proof}

\begin{proof}[\textbf{Proof Of Theorem 3}]
    By~\cite{Chen2010A}'s Theorem 1, we have
    \begin{equation*}
        \frac{n_1 n_2 (T_1-\|\tilde{V}(\mu_1-\mu_2)\|^2)}{\sqrt{2p}(n_1+n_2)\sigma^2}\xrightarrow{\mathcal{L}}N(0,1).
    \end{equation*}
    It suffices to prove
\begin{equation*}
\frac{n_1n_2(T_1-T_2)}{\sqrt{2p}(n_1+n_2)\sigma^2}\xrightarrow{P}0.
\end{equation*}
Write $T_1=T_1^{(1)}+T_1^{(2)}$, where
\begin{equation*}
\begin{aligned}
    T_1^{(1)}=&\frac{\sum_{i\neq j}^{n_1}{(X_{1i}-\mu_1)}^T\tilde{V}\tilde{V}^T(X_{1j}-\mu_1)}{n_1(n_1-1)}+\frac{\sum_{i\neq j}^{n_2}{(X_{2i}-\mu_2)}^T\tilde{V}\tilde{V}^T(X_{2j}-\mu_2)}{n_2(n_2-1)}
\\
    &-2\frac{\sum_{i=1}^{n_1}\sum_{j=1}^{n_2}{(X_{1i}-\mu_1)}^T\tilde{V}\tilde{V}^T(X_{2j}-\mu_2)}{n_1n_2}
\end{aligned}
\end{equation*}
and
\begin{equation*}
\begin{aligned}
    T_1^{(2)}=&2{(\bar{X_{1}}-\mu_1)}^T\tilde{V}\tilde{V}^T(\mu_1-\mu_2)+{(\bar{X_{2}}-\mu_2)}^T\tilde{V}\tilde{V}^T(\mu_2-\mu_1)
\\
&+\|\tilde{V}(\mu_1-\mu_2)\|^2.
\end{aligned}
\end{equation*}
    Similarly, $T_2=T_2^{(1)}+T_2^{(2)}$. By Theorem~\ref{myPanpan}, it holds that
    \begin{equation}\label{yintianHaha}
        \begin{aligned}
            \frac{n_1n_2(T_2^{(1)}-T_1^{(1)})}{\sqrt{2p}(n_1+n_2)\sigma^2}\xrightarrow{P}0.
        \end{aligned}
    \end{equation}
We are left with the task of dealing with $T_{1}^{(2)}$ and $T_{2}^{(2)}$. Note that
\begin{align}
    \frac{n_1n_2(T_2^{(2)}-T_1^{(2)})}{\sqrt{2p}(n_1+n_2)\sigma^2}=&O(1)\frac{n_1}{\sqrt{p}}{(\bar{X_{1}}-\mu_1)}^T(VV^T -\hat{V}\hat{V}^T)(\mu_1-\mu_2)+
\label{yumenHaha1}\\
    &O(1)\frac{n_2}{\sqrt{p}}{(\bar{X_{2}}-\mu_2)}^T(VV^T-\hat{V}\hat{V}^T)(\mu_2-\mu_1)+
\label{yumenHaha2}\\
    &O(1)\frac{n_1+n_2}{\sqrt{p}}{(\mu_1-\mu_2)}^T(VV^T-\hat{V}\hat{V}^T)(\mu_1-\mu_2).
\label{yumenHaha3}
\end{align}

    We proceed to deal with the~\eqref{yumenHaha1}.
\begin{align}
    &E|\frac{n_1}{\sqrt{p}}{(\bar{X_{1}}-\mu_1)}^T(VV^T-\hat{V}\hat{V}^T)(\mu_1-\mu_2)|^2
\label{theStartHaha}
\\
    &=\frac{n_1^2}{p}E{(\mu_1-\mu_2)}^T(VV^T-\hat{V}\hat{V}^T)\frac{\Sigma}{n_1}(VV^T-\hat{V}\hat{V}^T)(\mu_1-\mu_2)
    \notag\\
    &=O(n_1p^{\beta-1})E{(\mu_1-\mu_2)}^T{(VV^T-\hat{V}\hat{V}^T)}^2(\mu_1-\mu_2)
    \notag\\
&\leq
O(n_1p^{\beta-1})\|\mu_1-\mu_2\|^2E\|VV^T-\hat{V}\hat{V}^T\|^2_F
    \notag\\
    &=O(n_1p^{\beta-1})\frac{p}{p^{\beta}n_1}\|\mu_1-\mu_2\|^2
    \notag\\
    &=O(\|\mu_1-\mu_2\|^2).
    \notag
\end{align}
Combining with conditions $\frac{(n_1+n_2)}{\sqrt{p}}\|\mu_1-\mu_2\|^2=O(1)$ and $\frac{\sqrt{p}}{n_1+n_2}\to 0$, it follows that~\eqref{theStartHaha} converges to $0$ in probability.
    By symmetry,~\eqref{yumenHaha2} also converges to $0$ in probability. It remains to deal with~\eqref{yumenHaha3}. But
\begin{align}
    &E|\frac{n_1+n_2}{\sqrt{p}}{(\mu_1-\mu_2)}^T(VV^T-\hat{V}\hat{V}^T )(\mu_1-\mu_2)|
\label{mmmHaha}
\\
&\leq
\frac{n_1+n_2}{\sqrt{p}}\|\mu_1-\mu_2\|^2 E\|VV^T-\hat{V}\hat{V}^T\|
\notag
\\
    &\leq    \frac{n_1+n_2}{\sqrt{p}}\|\mu_1-\mu_2\|^2 \sqrt{E\|VV^T-\hat{V}\hat{V}^T\|^2}
\notag
\\
    &\leq O(1)\sqrt{E\|VV^T-\hat{V}\hat{V}^T\|^2_F}.
\notag
\end{align}
Since $E\|VV^T-\hat{V}\hat{V}^T\|^2_F=O(\frac{p}{p^{\beta}n_1})$ and $\beta\geq 1/2$,~\eqref{mmmHaha} converges to $0$.

It follows that
\begin{equation*}
\begin{aligned}
\frac{n_1n_2(T_2^{(2)}-T_1^{(2)})}{\sqrt{2p}(n_1+n_2)\sigma^2}\xrightarrow{P}0.
\end{aligned}
\end{equation*}
Together with~\eqref{yintianHaha}, the theorem follows.
\end{proof}
