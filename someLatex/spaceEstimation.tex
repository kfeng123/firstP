
\begin{proof}[\textbf{Proof Of Theorem~\ref{myPanpan}}]

    Note that $\mathrm{tr}(\hat{\tilde{V_i}}^T S_i\hat{\tilde{V_i}})=\sum_{i=r+1}^p \lambda_i(S_i)$, $i=1,2$. Similar to Proposition~\ref{varianceEstimation}, we have that $\mathrm{tr}(\hat{\tilde{V_i}}^T S_i\hat{\tilde{V_i}})=(p-r)\sigma^2+O_P(\frac{\max(n,p)}{n})$, $i=1,2$.
    Hence
\begin{equation*}
        \frac{T_2-\|\tilde{V}^T(\mu_1-\mu_2)\|^2}{\sigma^2\sqrt{2\tau^2 p}}
        =
        \frac{\|\hat{\tilde{V}}^T(\bar{X}_1-\bar{X}_2)\|^2-\|\tilde{V}^T(\mu_1-\mu_2)\|^2
        -\sigma^2\tau (p-r)
        }{\sigma^2\sqrt{2\tau^2 p}}
        +O_P(\frac{\max(n,p)}{n\sqrt{p}}).
\end{equation*}
    By Assumption~\ref{pAndN}, $\frac{\max(n,p)}{n\sqrt{p}}=\max(\frac{1}{\sqrt{p}},\frac{\sqrt{p}}{n})\to 0$.
    And
\begin{equation*}
    \begin{aligned}
        &\frac{\|\hat{\tilde{V}}^T(\bar{X}_1-\bar{X}_2)\|^2-\|\tilde{V}^T(\mu_1-\mu_2)\|^2
        -\sigma^2\tau (p-r)
        }{\sigma^2\sqrt{2\tau^2 p}}
        \\
        =&\frac{1}{\sigma^2\sqrt{2\tau^2 p}}\Big(
        \|\hat{\tilde{V}}^T\big((\bar{X}_1-\mu_1)-(\bar{X}_2-\mu_2)\big)\|^2-\sigma^2 \tau (p-r)+\\
        &2{(\mu_1-\mu_2)}^T \hat{\tilde{V}}\hat{\tilde{V}}^T\big((\bar{X}_1-\mu_1)-(\bar{X}_2-\mu_2)\big)
        +\|\hat{\tilde{V}}^T(\mu_1-\mu_2)\|^2-\|\tilde{V}^T(\mu_1-\mu_2)\|^2
        \Big).
    \end{aligned}
\end{equation*}
Let 
\begin{align*}
    P_1&=\|\hat{\tilde{V}}^T\big((\bar{X}_1-\mu_1)-(\bar{X}_2-\mu_2)\big)\|^2-\sigma^2 \tau (p-r),\\
    P_2&=2{(\mu_1-\mu_2)}^T \hat{\tilde{V}}\hat{\tilde{V}}^T\big((\bar{X}_1-\mu_1)-(\bar{X}_2-\mu_2)\big),\\
    P_3&=\|\hat{\tilde{V}}^T(\mu_1-\mu_2)\|^2-\|\tilde{V}^T(\mu_1-\mu_2)\|^2.
\end{align*}
To prove the theorem, we only need to show that
$$
    \frac{P_1}{\sigma^2\sqrt{2\tau^2 p}}\xrightarrow{\mathcal{L}} N(0,1),
    \quad
    \frac{P_2}{\sigma^2\sqrt{2\tau^2 p}}\xrightarrow{P} 0
    \quad
    \textrm{and}
    \quad
    \frac{P_3}{\sigma^2\sqrt{2\tau^2 p}}\xrightarrow{P}0.
    $$
    We first deal with $P_2$.
    To proves the convergence in probability, we only need to prove the convergence in $L^2$.
    Note that $\bar{X}_1$, $\bar{X}_2$ and $S$ are mutually independent. And $\hat{\tilde{V}}\hat{\tilde{V}}^T$ only depends on $S$, thus
    \begin{equation*}
        \begin{aligned}
            &\mathrm{E} P_2^2
            =
            \mathrm{E}[\mathrm{E} P_2^2|S]= 4\tau \mathrm{E}[{(\mu_1-\mu_2)}^T \hat{\tilde{V}}\hat{\tilde{V}}^T\Sigma \hat{\tilde{V}}\hat{\tilde{V}}^T(\mu_1-\mu_2)]\\
            \leq &
             4\tau\mathrm{E}[\lambda_1(\hat{\tilde{V}}^T\Sigma \hat{\tilde{V}}) {(\mu_1-\mu_2)}^T \hat{\tilde{V}}\hat{\tilde{V}}^T(\mu_1-\mu_2)]
            \leq 
             4\tau\|\mu_1-\mu_2\|^2
             \mathrm{E}[\lambda_1(\hat{\tilde{V}}^T\Sigma \hat{\tilde{V}}) ]\\
             =&
             O(\frac{\sqrt{p}}{n^2})
             \mathrm{E}[\lambda_1(\hat{\tilde{V}}^T (VD^2V^T +\sigma^2 I_p) \hat{\tilde{V}})]
             \leq 
             O(\frac{\sqrt{p}}{n^2})
             \big(\kappa p^{\beta}\mathrm{E}[\lambda_1(\hat{\tilde{V}}^T VV^T  \hat{\tilde{V}})]+\sigma^2\big).\\
        \end{aligned}
    \end{equation*}
    By the following useful relationship
    \begin{equation*}
        \begin{aligned}
\lambda_1(\hat{\tilde{V}}^T VV^T  \hat{\tilde{V}})
            \leq
            \mathrm{tr}(\hat{\tilde{V}}^T VV^T  \hat{\tilde{V}})
            =
            \frac{1}{2}\|VV^T-\hat{V}\hat{V}^T\|^2_F
        \end{aligned}
    \end{equation*}
    and Lemma~\ref{conRateLemma}, we have that
    \begin{equation*}
        \begin{aligned}
            &\mathrm{E} P_2^2
             =
             O(\frac{\sqrt{p}}{n^2})
             \big(O(\frac{p}{n})+\sigma^2\big)
             =o(\frac{p}{n^2}).
        \end{aligned}
    \end{equation*}
    As for $P_3$. To prove the convergence in probability, here we prove the convergence in $L^1$: 
    \begin{equation*}
        \begin{aligned}
            &\mathrm{E}|P_3|=
            \mathrm{E}\big|{(\mu_1-\mu_2)}^T(\hat{\tilde{V}}\hat{\tilde{V}}^T-\tilde{V}\tilde{V}^T)(\mu_1-\mu_2)\big|
            \leq 
            \|\mu_1-\mu_2\|^2\mathrm{E}\|\hat{\tilde{V}}\hat{\tilde{V}}^T-\tilde{V}\tilde{V}^T\|\\
            =& 
            \|\mu_1-\mu_2\|^2\mathrm{E}\|\hat{V}\hat{V}^T-VV^T\|
            \leq 
            \|\mu_1-\mu_2\|^2\sqrt{\mathrm{E}\|\hat{V}\hat{V}^T-VV^T\|^2}\\
            \leq &
            \|\mu_1-\mu_2\|^2\sqrt{\mathrm{E}\|\hat{V}\hat{V}^T-VV^T\|^2_F}
            =O(\frac{\sqrt{p}}{n})\sqrt{O(\frac{p}{p^{\beta}n})}=o(\frac{\sqrt{p}}{n}).
        \end{aligned}
    \end{equation*}

    Now we prove the asymptotic normality of $P_1$. To make clear the sense of convergence, we need a metric for weak convergence. For two distribution function $F$ and $G$, the Levy metric $\rho$ of $F$ and $G$ is defined as
    $$
   \rho(F,G) =\inf\{\epsilon:F(x-\epsilon)-\epsilon\leq G(x)\leq F(x+\epsilon)+\epsilon\quad \textrm{for all $x$}\}.
    $$
    It's well known that $\rho(F_n,F)\to 0$ if and only if $F_n\xrightarrow{\mathcal{L}}F$.

    The conditional distribution of
    $\hat{\tilde{V}}^T\big((\bar{X}_1-\mu_1)-(\bar{X}_2-\mu_2)\big)$ given $S$ is $N(0,\tau \hat{\tilde{V}}^T\Sigma\hat{\tilde{V}})$.
As we have shown,
    $$
    \lambda_1(\hat{\tilde{V}}^T\Sigma\hat{\tilde{V}})\leq \frac{1}{2}\kappa p^\beta \|VV^T -\hat{V}\hat{V}^T\|^2_F+\sigma^2=O_P(\frac{p}{n}+1).
    $$
    On the other hand,
    $
    \lambda_i(\hat{\tilde{V}}^T\Sigma\hat{\tilde{V}})=\sigma^2
    $ for $i=r+1,\ldots,p-r$. Then
    $$
    (p-2r)\sigma^4
    \leq
    \mathrm{tr}(\hat{\tilde{V}}^T\Sigma\hat{\tilde{V}})^2
    \leq
    {\big(\frac{p}{n}+1\big)}^2O_P(1)
    +
    (p-2r)\sigma^4,
    $$ 
or 
\begin{equation}\label{traceA1}
\mathrm{tr}(\hat{\tilde{V}}^T\Sigma\hat{\tilde{V}})^2=p\sigma^4(1+o_P(1)).
\end{equation}

    It follows that
\begin{equation}\label{inProbC}
        \frac{\lambda_1^2(\hat{\tilde{V}}^T\Sigma\hat{\tilde{V}})}{\mathrm{tr}(\hat{\tilde{V}}^T\Sigma\hat{\tilde{V}})^2}
        =O_P\Big(\frac{{(p/n+1)}^2}{p}\Big)=o_P(1).
\end{equation}
Then for every subsequence of $\{n\}$, there's a further subsequence along which~\eqref{inProbC} holds almost surely.
By Lemma~\ref{quadraticFormCLT}, for every subsequence of $\{n\}$, there's a further subsequence along which we have
$$
\rho\Big(\mathcal{L}\Big(\frac{\|\hat{\tilde{V}}^T\big((\bar{X}_1-\mu_1)-(\bar{X}_2-\mu_2)\big)\|^2-\tau\mathrm{tr}(\hat{\tilde{V}}^T\Sigma\hat{\tilde{V}})}{\sqrt{2\tau^2\mathrm{tr}(\hat{\tilde{V}}^T\Sigma\hat{\tilde{V}})^2}}\Big|S\Big),N(0,1)\Big)\xrightarrow{a.s.} 0.
$$
It means that
$$
\rho\Big(\mathcal{L}\Big(\frac{\|\hat{\tilde{V}}^T\big((\bar{X}_1-\mu_1)-(\bar{X}_2-\mu_2)\big)\|^2-\tau\mathrm{tr}(\hat{\tilde{V}}^T\Sigma\hat{\tilde{V}})}{\sqrt{2\tau^2\mathrm{tr}(\hat{\tilde{V}}^T\Sigma\hat{\tilde{V}})^2}}\Big|S\Big),N(0,1)\Big)\xrightarrow{P} 0.
$$
Thus the weak convergence also holds unconditionally:
$$
\frac{\|\hat{\tilde{V}}^T\big((\bar{X}_1-\mu_1)-(\bar{X}_2-\mu_2)\big)\|^2-\tau\mathrm{tr}(\hat{\tilde{V}}^T\Sigma\hat{\tilde{V}})}{\sqrt{2\tau^2\mathrm{tr}(\hat{\tilde{V}}^T\Sigma\hat{\tilde{V}})^2}}\xrightarrow{\mathcal{L}}N(0,1).
$$

Similar to~\eqref{traceA1} we have
\begin{equation}\label{traceA2}
    \mathrm{tr}(\hat{\tilde{V}}^T\Sigma\hat{\tilde{V}})=(p-r)\sigma^2\big(1+O_P\big(\frac{1}{n}+\frac{1}{p}\big)\big).
\end{equation}
By~\eqref{traceA1},~\eqref{traceA2} and Slulsk's theorem,
$$
\frac{\|\hat{\tilde{V}}^T\big((\bar{X}_1-\mu_1)-(\bar{X}_2-\mu_2)\big)\|^2-\sigma^2\tau(p-r) }{\sigma^2\sqrt{2\tau^2 p}}\xrightarrow{\mathcal{L}}N(0,1).
$$
Now the desired asymptotic properties of $P_1$, $P_2$ and $P_3$ are established, the theorem follows.
\end{proof}

