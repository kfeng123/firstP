\section{PCA Theory}

We give some PCA theory here. Compared with existing results, we impose less assumptions since our main task is to obtain the properties of principal space.
    \begin{assumption}\label{PCAassump}
    Suppose $X_i=\mu +U\Lambda^{1/2}Z_i$, $i=1,\ldots,n$, where $Z_{i}$ is a random vector with $p$ indepent entries with $\mathrm{E}Z_{ij}=0$ and $\mathrm{Var}Z_{ij}=1$, $i=1,\ldots,n$, $j=1,\ldots,p$, $\Lambda=\mathrm{diag}(\lambda_1,\ldots,\lambda_p)$ with $\lambda_1\geq\cdots\geq \lambda_p$ and $U$ is a $p$ dimensional orthogonal matrix.
    Suppose  $\lambda_{r+1}\asymp \lambda_p\asymp 1$. 
    \end{assumption}


    Denote by $S=\frac{1}{n}\sum_{i=1}^n (X_i-\bar{X}){(X_i-\bar{X})}^T$ the sample covariance matrix, $S=\hat{U}\hat{\Lambda}\hat{U}^T$ the spectral decomposition of $S$ where $\hat{\Lambda}=\mathrm{diag}(\hat{\lambda}_1,\ldots,\hat{\lambda}_p)$ and $\hat{U}$ is a orthogonal matrix.
    
    Let $u_i$ be the $i$th column of $U$, $1\leq i\leq p$. And denote $U=(V,\tilde{V})$, where $V$ is the first $r$ columns of $U$ and $\tilde{V}$ is the larst $p-r$ columns of $U$. Similarly define $\hat{u}_i$, $\hat{V}$ and $\hat{\tilde{V}}$.

The proof is similar to Dan Shen's paper.

    Let $Z\overset{def}{=}(Z_1,\ldots,Z_n)\overset{def}{=}{(\tilde{Z}_1^T,\ldots,\tilde{Z}_p^T)}^T$ and $\tilde{Z}_{(1)}={(\tilde{Z}_1^T,\ldots,\tilde{Z}_r^T)}^T$, $\tilde{Z}_{(2)}={(\tilde{Z}_{r+1}^T,\ldots,\tilde{Z}_p^T)}^T$.
    Let $\Lambda_{(1)}=\mathrm{diag}(\lambda_1,\ldots,\lambda_r)$ and $\Lambda_{(2)}=\mathrm{diag}(\lambda_{r+1},\ldots,\lambda_p)$. Define $\hat{\Lambda}_{(1)}$ and $\hat{\Lambda}_{(2)}$ in a similar way.
    Deote by $J$ the $p\times p$ matrix with all elements equal to $1$. Then
    \begin{equation*}
        S=\frac{1}{n}U\Lambda^{1/2}Z(I-\frac{1}{n}J)Z^T \Lambda^{1/2} U^T.
    \end{equation*}
    The positive eigenvalues of $S$ are the same as those of  $S^*=\frac{1}{n}(I-\frac{1}{n}J)Z^T \Lambda Z (I-\frac{1}{n}J)$, the dual matrix of $S$.


\begin{theorem}
    Suppose Assumption~\ref{PCAassump} holds and $p/n\to \infty$.
    For $1\leq i\leq r$, we have:

    If  $\frac{p}{n\lambda_i}\to 0$, then
    $\frac{\hat{\lambda}_i}{\lambda_i}\xrightarrow{a.s.} 1$.
    If  $\frac{p}{n\lambda_i}\to \infty$, then
    $\hat{\lambda}_i\asymp\frac{p}{n}$ almost surely.
\end{theorem}
\begin{proof}
    Let
    \begin{equation*}
        S^*=\frac{1}{n}(I-\frac{1}{n}J)\tilde{Z}_{(1)}^T \Lambda_{(1)} \tilde{Z}_{(1)} (I-\frac{1}{n}J)+\frac{1}{n}(I-\frac{1}{n}J)\tilde{Z}_{(2)}^T \Lambda_{(2)} \tilde{Z}_{(2)} (I-\frac{1}{n}J)\overset{def}{=}A+B.
    \end{equation*}

    For $1\leq i\leq r$, we prove the $i$th sample eigenvalue is consistent. By Corollary~\ref{WeylCor}, we have
    \begin{equation*}
        \begin{aligned}
            \frac{\lambda_i(A)}{\lambda_i}&\leq \frac{1}{\lambda_i}\lambda_{\max}\Big(\frac{1}{n}(I-\frac{1}{n}J)\tilde{Z}_{(1)}^T \mathrm{diag}(\underbrace{0,\ldots,0}_{i-1},\underbrace{\lambda_i,\ldots,\lambda_i}_{r-i+1}) \tilde{Z}_{(1)} (I-\frac{1}{n}J)\Big)\\
            &\leq \lambda_{\max}\Big(\frac{1}{n} \tilde{Z}_{(1)} (I-\frac{1}{n}J)\tilde{Z}_{(1)}^T \Big)\\
        \end{aligned}
    \end{equation*}
    But $\frac{1}{n}\tilde{Z}_{(1)} (I-\frac{1}{n}J)\tilde{Z}_{(1)}^T\xrightarrow{a.s.}I_r$ by law of large number. It follows that the right hand side tends to $1$ almost surely. On the other hand, by Weyl's inequility, we have

\begin{equation*}
    \begin{aligned}
        \frac{\lambda_i(A)}{\lambda_i}&\geq
        \frac{1}{\lambda_i}\lambda_{i}\Big(\frac{1}{n}(I-\frac{1}{n}J)\tilde{Z}_{(1)}^T \mathrm{diag}(\underbrace{\lambda_i,\ldots,\lambda_i}_{i},\underbrace{0,\ldots,0}_{r-i}) \tilde{Z}_{(1)} (I-\frac{1}{n}J)\Big).\\
    \end{aligned}
\end{equation*}
    The right hand side tends to $1$ almost surely. As a result, $\lambda_i(A)/\lambda_i$ tends to $1$ almost surely.


    By Weyl's inequality, we have that
    \begin{equation}\label{PCA19}
        \frac{\max(\lambda_i(A),\lambda_i(B))}{\lambda_i}\leq\frac{\lambda_i(S)}{\lambda_i}\leq \frac{\lambda_i(A)}{\lambda_i} +\frac{\lambda_{\max}(B)}{\lambda_i}.
    \end{equation}
    By Bai Yin's law, $\frac{n}{p}\lambda_{i}(B)\asymp 1$ almost surely for $1\leq i\leq n-1$. It follows that 
    \begin{equation}\label{PCA20}
        \frac{\lambda_{i}(B)}{\lambda_i}\asymp\frac{p}{n\lambda_i}
        \quad \textrm{and}\quad
        \frac{\lambda_{\max}(B)}{\lambda_i}\asymp\frac{p}{n\lambda_i},
    \end{equation}
    where $1\leq i\leq n-1$. The theorem follows from~\eqref{PCA19} and~\eqref{PCA20}.
\end{proof}



\begin{theorem}
    Suppose Assumption~\ref{PCAassump} holds and $p/n\to \infty$.
    If  $\frac{p}{n\lambda_r}\to 0$, then almost surely we have

    %\begin{equation}\label{PCAINeedThis}
    %    \mathrm{tr}\tilde{V}^T \hat{V}\Lambda_{(1)}\hat{V}^T \tilde{V}\asymp \frac{p}{n}
    %\end{equation}
%and
    \begin{equation}\label{PCAtheorem101}
        \|\hat{V}\hat{V}^T-VV^T\|^2_F\asymp\frac{p}{n\lambda_r }.
    \end{equation}
    If  $\frac{p}{n\lambda_r}\to \infty$, then
    \begin{equation}\label{PCAtheorem102}
        r-\frac{1}{2}\|\hat{V}\hat{V}^T-VV^T\|^2_F=O_{a.s.}(\frac{n\lambda_1}{p}).
    \end{equation}
\end{theorem}


\begin{proof}

    Since
    \begin{equation*}
        S=\hat{U}\hat{\Lambda}\hat{U}^T=
        \frac{1}{n}U\Lambda^{1/2}Z(I-\frac{1}{n}J)Z^T \Lambda^{1/2} U^T,
    \end{equation*}
    we have
    \begin{equation}\label{crucialEqInPCA1}
        \Lambda^{-1/2}U^T \hat{U}\hat{\Lambda}\hat{U}^T U\Lambda^{-1/2}=
        \frac{1}{n}Z(I-\frac{1}{n}J)Z^T. 
    \end{equation}
    First, we prove~\eqref{PCAtheorem101}.
    It follows from~\eqref{crucialEqInPCA1} that
    \begin{equation}\label{crucialEqInPCA3}
        \Lambda^{-1/2}_{(2)}\tilde{V}^T \hat{U}\hat{\Lambda}\hat{U}^T \tilde{V}\Lambda^{-1/2}_{(2)}=
        \frac{1}{n}\tilde{Z}_{(2)}(I-\frac{1}{n}J)\tilde{Z}_{(2)}^T.
    \end{equation}

    The left hand side of~\eqref{crucialEqInPCA3} equals to $C+D$, where 
    $C= \Lambda^{-1/2}_{(2)}\tilde{V}^T \hat{V}\hat{\Lambda}_{(1)}\hat{V}^T \tilde{V}\Lambda^{-1/2}_{(2)}$ 
    and 
    $D= \Lambda^{-1/2}_{(2)}\tilde{V}^T \hat{\tilde{V}}\hat{\Lambda}_{(2)}\hat{\tilde{V}}^T \tilde{V}\Lambda^{-1/2}_{(2)}$.
    Note that $\mathrm{Rank}(C)=r$, $\mathrm{Rank}(D)=n-1-r$ and $\mathrm{Rank}(C+D)=n-1$. By Bai Yin's law, we have that
    \begin{equation*}
        \lambda_1\big(\frac{1}{p}\tilde{Z}_{(2)}(I-\frac{1}{n}J)\tilde{Z}_{(2)}^T\big)\to 1,\quad
        \lambda_{n-1}\big(\frac{1}{p}\tilde{Z}_{(2)}(I-\frac{1}{n}J)\tilde{Z}_{(2)}^T\big)\to 1\quad  a.s..
    \end{equation*}
    By Lemma~\ref{lemmaRankLim}, $\lambda_{1}(C)\xrightarrow{a.s.}1$ and $\lambda_{r}(C)\xrightarrow{a.s.}1$. It follows that
    \begin{equation}\label{PCA27eq}
        \frac{n}{p}\hat{\Lambda}_{(1)}^{1/2}\hat{V}^T \tilde{V}\Lambda^{-1}_{(2)}\tilde{V}^T \hat{V}\hat{\Lambda}_{(1)}^{1/2}\xrightarrow{a.s.} I_r.
    \end{equation}
    When $\frac{p}{n\lambda_r}\to 0$, $\hat{\lambda}_i$'s are ratio consistent for $1\leq i\leq r$. That is, $\Lambda_{(1)}^{-1}\hat{\Lambda}_{(1)}\to I_r$ almost surely. Then it follows from~\eqref{PCA27eq} that
    \begin{equation}
        \frac{n}{p}\Lambda_{(1)}^{1/2}\hat{V}^T \tilde{V}\Lambda^{-1}_{(2)}\tilde{V}^T \hat{V}\Lambda_{(1)}^{1/2}\xrightarrow{a.s.} I_r.
    \end{equation}

    %Since $\Lambda_{(2)}$ is bounded from below and above,~\eqref{PCAINeedThis} holds.
    Notice that
    \begin{equation*}
        \begin{aligned}
         \frac{n}{p}   \mathrm{tr}\big(\Lambda_{(1)}^{1/2}\hat{V}^T \tilde{V}\Lambda^{-1}_{(2)}\tilde{V}^T \hat{V}\Lambda_{(1)}^{1/2}\big)&\geq
          \frac{n}{p}  \lambda_r\mathrm{tr}\big(\hat{V}^T \tilde{V}\Lambda^{-1}_{(2)}\tilde{V}^T \hat{V}\big)
            \geq
          \frac{n}{p}  e_r^T \hat{\Lambda}_{(1)}^{1/2}\hat{V}^T\tilde{V}\Lambda_{(2)}^{-1}\tilde{V}^T\tilde{V}\hat{\Lambda}_{(1)}^{1/2}e_1
        \end{aligned}
    \end{equation*}
    where $e_r=(\underbrace{0,\ldots,0}_{r-1},1)$. It follows that the medium term is bounded above and below asymptotically. Notice that
    \begin{equation*}
        \begin{aligned}
            \frac{n}{p}\lambda_r\mathrm{tr}\big(\hat{V}^T \tilde{V}\Lambda^{-1}_{(2)}\tilde{V}^T \hat{V}\big)
            &\asymp
            \frac{n}{p}\lambda_r\mathrm{tr}\big(\hat{V}^T \tilde{V}\tilde{V}^T \hat{V}\big)
            =\frac{n}{p}\lambda_r\frac{1}{2}\|VV^T -\hat{V}\hat{V}^T\|^2_F.
        \end{aligned}
    \end{equation*}
     Therefore
    $\|VV^T -\hat{V}\hat{V}^T\|^2_F\asymp \frac{p}{n\lambda_r}$ almost surely.

    Then we prove~\eqref{PCAtheorem102}. It follows from~\eqref{crucialEqInPCA1} that
    \begin{equation}\label{crucialEqInPCA2}
        \Lambda^{-1/2}_{(1)}V^T \hat{U}\hat{\Lambda}\hat{U}^T V\Lambda^{-1/2}_{(1)}=
        \frac{1}{n}\tilde{Z}_{(1)}(I-\frac{1}{n}J)\tilde{Z}_{(1)}^T\xrightarrow{a.s.} I_{r}.
    \end{equation}
But
    \begin{equation}
        \begin{aligned}
        \mathrm{tr}\big(\Lambda^{-1/2}_{(1)}V^T \hat{U}\hat{\Lambda}\hat{U}^T V\Lambda^{-1/2}_{(1)}\big)
            &\geq
        \mathrm{tr}\big(\Lambda^{-1/2}_{(1)}V^T \hat{V}\hat{\Lambda}_{(1)}\hat{V}^T V\Lambda^{-1/2}_{(1)}\big)
            \\
            &\geq
            \frac{\hat{\lambda}_r}{\lambda_1}\Big(r-\frac{1}{2}\|\hat{V}\hat{V}^T-VV^T\|^2_F\Big).
        \end{aligned}
    \end{equation}
When $\frac{p}{n\lambda_r}\to \infty$, $\hat{\lambda}_r\asymp p/n$.Then~\eqref{PCAtheorem102} holds.

\end{proof}



Suprisingly, from our proof we can see that the error of PCA can be estimated well!
