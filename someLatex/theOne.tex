\begin{proof}[\textbf{Proof Of Theorem 5}]
    The proof is based on Theorem~\ref{myPanpan} and procedure runs almost the same. The notation is parallel to Theorem~\ref{myPanpan}. Denote by $O_{m\times n}$ the $m\times n$ matrix with all elements equal to $0$. First we consider term
\begin{equation*}
    R_1'=\frac{n_1}{\sqrt{p}}(\bar{X}_1^T VV^T\bar{X}_1-\bar{X}_1^T \hat{V}\hat{V}^T\bar{X}_1),
\end{equation*}
and
\begin{equation*}
    R_2'=\frac{1}{n_1\sqrt{p}}\sum_{i=1}^{n_1}(X_{1i}^T VV^T X_{1i}-X_{1i}^T \hat{V}\hat{V}^T X_{1i}).
\end{equation*}
We note that $VV^T\geq V_1 V_1^T$. Define $V_{2\ominus 1}V_{2\ominus 1}^T=VV^T-V_1 V_1^T$. From the fact that $rank(V)\leq r_1+r_2$ and $V_{2\ominus 1}^T V_1=O_{r_2\times r_1}$, it can easily deduced that $\frac{n_1}{\sqrt{p}}\bar{X}_1^T VV^T\bar{X}_1-\frac{n_1}{\sqrt{p}}\bar{X}_1^T V_1V_1^T\bar{X}_1\xrightarrow{P}0$. Hence to proof $R_1'\xrightarrow{P}0$, we only need to consider
\begin{equation*}
\frac{n_1}{\sqrt{p}}(\bar{X}_1^T V_1V_1^T \bar{X}_1-\bar{X}_1^T \hat{V}\hat{V}^T \bar{X}_1),
\end{equation*}
which can be further decomposed into three parts

    \begin{equation*}
        Q_1'=\frac{n_1}{\sqrt{p}}\bar{X}_1^T V_1(I_{r_1}-V_1^T\hat{V}\hat{V}^T V_1)V_1^T \bar{X}_1,
    \end{equation*}
    \begin{equation*}
        Q_2'=\frac{n_1}{\sqrt{p}}\bar{X}_1^T \tilde{V}_1\tilde{V}_1^T\hat{V}\hat{V}^T V_1V_1^T \bar{X}_1,
    \end{equation*}
    \begin{equation*}
        Q_3'=\frac{n_1}{\sqrt{p}}\bar{X}_1^T \tilde{V}_1\tilde{V}_1^T\hat{V}\hat{V}^T \tilde{V}_1\tilde{V}_1^T \bar{X}_1.
    \end{equation*}
We note that $Q_1'\leq n_1\bar{X}_1^T V_1(I_{r_1}-V_1^T\hat{V}_1\hat{V}_1^T V_1)V_1^T \bar{X}_1/\sqrt{p}$ which converges to $0$ by Theorem~\ref{myPanpan}.  $Q_2'$ can be written as

    \begin{equation*}
    \begin{aligned}
        Q_2'=\frac{n_1}{\sqrt{p}}\bar{X}_1^T \tilde{V}_1\tilde{V}_1^T\hat{V}_1\hat{V}_1^T V_1V_1^T \bar{X}_1
        +\frac{n_1}{\sqrt{p}}\bar{X}_1^T \tilde{V}_1\tilde{V}_1^T\hat{V}_{2\ominus 1}\hat{V}_{2\ominus 1}^T V_1V_1^T \bar{X}_1.
        \end{aligned}
    \end{equation*}
    The first term converges to $0$ by Theorem~\ref{myPanpan}.  For the second term we have
    \begin{equation*}
    \begin{aligned}
&\frac{n_1}{\sqrt{p}}\bar{X}_1^T \tilde{V}_1\tilde{V}_1^T\hat{V}_{2\ominus 1}\hat{V}_{2\ominus 1}^T V_1V_1^T \bar{X}_1\\
&\leq
\sqrt{\frac{n_1}{\sqrt{p}}\bar{X}_1^T \tilde{V}_1\tilde{V}_1^T\hat{V}_{2\ominus 1}\hat{V}_{2\ominus 1}^T \tilde{V}_1\tilde{V}_1^T \bar{X}_1}\sqrt{\frac{n_1}{\sqrt{p}}\bar{X}_1^T {V}_1{V}_1^T\hat{V}_{2\ominus 1}\hat{V}_{2\ominus 1}^T V_1V_1^T \bar{X}_1}.
    \end{aligned}
    \end{equation*}
The first term converges to $0$ because $\hat{V}_{2\ominus 1}^T \tilde{V}_1\tilde{V}_1^T \bar{X}_1|(S_1,S_2)\sim N(0,\frac{\sigma_1^2}{n_1}\hat{V}_{2\ominus 1}^T \tilde{V}_1\tilde{V}_1^T\hat{V}_{2\ominus 1})$ whose conditional variance is dominated by $\frac{\sigma^2_1}{n_1}$. And for the second term, we have
\begin{equation*}
\begin{aligned}
\frac{n_1}{\sqrt{p}}\bar{X}_1^T {V}_1{V}_1^T\hat{V}_{2\ominus 1}\hat{V}_{2\ominus 1}^T V_1V_1^T \bar{X}_1
&\leq
\frac{n_1}{\sqrt{p}}\bar{X}_1^T {V}_1{V}_1^T\hat{\tilde{V}}_{1}\hat{\tilde{V}}_{1}^T V_1V_1^T \bar{X}_1\\
&\leq
\frac{n_1}{\sqrt{p}}\|{V}_1^T\hat{\tilde{V}}_{1}\hat{\tilde{V}}_{1}^T V_1\|\|V_1^T \bar{X}_1\|^2\\
&\leq O(1)\frac{n_1}{\sqrt{p}}\|V_1 V_1^T -\hat{V}_1\hat{V}_1^T\|^2_F \frac{p^{\beta}}{n_1}\\
&=O(\frac{\sqrt{p}}{n_1})\xrightarrow{P}0.
\end{aligned}
\end{equation*}
$Q_3'=
\frac{n_1}{\sqrt{p}}\bar{X}_1^T \tilde{V}_1\tilde{V}_1^T\hat{V}_{2\ominus 1}\hat{V}_{2\ominus 1}^T \tilde{V}_1\tilde{V}_1^T \bar{X}_1+
\frac{n_1}{\sqrt{p}}\bar{X}_1^T \tilde{V}_1\tilde{V}_1^T\hat{V}_{1}\hat{V}_{1}^T \tilde{V}_1\tilde{V}_1^T \bar{X}_1$
    converges to $0$ in probability by combining Theorem~\ref{myPanpan} and the argument above. Hence $R_1'\xrightarrow{P}0$.

To prove $R_2'\xrightarrow{P}0$, we note that
 $\frac{1}{n_1\sqrt{p}}\sum_{i=1}^{n_1}X_{1i}^T VV^T X_{1i}-
\frac{1}{n_1\sqrt{p}}\sum_{i=1}^{n_1}X_{1i}^T V_1V_1^T X_{1i}\xrightarrow{P}0$.
    Hence it suffices to consider the following three parts
    \begin{equation*}
        W_1'=\frac{1}{n_1\sqrt{p}}\sum_{i=1}^{n_1}X_{1i}^T V_1(I_{r_1}-V_1^T\hat{V}\hat{V}^T V_1)V_1^T X_{1i},
    \end{equation*}
    \begin{equation*}
        W_2'=\frac{1}{n_1\sqrt{p}}\sum_{i=1}^{n_1}X_{1i}^T \tilde{V}_1\tilde{V}_1^T\hat{V}\hat{V}^T V_1V_1^T X_{1i}
    \end{equation*}
    and
    \begin{equation*}
        W_3'=\frac{1}{n_1\sqrt{p}}\sum_{i=1}^{n_1}X_{1i}^T \tilde{V}_1\tilde{V}_1^T\hat{V}\hat{V}^T \tilde{V}_1\tilde{V}_1^T X_{1i}.
    \end{equation*}
$W_1'\xrightarrow{P}0$ by $\hat{V}\hat{V}^T\leq \hat{V}_1\hat{V}_1^T$ and Theorem~\ref{myPanpan}.
\begin{equation*}
\begin{aligned}
        W_2'=\frac{1}{n_1\sqrt{p}}\sum_{i=1}^{n_1}X_{1i}^T \tilde{V}_1\tilde{V}_1^T\hat{V}_1\hat{V}_1^T V_1V_1^T X_{1i}+\frac{1}{n_1\sqrt{p}}\sum_{i=1}^{n_1}X_{1i}^T \tilde{V}_1\tilde{V}_1^T\hat{V}_{2\ominus 1}\hat{V}_{2\ominus 1}^T V_1V_1^T X_{1i}.
\end{aligned}
\end{equation*}
The first term converges to $0$ by Theorem~\ref{myPanpan}. As for the second term, we can use the similar technique as we deal with $W_2$ and the fact $\hat{V}_{2\ominus 1}\hat{V}_{2\ominus 1}^T\leq \hat{\tilde{V}}\hat{\tilde{V}}^T$.
The proof of $W_3'\xrightarrow{P}0$ runs the same as the case of $W_3$.

Finally it suffices to proof following four terms converges to $0$ in probability

    \begin{equation*}
        M_1'=\frac{n_1}{\sqrt{p}}\bar{X}_1^T V(I_{r_1+r_2}-V^T\hat{V}\hat{V}^T V)V^T \bar{X}_2,
    \end{equation*}
    \begin{equation*}
        M_2'=\frac{n_1}{\sqrt{p}}\bar{X}_1^T \tilde{V}\tilde{V}^T\hat{V}\hat{V}^T VV^T \bar{X}_2,
    \end{equation*}
    \begin{equation*}
        M_3'=\frac{n_1}{\sqrt{p}}\bar{X}_1^T {V}{V}^T\hat{V}\hat{V}^T \tilde{V}\tilde{V}^T \bar{X}_2,
    \end{equation*}
    and
    \begin{equation*}
        M_4'=\frac{n_1}{\sqrt{p}}\bar{X}_1^T \tilde{V}\tilde{V}^T\hat{V}\hat{V}^T \tilde{V}\tilde{V}^T \bar{X}_2.
    \end{equation*}

We note that 
\begin{equation*}
M_1'\leq\sqrt{\frac{n_1}{\sqrt{p}}\bar{X}_1^T V(I_{r_1+r_2}-V^T\hat{V}\hat{V}^T V)V^T \bar{X}_1}\sqrt{\frac{n_1}{\sqrt{p}}\bar{X}_1^T V(I_{r_1+r_2}-V^T\hat{V}\hat{V}^T V)V^T \bar{X}_2}.
\end{equation*}
Hence to prove $M_1'\xrightarrow{P} 0$, it suffices to prove $\frac{n_1}{\sqrt{p}}\bar{X}_1^T V(I_{r_1+r_2}-V^T\hat{V}\hat{V}^T V)V^T \bar{X}_1\xrightarrow{P}0$. 
We note that
\begin{equation*}
\begin{aligned}
&\frac{n_1}{\sqrt{p}}\bar{X}_1^T V(I_{r_1+r_2}-V^T\hat{V}\hat{V}^T V)V^T \bar{X}_1\\
&=(\frac{n_1}{\sqrt{p}}\bar{X}_1^T V_1V_1^T \bar{X}_1-\frac{n_1}{\sqrt{p}}\bar{X}_1^T V_1V_1^T \hat{V}\hat{V}^T V_1 V_1^T \bar{X}_1)
\\
&+\frac{n_1}{\sqrt{p}}\bar{X}_1^T V_{2\ominus 1}V_{2\ominus 1}^T \bar{X}_1-
2\frac{n_1}{\sqrt{p}}\bar{X}_1^T V_{2\ominus 1}V_{2\ominus 1}^T \hat{V}\hat{V}^T V_1 V_1^T \bar{X}_1
\\
&-\frac{n_1}{\sqrt{p}}\bar{X}_1^T V_{2\ominus 1}V_{2\ominus 1}^T \hat{V}\hat{V}^T V_{2\ominus 1}V_{2\ominus 1}^T\bar{X}_1
\\
&=L_1+L_2-2L_3-L_4.
\end{aligned}
\end{equation*}
Check that $L_1=Q_1'$. It's also clear that $L_2\xrightarrow{P}0$. To deal with $L_3$, we note that
\begin{equation}\label{xiaojiqiao1}
\begin{aligned}
    &|\frac{n_1}{\sqrt{p}}\bar{X}_1^T V_{2\ominus 1}V_{2\ominus 1}^T \hat{V}_1\hat{V}_1^T V_1 V_1^T \bar{X}_1|
\\
    &\leq \frac{n_1}{\sqrt{p}}\|V_{2\ominus 1}^T\bar{X}_1\|\|V^T_{2\ominus 1}\hat{V}_1\|_F\|\hat{V}_1^T V_1{V}_1^T\bar{X}_1\|\\
&\leq
\frac{n_1}{\sqrt{p}}O(\frac{1}{\sqrt{n_1}})\sqrt{tr(\hat{V}_1^T V_{2\ominus 1}V_{2\ominus 1}^T\hat{V}_1)}O(\frac{\sqrt{p^{\beta}}}{\sqrt{n_1}}).
\end{aligned}
\end{equation}
But 
\begin{equation*}
tr(\hat{V}_1^T V_{2\ominus 1}V_{2\ominus 1}^T\hat{V}_1)\leq tr(\hat{V}_1^T \tilde{V}_1\tilde{V}_1^T\hat{V}_1)=\|V_1V_1^T- \hat{V}_1\hat{V}_1^T\|_F^2=O_P(\frac{p}{p^{\beta}n_1}).
\end{equation*}
And
\begin{equation}\label{xiaojiqiao2}
\begin{aligned}
|\frac{n_1}{\sqrt{p}}\bar{X}_1^T V_{2\ominus 1}V_{2\ominus 1}^T \hat{V}_{2\ominus 1}\hat{V}_{2\ominus 1}^T V_1 V_1^T \bar{X}_1|
&\leq \frac{n_1}{\sqrt{p}}\|V^T_{2\ominus 1}\hat{V}_{2\ominus 1}V_{2\ominus 1}^T\bar{X}_1\|\|\hat{V}_{2\ominus 1}^T V_1\|_F \|{V}_1^T\bar{X}_1\|\\
&\leq
\frac{n_1}{\sqrt{p}}O(\frac{1}{\sqrt{n_1}})\sqrt{tr(\hat{V}_{2\ominus 1}^T V_{1}V_{1}^T\hat{V}_{2\ominus 1})}O(\frac{\sqrt{p^{\beta}}}{\sqrt{n_1}}).
\end{aligned}
\end{equation}
But
\begin{equation*}
tr(\hat{V}_{2\ominus 1}^T V_{1}V_{1}^T\hat{V}_{2\ominus 1})\leq tr(\hat{\tilde{V}}_1^T {V}_1{V}_1^T\hat{\tilde{V}}_1)=\|V_1V_1^T- \hat{V}_1\hat{V}_1^T\|_F^2=O_P(\frac{p}{p^{\beta}n_1}).
\end{equation*}
Therefore $L_3\xrightarrow{P}0$. And $L_4\xrightarrow{P}0$ for trival reason.
 $M_2'$ can be similarly treated by technique~\eqref{xiaojiqiao1} and~\eqref{xiaojiqiao2}. Since $\sqrt{n_1}\hat{V}^T\tilde{V}\tilde{V}^T\bar{X}_2$ is bounded in probability, we have $M_4'\xrightarrow{P} 0$, which completes the proof.

    \end{proof}
\begin{proof}[\textbf{Proof Of Theorem 6}]
    The theorem follows by the same method as Theorem~\ref{spaceEstimation}.
\end{proof}
